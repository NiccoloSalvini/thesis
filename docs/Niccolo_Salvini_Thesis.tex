% Options for packages loaded elsewhere
\PassOptionsToPackage{unicode}{hyperref}
\PassOptionsToPackage{hyphens}{url}
%
\documentclass[
  12pt,
  a4paper,
  oneside]{book}
\usepackage{lmodern}
\usepackage{setspace}
\usepackage{amssymb,amsmath}
\usepackage{ifxetex,ifluatex}
\ifnum 0\ifxetex 1\fi\ifluatex 1\fi=0 % if pdftex
  \usepackage[T1]{fontenc}
  \usepackage[utf8]{inputenc}
  \usepackage{textcomp} % provide euro and other symbols
\else % if luatex or xetex
  \usepackage{unicode-math}
  \defaultfontfeatures{Scale=MatchLowercase}
  \defaultfontfeatures[\rmfamily]{Ligatures=TeX,Scale=1}
\fi
% Use upquote if available, for straight quotes in verbatim environments
\IfFileExists{upquote.sty}{\usepackage{upquote}}{}
\IfFileExists{microtype.sty}{% use microtype if available
  \usepackage[]{microtype}
  \UseMicrotypeSet[protrusion]{basicmath} % disable protrusion for tt fonts
}{}
\makeatletter
\@ifundefined{KOMAClassName}{% if non-KOMA class
  \IfFileExists{parskip.sty}{%
    \usepackage{parskip}
  }{% else
    \setlength{\parindent}{0pt}
    \setlength{\parskip}{6pt plus 2pt minus 1pt}}
}{% if KOMA class
  \KOMAoptions{parskip=half}}
\makeatother
\usepackage{xcolor}
\IfFileExists{xurl.sty}{\usepackage{xurl}}{} % add URL line breaks if available
\IfFileExists{bookmark.sty}{\usepackage{bookmark}}{\usepackage{hyperref}}
\hypersetup{
  pdftitle={REST Scraping API for Real Estate data, a Spatial Bayesian modeling perspective with INLA},
  pdfauthor={Niccolò Salvini},
  hidelinks,
  pdfcreator={LaTeX via pandoc}}
\urlstyle{same} % disable monospaced font for URLs
\usepackage[left=4cm, right=3cm, top=3cm, bottom=3cm]{geometry}
\usepackage{color}
\usepackage{fancyvrb}
\newcommand{\VerbBar}{|}
\newcommand{\VERB}{\Verb[commandchars=\\\{\}]}
\DefineVerbatimEnvironment{Highlighting}{Verbatim}{commandchars=\\\{\}}
% Add ',fontsize=\small' for more characters per line
\usepackage{framed}
\definecolor{shadecolor}{RGB}{248,248,248}
\newenvironment{Shaded}{\begin{snugshade}}{\end{snugshade}}
\newcommand{\AlertTok}[1]{\textcolor[rgb]{0.94,0.16,0.16}{#1}}
\newcommand{\AnnotationTok}[1]{\textcolor[rgb]{0.56,0.35,0.01}{\textbf{\textit{#1}}}}
\newcommand{\AttributeTok}[1]{\textcolor[rgb]{0.77,0.63,0.00}{#1}}
\newcommand{\BaseNTok}[1]{\textcolor[rgb]{0.00,0.00,0.81}{#1}}
\newcommand{\BuiltInTok}[1]{#1}
\newcommand{\CharTok}[1]{\textcolor[rgb]{0.31,0.60,0.02}{#1}}
\newcommand{\CommentTok}[1]{\textcolor[rgb]{0.56,0.35,0.01}{\textit{#1}}}
\newcommand{\CommentVarTok}[1]{\textcolor[rgb]{0.56,0.35,0.01}{\textbf{\textit{#1}}}}
\newcommand{\ConstantTok}[1]{\textcolor[rgb]{0.00,0.00,0.00}{#1}}
\newcommand{\ControlFlowTok}[1]{\textcolor[rgb]{0.13,0.29,0.53}{\textbf{#1}}}
\newcommand{\DataTypeTok}[1]{\textcolor[rgb]{0.13,0.29,0.53}{#1}}
\newcommand{\DecValTok}[1]{\textcolor[rgb]{0.00,0.00,0.81}{#1}}
\newcommand{\DocumentationTok}[1]{\textcolor[rgb]{0.56,0.35,0.01}{\textbf{\textit{#1}}}}
\newcommand{\ErrorTok}[1]{\textcolor[rgb]{0.64,0.00,0.00}{\textbf{#1}}}
\newcommand{\ExtensionTok}[1]{#1}
\newcommand{\FloatTok}[1]{\textcolor[rgb]{0.00,0.00,0.81}{#1}}
\newcommand{\FunctionTok}[1]{\textcolor[rgb]{0.00,0.00,0.00}{#1}}
\newcommand{\ImportTok}[1]{#1}
\newcommand{\InformationTok}[1]{\textcolor[rgb]{0.56,0.35,0.01}{\textbf{\textit{#1}}}}
\newcommand{\KeywordTok}[1]{\textcolor[rgb]{0.13,0.29,0.53}{\textbf{#1}}}
\newcommand{\NormalTok}[1]{#1}
\newcommand{\OperatorTok}[1]{\textcolor[rgb]{0.81,0.36,0.00}{\textbf{#1}}}
\newcommand{\OtherTok}[1]{\textcolor[rgb]{0.56,0.35,0.01}{#1}}
\newcommand{\PreprocessorTok}[1]{\textcolor[rgb]{0.56,0.35,0.01}{\textit{#1}}}
\newcommand{\RegionMarkerTok}[1]{#1}
\newcommand{\SpecialCharTok}[1]{\textcolor[rgb]{0.00,0.00,0.00}{#1}}
\newcommand{\SpecialStringTok}[1]{\textcolor[rgb]{0.31,0.60,0.02}{#1}}
\newcommand{\StringTok}[1]{\textcolor[rgb]{0.31,0.60,0.02}{#1}}
\newcommand{\VariableTok}[1]{\textcolor[rgb]{0.00,0.00,0.00}{#1}}
\newcommand{\VerbatimStringTok}[1]{\textcolor[rgb]{0.31,0.60,0.02}{#1}}
\newcommand{\WarningTok}[1]{\textcolor[rgb]{0.56,0.35,0.01}{\textbf{\textit{#1}}}}
\usepackage{longtable,booktabs}
% Correct order of tables after \paragraph or \subparagraph
\usepackage{etoolbox}
\makeatletter
\patchcmd\longtable{\par}{\if@noskipsec\mbox{}\fi\par}{}{}
\makeatother
% Allow footnotes in longtable head/foot
\IfFileExists{footnotehyper.sty}{\usepackage{footnotehyper}}{\usepackage{footnote}}
\makesavenoteenv{longtable}
\usepackage{graphicx,grffile}
\makeatletter
\def\maxwidth{\ifdim\Gin@nat@width>\linewidth\linewidth\else\Gin@nat@width\fi}
\def\maxheight{\ifdim\Gin@nat@height>\textheight\textheight\else\Gin@nat@height\fi}
\makeatother
% Scale images if necessary, so that they will not overflow the page
% margins by default, and it is still possible to overwrite the defaults
% using explicit options in \includegraphics[width, height, ...]{}
\setkeys{Gin}{width=\maxwidth,height=\maxheight,keepaspectratio}
% Set default figure placement to htbp
\makeatletter
\def\fps@figure{htbp}
\makeatother
% Make links footnotes instead of hotlinks:
\DeclareRobustCommand{\href}[2]{#2\footnote{\url{#1}}}
\setlength{\emergencystretch}{3em} % prevent overfull lines
\providecommand{\tightlist}{%
  \setlength{\itemsep}{0pt}\setlength{\parskip}{0pt}}
\setcounter{secnumdepth}{5}
% \usepackage{color}
% \usepackage{fancyvrb}
% \newcommand{\VerbBar}{|}
% \newcommand{\VERB}{\Verb[commandchars=\\\{\}]}
% % Add ',fontsize=\small' for more characters per line
% \DefineVerbatimEnvironment{Highlighting}{Verbatim}{commandchars=\\\{\}}
% \usepackage{framed}
% \definecolor{shadecolor}{RGB}{248,248,248}
% \newenvironment{Shaded}{\begin{snugshade}}{\end{snugshade}}
\usepackage{booktabs}
\usepackage{pdfpages}
\usepackage{amsthm}
\usepackage{tabu}
\usepackage{booktabs}
\usepackage{longtable}
\usepackage{array}
\usepackage{multirow}
\usepackage{wrapfig}
\usepackage{float}
\usepackage{colortbl}
\usepackage{pdflscape}
\usepackage{algorithm2e}
\usepackage{tabu}
\usepackage{threeparttable}
\usepackage{threeparttablex}
\usepackage[normalem]{ulem}
\usepackage{makecell}
\usepackage{xcolor}
\makeatletter
\def\thm@space@setup{%
  \thm@preskip=8pt plus 2pt minus 4pt
  \thm@postskip=\thm@preskip
}
\makeatother
\let\oldmaketitle\maketitle
\AtBeginDocument{\let\maketitle\relax}
\usepackage{booktabs}
\usepackage{longtable}
\usepackage{array}
\usepackage{multirow}
\usepackage{wrapfig}
\usepackage{float}
\usepackage{colortbl}
\usepackage{pdflscape}
\usepackage{tabu}
\usepackage{threeparttable}
\usepackage{threeparttablex}
\usepackage[normalem]{ulem}
\usepackage{makecell}
\usepackage{xcolor}
\usepackage[]{natbib}
\bibliographystyle{apalike}

\title{REST Scraping API for Real Estate data, a Spatial Bayesian modeling perspective with INLA}
\author{\href{https://niccolosalvini.netlify.app/}{Niccolò Salvini}}
\date{Lastest build: 14 dicembre, 2020}

\usepackage{amsthm}
\newtheorem{theorem}{Theorem}[chapter]
\newtheorem{lemma}{Lemma}[chapter]
\newtheorem{corollary}{Corollary}[chapter]
\newtheorem{proposition}{Proposition}[chapter]
\newtheorem{conjecture}{Conjecture}[chapter]
\theoremstyle{definition}
\newtheorem{definition}{Definition}[chapter]
\theoremstyle{definition}
\newtheorem{example}{Example}[chapter]
\theoremstyle{definition}
\newtheorem{exercise}{Exercise}[chapter]
\theoremstyle{remark}
\newtheorem*{remark}{Remark}
\newtheorem*{solution}{Solution}
\begin{document}
\maketitle

\includepdf[pages={1}, scale=1]{cover.pdf}
\newpage

\let\maketitle\oldmaketitle
\maketitle

{
\setcounter{tocdepth}{2}
\tableofcontents
}
\listoftables
\listoffigures
\setstretch{1.5}
\hypertarget{intro}{%
\chapter{Introduction}\label{intro}}

Trento:

\begin{itemize}
\tightlist
\item
  Argomento
\item
  Problema
\item
  Obiettivi
\item
  Metodo
\item
  Struttura della tesi
\end{itemize}

Main themes:

\begin{itemize}
\tightlist
\item
  Research Question
\item
  Milan Real Estate Controversies in relation to research question
\item
  why the API (perchè mi mancano i dati e perchè è il futuro)
\item
  Open Data discussion personal hope of data sharing and benefits from oper source
\item
  Why a Bayesian approach
\item
  Why INLA
\end{itemize}

As a general discussion technologies implied can be thought as the distance between a service running locally on a laptop and something that it can actually be put into production, shared among company stakeholders, solving business related problems. When such technologies are applied data scientist and interlocutors gradually close the gap. Insights are better communicated, data is up-to-date and automation can save time. Nonetheless when the infrastructure is structured with vision then integrating or substituting existing technologies is not trivial. Anyway technologies can not be always embedded because they might be exclusively designed to work only on certain back ends, therefore some choices are not into discussion. With foresight RStudio by setting future-oriented guidelines has spent a lot of effort giving its users an easy, integrated and interconnected environment. By that it is meant that the RStudio community has tried to either integrate or open the possibility to a number of technologies that fill the blanks in their weaker parts. On top of many, an entire package has been dedicated to democratize REST APIs (Plumber \citep{plumber}). As a further example developers in RStudio have created an entire new paradigm i.e.~Shiny \citep{shiny}, a popular web app development package, that enforces the developer to have front-end and back-end technologies tied up in the same IDE. They also added performance monitoring and optimization packages that are fitted into shiny such as shinytest {[}metti tag{]} and shinyloadtest {[}metti tag{]} to simulate sessions and verify network traffic congestion.

\hypertarget{scraping}{%
\chapter{Web Scraping}\label{scraping}}

The following chapter covers advanced techniques for web scraping in R and related main challenges with a focus on the immobiliare.it case. A quicker and less sophisticated \emph{scraping workflow} is proposed that integrates url reverse engineering inside the proper scraping part, this removes frictions related to inner scraping complexities.
That means instead of crawling the entire website and then searching for keywords, it retrieves a cluster of urls in the sitemaps given some parameter arguments and then applies custom made proper scraping on this closed set. By doing that the scraper takes advantage of the url clean structure and adopts an ``inverse url composition'' methodology. At first url sematic is reverse engineered so that the sitemap is made explicit and urls can be freely manipulated at will. Secondly links that are children of the manipualted urlss are gathered and collected into a list. Up to this point each single link belonging to the list can be directly targeted by a number of scraping functions, with that said the focus can shift to the proper scraping part. An example of scraping function with \texttt{rvest} \citet{rvest} is presented outlining the algorithm adopted to search for the content within immobiliare.it HTML/CSS. The skeleton of the algorithm is then reproduced for all the other functions that shares the same CSS query location. For all the functions that necessitate other CSS query as input nodes the search strategy is exclusively built on top of them.
Scraping common shared best practices are applied from the \emph{web server} point of view, this is taken care by kindly asking for permission and sending delayed requests rate. As well as from the the \emph{web client} point of view by preventing scraping discontinuity caused by server blocks through \emph{User Agent} pool \emph{rotation} and \emph{fail dealers}.
\emph{Parallel} execution is carried out since data becomes obsolete very fast, and so happens to the analysis that relied on those data. A run time Parallel scraping benchmark is presented for two different back end options \texttt{future} \citet{future} and \texttt{doParallel} \citet{doParallel}, along with their respective two parallel looping constructors, \texttt{furrr} \citet{furrr} and \texttt{foreach} \citet{foreach}. Both of the two combination have showed similar results, nevertheless the former offers has a more \{Tidiverse\} orientation and a more comfortable debugging experience. Furthermore an overview of the still open challenges and improvements is given with the hope that the effort put into this project might be extended or integrated. In the end legal profiles are addressed comparing results and popular legal case studies.

\hypertarget{a-gentle-introduction-on-web-scraping}{%
\section{A Gentle Introduction on Web Scraping}\label{a-gentle-introduction-on-web-scraping}}

\begin{definition}[Scraping]
\protect\hypertarget{def:scraping}{}{\label{def:scraping} \iffalse (Scraping) \fi{} }Web Scraping is a technique aimed at extracting unstructured data from static or dynamic internet web pages and collecting it in a structured way.
Automated data collection, web extraction, web crawling, or web data mining are often used as synonyms to web scraping.
\end{definition}

The World Wide Web (WWW or just the web") data accessible today is calculated in zettabytes (Cisco Systems, 2017 \emph{miss lit}) (1 zettabyte = \(10^{21}\) bytes). This huge volume of data provides a wealth of resources for researchers and practitioners to obtain new insights in real time regarding individuals, organizations, or even macro-level socio-technical phenomena \href{https://www.researchgate.net/publication/328513839_Tutorial_Web_Scraping_in_the_R_Language}{miss lit}. Unsurprisingly, researchers of Information Systems are increasingly turning to the internet for data that can address their research questions.
Taking advantage of the immense web data also requires a programmatic approach \href{https://www.researchgate.net/publication/328513839_Tutorial_Web_Scraping_in_the_R_Language}{miss lit} and a strong foundation in different web technologies.
Besides the large amount of web access and data to be analyzed, there are three rather common problems related to big web data: variety, velocity, and veracity (Goes,z 2014; Krotov \& Silva, 2018 \emph{miss lit}), each of which exposes a singular aspect of scraping and constitutes some of the challenges fronted in later chapters.
\emph{Variety} mainly accounts the most commonly used mark-up languages on the web used for content creation and organization such as such as HTML \citep{html_2020}, CSS \citep{css_2020}, and XML \emph{miss lit}. Sometimes Javascript \emph{miss lit} components are also embedded into websites, dedicated parsers are required in these contexts. Starting from that scraping requires at least to know the ropes of these technologies which are also assumed in this analysis.
\emph{Velocity}: web data are in continuous flow status: it is created in real time, modified and changed continuously. This massively impacts the analysis that relies on those data which, as times passes, becomes obsolete. However it also suggests the speed and pace at which data should be collected. From one hand data should be gathered as quicker as possible so that analysis or softwares are up-to-date, section \ref{parallelscraping}. From the other this should happen in any case by constraining speed to common shared scraping best practices, which require occasional rests in requesting information. The latter issue is faced later in section \ref{best-practices}.
\emph{Veracity}: Internet data quality and usability are still surrounded by confusion. A researcher can never entirely be sure if the data he wants are available on the internet and if the data are sufficiently accurate to be used in analysis. While for the former point data can be, from a theoretical perspective, accurately selected it can not be by any means predicted to exist. This is a crucial aspects of scraping, it should take care of dealing with the possibility to fail, in other words to be lost. The latter point that points to data quality is also crucial and it is assessed by a thoughtful market analysis that mostly assures the importance of immobiliare.it as a top player in italian real estate. As a consequence data coming from reliable source is also assumed to be reliable.
Furthermore web scraping can be mainly applied in two common forms as in \href{https://www.tandfonline.com/doi/full/10.1080/10691898.2020.1787116}{miss lit}: the first is browser scraping, where extractions takes place through HTML/XML parser with regular expression matching from the website's source code. The second uses programming interfaces for applications, usually referred to APIs. The main goal of this chapter is to combine the two by shaping a HTML parser and make the code portable into an RESTful API whose software structure is found at \ref{infrastructure}.
Regardless of how difficult it is the process of scraping, the circle introduced in \citet{automateddata} and here revised, is almost always the same. Most scraping activities include the following tasks:

\begin{itemize}
\tightlist
\item
  Identification of data
\item
  Algorithm search Strategy selection
\item
  Data collection
\item
  Data preprocess
\item
  Data conversion
\item
  Debugging and maintenance
\item
  Portability
\end{itemize}

Scraping essentially is a clever combination and iteration of the previous tasks which should be heavily shaped on the target website or websites.

\hypertarget{inverse-url-compostion}{%
\section{Inverse url compostion}\label{inverse-url-compostion}}

\begin{itemize}
\item
  A website contains a collection of standardized HTTP requests that return JSON or XML files. . Popular options are scraping APIs, proprietary crawling softwares, browser integration, in the end open source libraries. Challenges in scraping mainly regards \emph{security}, \emph{exception handling} and \emph{run-time}, therefore those will be the points touched during the dissertation.
\item
  RSelenium library:Docker. Open Docker Terminal and run docker pull- selenium/standalone-chrome. Replace chrome with firefox if you're a Firefox user. Then docker run -d -p 4445:4444 selenium/standalone-chrome.If above two codes are successful, run docker-machine ip and note the IP address to be used in the R code (nrowse automation) \href{https://www.pluralsight.com/guides/advanced-web-scraping-with-r}{miss lit}
\item
  To access content on the Web, we are used to typing URLs into our browser or to simply
  clicking on links to get from one place to another, to check our mails, to read news, or
  to download files. Behind this program layer that is designed for user interaction there are
  several more layers---techniques, standards, and protocols---that make the whole thing work.
  Together they are called the Internet Protocol Suite (IPS). Two of the most prominent players
  of this Protocol Suite are TCP (Transmission Control Protocol) and IP (Internet Protocol).
  They represent the Internet layer (IP) and the transportation layer (TCP). The inner workings
  of these techniques are beyond the scope of this book, but fortunately there is no need to
  manually manipulate contents of either of these protocols to conduct successful web scraping.
  What is worth mentioning, however, is that TCP and IP take care of reliable data transfer
  between computers in the network
\end{itemize}

Scraping for convenience can be mainly decomposed into 2 separable collectively exhaustive tasks: and \emph{Proper Scraping} which they need to happen in the order they are presented. As a matter of fact nearby all the scraper are also crawler since at first they neex to collect urls from sitemaps. Then they dive into the proper scraping where data is extracted beased on those urls.
The previous approach in the analysis context would be time consuming and inappropriate since data should be

As a matter of fact given the nature of scraping libraries and the programming languages used for website content creation as a first step urls need to be collected based on search parameters. Then on those urls proper scraping needs to be pplied to actually extracts data. What is commonly done among open source libraries

The forced aspects regards essentially the way websites are made and the language used for content creation and organization. HTML stands for Hyper Text Markup Language and is the standard \emph{markup} language for documents designed to be showed into a web browser. It can be supported by technologies such as Cascading Style Sheets (CSS) and other scripting languages, such as JavaScript \citep{html_2020}.
CSS is a style sheet language used for modifying the appearance of a document written in a \emph{markup} language\citep{css_2020}.
Generally speaking website try to reflect both the user expectations on the product and the creative design expression of the web developer. This is also constrained to the programming languages chosen which defines the capability to shape the website based on the specific requirements. For all the reason said, for each product to sell online, whether it is physical product or a service, there exists a multitude of website designs. For each design there exists many front-end languages which may ultimately satisfy multiple end users. A projection in a very near future may depict a scenario where websites will be displaying tailor made appearances based on personal preferences, device options etc.

As a further addition web design in scraping plays an important role since the more are implied sophisticated graphical technologies, the harder will be scraping information.

\hypertarget{graph-representation-of-html}{%
\section{Graph Representation of HTML}\label{graph-representation-of-html}}

Graph based data structures named as \textbf{Rooted Trees}. By analyzing the first dimension through the lenses of Rooted trees it is possible to compress the whole setting into tree graph jargon, as a further reference on notation and wordings can be found in \citet{Graph_Diestel}. Rooted trees must start with a root node which in this context is the domain of the web page. Each \emph{Node} is a url destination and \emph{Edges} are the connections to web pages. Jumps from one page to the others (i.e.~connections) are possible in the website by nesting urls inside webpages so that within a single webpage the user can access to a limited number of other links. Each edge is associated to a \emph{Weight} whose interpretation is the run time cost to walk from one node to its connected others (i.e.~from a url to the other). In addition the content inside each node takes the name of payload, which is ultimately the goal of the scraping processes.
The walk from node ``body'' to node ``h2'' in figure below is called path and it represented as an ordered list of nodes connected by edges. In this context each node can have both a fixed and variable outgoing sub-nodes that are called \emph{Children} . When root trees have a fixed set of children are called \emph{k-ary} rooted trees. A node is said to be \emph{Parent} to other nodes when it is connected to them by outgoing edges, in the figure below ``headre'' is the parent of nodes ``h1'' and ``p''. Nodes in the tree that shares the same parent node are said \emph{Siblings}, ``h1'' and ``p'' are siblings in figure @ref(fig:html\_tree). Moreover \emph{Subtrees} are a set of nodes and edges comprised of a parent and its descendants e.g.~node ``main'' with all of its descendants might constitute a subtree. The concept of subtree in both of the problem dimensions plays crucial role in cutting run time scraping processes as well as fake headers provision (see section \ref{spoofing}). If the website strucuture is locally reproducible and the content architecture within webpages tends to be equal, then functions for a single subtree might be extended to the rest of others siblings subtrees. Local reproducibility is a property according to which starting from a single url all the related urls can be inferred from a pattern. Equal content architecture throughout different single links means to have a standard shared-within-webpages criteria according to which each single rental advertisement has to refer (e.g.~each new advertisement replicates the structure of the existing ones). In addition two more metrics describe the tree: \emph{level} and \emph{height}. The level of a node \(\mathbf{L}\) counts the number of edges on the path from the root node to \(\mathbf{L}\) , e.g.~``head'' and ``body'', are at the same level. The height is the maximum level for any node in the tree, from now on \(\mathbf{H}\), in figure @ref(fig:html\_tree). What is worth to be anticipating is that functions are not going to be applied directly to siblings in the ``upper'' general rooted tree (i.e.~from the domain). Instead the approach follwed is segmenting the highest tree into a sequence of single children unit that shares the same level (``nav'', ``main'', ``header'', ``title'' and ``footer'') for reasons explained in section \ref{spoofing}.

Some websites' components also might be tuned by a scripting language as Javascript. JavaScript enables interactive web pages and the vast majority of websites use it for all the operations that are performed by the client in a client-server relationship \citep{Javascript_2020}.
In the context of scraping Javascript adds a further layer of difficulty. As a matter of fact Javascript components are dynamic and scraping requires specialized libraries or remote web browser automation (\citep{RSelenium} R Bindings for Selenium 2.0 Remote WebDriver) to catch the website content.

\hypertarget{crawling}{%
\section{Crawling}\label{crawling}}

\begin{itemize}
\item
  general idea, definition (with latex def. component)
\item
  urllib
\item
  clean url
\item
  representation of crawling
\item
\end{itemize}

\hypertarget{proper-scraping}{%
\section{Proper Scraping}\label{proper-scraping}}

\begin{figure}
\centering
\includegraphics{images/netstruc_vs_hierstruc.jpg}
\caption{(\#fig:html\_tree)Linearity in Website Structure vs Audience Education}
\end{figure}

A \emph{second dimension} of hierarchy is brought by content architecture by means of the language used for content creation and organization i.e.~HTML. HTML stands for Hyper Text Markup Language and is the standard \emph{markup} language for documents designed to be showed into a web browser. It can be supported by technologies such as Cascading Style Sheets (CSS) and other scripting languages, as an example JavaScript \citep{html_2020}.
HTML inner language properties brings along the hierarchy that is then inherited from the website structure. According to this point of view the hierarchical website structure is a consequence of the language chosen for building content architecture.
Since a hierarchy structure is present a direction must be chosen, this direction is from root to leaves i.e.~\emph{arborescence}.
CSS language stands for Cascading Style Sheets and is a style sheet language used for modifying the appearance of a document written in a \emph{markup} language\citep{css_2020}.
The combination of HTML and CSS offers a wide flexibility in building web sites, once again expressed by the vast amount of different websites designs on the web. Some websites' components also might be tuned by a scripting language as Javascript. JavaScript enables interactive web pages and the vast majority of websites use it for all the operations that are performed by the client in a client-server relationship \citep{Javascript_2020}.
In the context of scraping Javascript adds a further layer of difficulty. As a matter of fact Javascript components are dynamic and scraping requires specialized libraries or remote web browser automation (\citep{RSelenium} R Bindings for Selenium 2.0 Remote WebDriver) to catch the website content.
CSS instead allows the scraper to target a class of objects in the web page that shares same style (e.g.~same CSS query) so that each element that belongs to the class (i.e.~share the same style) can be gathered. This practice provides tremendous advantages since by a single CSS query a precise set of objects can be obtained within a unique function call.
First and Second dimension of the scraping problem imply hierarchy. One way to imagine hierarchy in both of the two dimensions are

\begin{figure}
\centering
\includegraphics{images/html_general_representation.jpg}
\caption{(\#fig:html\_tree) html tree structure of a general website, randomly generated online}
\end{figure}

\hypertarget{webstructure}{%
\subsection{Immobiliare.it website structure}\label{webstructure}}

The website structure of immobiliare can be assumed to be similar to the one of the largest online retailer Amazon. For that reason they both fall into the same website structure category. Sharing the same category might imply that the transition from customized website structure scraping functions (i.e.~immobiliare) do not take extraordinary sophistication to be extended to other comparable websites (i.e.~Amazon). Assuming that the scraper knows where data is stored (i.e.~payloads), the mandatory step is a way to compose and decompose url anatomy. As a matter of fact each time the scraper script visits the website it should not step back from domain root node and then down the longest path reaching the final content node. Instead it should try to shorten the path by minimizing the number of nodes encountered, conditioned to the respective nodes' weights. This is a first important conclusion since by separating the website strcuture from the content architecture scraping is massively faster and should not no more rely on the website forced root-to-node paths.
immobiliare.it is a \href{https://en.wikipedia.org/wiki/Clean_URL}{clean url} \emph{miss lit} and it can be easily parsed and queried according to some parameters (i.e.~filters) selected in their dedicated section (e.g.~city, number of rooms 5, square footage less than 60 \(m^2\), macrozone ``fiera'' and ``centro''). The url is shaped so that each further parameters and its respcetive values are appended at the end of the domain url \texttt{https://www.immobiliare.it/}. Parameters and values are appended with a proper semantic, not all the sematics are equal, that is why scraping needs sophostication when applied to other websites. One major adavatge in this context is immobilaire being a \href{https://en.wikipedia.org/wiki/Clean_URL}{clean url}, whose sematic is oriented to usability and accessibility.
Once parameters are applied to the root domain this constitutes a newer rooted tree whose url root node is the parametrized.It might have this appearance (params are city of Milan, square footage is less than 60 \(m^2\): domain + filters i.e.~\texttt{affitto-case/milano/?superficieMinima=60}. Since for the moment are generated only links related to page n°1 containing the first 25 advs links (see figure \ref{fig:websitetree}) all the remaining siblings nodes corresponding to the subsequent pages have to be initialized. In here resides the utility of Local reproducibility property introduced in the previous section. The remaining siblings, e.g.~the ones belonging to page 2 (with the attached 25 links), to page 3 etc. can be generated by adding a further parameter \texttt{\&pag=n}, where n is the page number reference (from now on referred as \emph{pagination}). Author customary choice is to stop pagination up to 300 pages since spatial data can not be to too large due to computational requirements imposed by inla methodology \ref{inla}. The code chunk below has the aim to mimic the url syntax filters building, s given a set of information it can reproduce any related sibling. detaching website structure from content architecture.

\emph{pseudo code get\_link}

\begin{Shaded}
\begin{Highlighting}[]
\ControlFlowTok{if}\NormalTok{ (}\OperatorTok{!}\KeywordTok{missing}\NormalTok{(}\StringTok{"macrozone"}\NormalTok{)) \{}
\NormalTok{    macrozone =}\StringTok{ }\KeywordTok{tolower}\NormalTok{(macrozone) }\OperatorTok\StringTok{ }\KeywordTok{iconv}\NormalTok{(}\DataTypeTok{to =} \StringTok{"ASCII//TRANSLIT"}\NormalTok{) }\OperatorTok\StringTok{ }\KeywordTok{str_trim}\NormalTok{()}
\NormalTok{    idzone =}\StringTok{ }\KeywordTok{list}\NormalTok{()}
\NormalTok{    zone =}\StringTok{ }\KeywordTok{fromJSON}\NormalTok{(here}\OperatorTok{::}\KeywordTok{here}\NormalTok{(}\StringTok{"ALLzone.json"}\NormalTok{))}
    \ControlFlowTok{for}\NormalTok{ (i }\ControlFlowTok{in} \KeywordTok{seq_along}\NormalTok{(macrozone)) \{}
\NormalTok{        zone}\OperatorTok{$}\NormalTok{name =}\StringTok{ }\NormalTok{zone}\OperatorTok{$}\NormalTok{name }\OperatorTok\StringTok{ }\KeywordTok{tolower}\NormalTok{()}
        \ControlFlowTok{if}\NormalTok{ (}\KeywordTok{grepl}\NormalTok{(macrozone[i], zone)[}\DecValTok{2}\NormalTok{]) \{}
\NormalTok{            pos =}\StringTok{ }\KeywordTok{grepl}\NormalTok{(macrozone[i], zone}\OperatorTok{$}\NormalTok{name, }\DataTypeTok{ignore.case =}\NormalTok{ T)}
\NormalTok{            idzone[i] =}\StringTok{ }\NormalTok{zone[pos, ] }\OperatorTok\StringTok{ }\KeywordTok{select}\NormalTok{(id)}
\NormalTok{        \} }\ControlFlowTok{else}\NormalTok{ \{}
            \KeywordTok{stop}\NormalTok{(}\KeywordTok{paste0}\NormalTok{(}\StringTok{"zone:"}\NormalTok{, macrozone[i], }\StringTok{" is not recognized"}\NormalTok{))}
\NormalTok{        \}}
\NormalTok{    \}}
\NormalTok{    idzone =}\StringTok{ }\NormalTok{idzone }\OperatorTok\StringTok{ }\KeywordTok{unlist}\NormalTok{() }\OperatorTok\StringTok{ }\KeywordTok{unique}\NormalTok{()}
\NormalTok{    mzones =}\StringTok{ }\NormalTok{glue}\OperatorTok{::}\KeywordTok{glue_collapse}\NormalTok{(}\DataTypeTok{x =}\NormalTok{ idzone, }\StringTok{"&idMZona[]="}\NormalTok{)}
    
\NormalTok{    dom =}\StringTok{ "https://www.immobiliare.it/"}
\NormalTok{    stringa =}\StringTok{ }\KeywordTok{paste0}\NormalTok{(dom, tipo, }\StringTok{"-case/"}\NormalTok{, citta, }\StringTok{"/?"}\NormalTok{, mzones)}
\NormalTok{    npages_vec =}\StringTok{ }\KeywordTok{str_c}\NormalTok{(stringa, }\StringTok{"&pag="}\NormalTok{, }\DecValTok{2}\OperatorTok{:}\NormalTok{npages) }\OperatorTok\StringTok{ }\KeywordTok{append}\NormalTok{(stringa, }\DataTypeTok{after =} \DecValTok{0}\NormalTok{)}
    
\NormalTok{\} }\ControlFlowTok{else}\NormalTok{ \{}
\NormalTok{    dom =}\StringTok{ "https://www.immobiliare.it/"}
\NormalTok{    stringa =}\StringTok{ }\KeywordTok{paste0}\NormalTok{(dom, tipo, }\StringTok{"-case/"}\NormalTok{, citta, }\StringTok{"/"}\NormalTok{)  }\CommentTok{# mzones}
\NormalTok{    npages_vec =}\StringTok{ }\KeywordTok{glue}\NormalTok{(}\StringTok{"\{stringa\}?pag=\{2:npages\}"}\NormalTok{) }\OperatorTok\StringTok{ }\KeywordTok{append}\NormalTok{(stringa, }\DataTypeTok{after =} \DecValTok{0}\NormalTok{)}
    
\NormalTok{\}}
\end{Highlighting}
\end{Shaded}

Up to this point pagination has generated a vector of siblings nodes whose children elements number is fixed (i.e.~25 links per page \ref{fig:websitetree} lower part). That makes those trees \emph{k-ary}, where k is 25 indicating the number of children leaves. K-ary trees are rooted trees in which each node has no more than k children, in this particular case final leaves. The well known binary rooted tree is actually a special case of k-ary when \(k = 2\). parameters reverse engineering process and 25-ary trees with equal content structure across siblings allow to design a single function to call that could be mapped for all the other siblings. In addition in order to further unroll the website a specific scraping function grabs the whole set of 25 links per page. As a result a single function call of \texttt{scrape\_href()} can grab the links corresponding to page 1. Then the function is mapped for all the generated siblings nodes (i.e.~up to 300) obtaining a collection of all links belonging to the set of pages. Ultimately the complete set of links corresponds to every single advertisement posted on immobiliare.it at a given time.

\begin{figure}
\centering
\includegraphics{images/website_tree1.jpg}
\caption{\label{fig:websitetree}immobiliare.it website structure, author's source}
\end{figure}

\hypertarget{immobiliare.it-content-architecture-with-rvest}{%
\subsection{\texorpdfstring{Immobiliare.it content architecture with \texttt{rvest}}{Immobiliare.it content architecture with rvest}}\label{immobiliare.it-content-architecture-with-rvest}}

To start a general scraping function the only requirements are a target url (i.e.~the filtered root node url) and a way to compose url (i.e.~pagination ). Then a session class object \texttt{html\_session} is opened by specifying the url and the request data that the user needs to send to the web server, see left part to dashed line in figure \ref{fig:workflow}. Information to be attached to the web server request will be further explored later, tough they are mainly three: User Agents, emails references and proxy servers. \texttt{html\_session} class objects contains a list number of useful data such as: the url, the response, cookies, session times etc. Once the connection is established (response request 200) all the following operations rely on the opened session, in other words for the time being in the session the user will be authorized with the already provided request data. The list object contains the xml/html content response of the webpage and that is where data needs to be parsed and class converted. The list can disclose as well other interesting meta information related to the session but in this context are not collected. The light blue wavy line follows the steps required to get the content parsed from the beginning to the end.

\begin{figure}
\centering
\includegraphics{images/workflow.png}
\caption{\label{fig:workflow}rvest general flow chart, author's source}
\end{figure}

To the right of dashed line in the flow chart \ref{fig:workflow} are painted a sequence of \texttt{rvest}\citep{rvest} functions that follow a general step by step text comprehension rules. \texttt{rvest} first handles parsing the html respose content of the web page within the session through \texttt{read\_html()}. Secondly, as in figure \ref{fig:ContentStructure}, it looks for a single node \texttt{html\_nodes()} through a specified CSS query. CSS is a way to route \texttt{rvest} to consider a precise node or set of nodes in the web page. For each information contained in each of the web page a different CSS query has to be called.
Thirdly it converts the content (i.e.~payload) into a human readable text with \texttt{html\_text()}. A simplified version of the important contents to be scraped in each single link is sketched in figure \ref{fig:ContentStructure}

\begin{figure}
\centering
\includegraphics{images/content_structure.jpg}
\caption{\label{fig:ContentStructure}immobiliare.it important content structure, author's source}
\end{figure}

The code chunk below exemplifies a function that can scrape the price. The function explicitly covers only the right part to the dashed line (figure \ref{fig:workflow}) of the whole scraping process. The initial part (left dashed in same figure), where session is opened and response is converted is handles inside the second code chunk where \texttt{get.data.catsing()} is.

\hypertarget{ProperScraping}{%
\section{Proper Scarping}\label{ProperScraping}}

The present algorithm imposes a nest sequential search strategy gravitating around 2 main criterias: shortest paths and insistent search. At the starting point it is initialized, providing a url, a single session object \texttt{opensess}. The object opensess constitutes a check point obj because it is reused more than once along the algorithm flow. The object contains session data as well as HTML content. Immediately after another object \texttt{price} parses the sessions and points to a CSS query through a set of HTML nodes. The CSS location \texttt{.im-mainFeatures\_\_title} addresses a precise group of data which are found right below the main title. Expectations are that monthly price amount in that location is a single character vector string, containing price along with unnecessary non-UTF characters. Then the algorithm bumps into the first \texttt{if} statement. The logical condition checks whether the object \texttt{price} first CSS search went lost. If it does not the algorithm directly jumps to the end of the algorithm and returns a preprocessed singl quantity. Indeed if it does it considers again the check up \texttt{opensess} and hits with a second css query \texttt{.im-features\_\_value\ ,\ .im-features\_\_title}, pointing to second data location. Note that the whole search is done within the same session (i.e.~reusing the same session object), so no more additional request headers \ref{spoofing} has to be sent). Since the second CSS query points to data sequentially stored into a list object, the newly initialized \texttt{price2} is a type list object containing various information. Then the algorithm flows through a second \texttt{if} statement that checks whether \texttt{"prezzo"} is matched in the list, if it does the algorithm returns the +1 position index element with respect to the ``prezzo'' position. This happens because data in the list is stored by couples sequentially (as a flattened list), e.g.~list(title, ``Appartamento Sempione'', energy class, ``G'', ``prezzo'', 1200/al mese). Then in the end a third CSS query is called and a further nested if statement checks the emptiness of the latest CSS query. \texttt{price3} points to a hidden JSON object within the HTML content. If even the last search is lost then the algorithm escapes in the else statement by setting \texttt{NA\_Character\_}, ending with any CSS query is able to find price data.
The search skeleton used for price scraping constitutes a standard reusable search method in the analysis for all the scraping functions. However for some of the information not all the CSS location points are available and the algorithm is forced to be following only certain paths, e.g.~condizionatore can not be found under main title and so on.

\begin{figure}
\centering
\includegraphics[width=0.7\textwidth,height=\textheight]{images/pseudocode_latex/pseudocode_price.jpg}
\caption{\label{fig:pseudocode1}pseudo code algorithm for price search, author's source}
\end{figure}

Once all the functions have been designed and optmized with respect to their scraping target they need to be grouped into a single function. This is done into the API endpoint which also Which also checks the validity of the url, and registers the parallel back end.

\hypertarget{best-practices}{%
\section{Scraping Best Practices and Security provisions}\label{best-practices}}

\begin{itemize}
\tightlist
\item
  http introduction presa dal libro sezione 5
\end{itemize}

Web scraping have to naturally interact multiple times with both the \emph{client} and \emph{server side} and as a result many precautions must be seriously taken into consideration. From the server side a scraper can forward as many requests as it could (in the form of sessions opened) which might cause a traffic bottleneck (DOS attack \citet{wiki:DOS}) impacting the overall server capacity. As a further side effect it can confuse the nature of traffic due to fake user agents \ref{spoofing} and proxy servers, consequently analytics reports might be driven off track.
Those are a small portion of the reasons why most of the servers have their dedicated Robots.txt files. Robots.txt \citet{meissner_2020} are a way to kindly ask webbots, spiders, crawlers to access or not access certain parts of a webpage. The de facto ``standard'' never made it beyond a \emph{informal} ``Network Working Group INTERNET DRAFT''. Nonetheless, the use of robots.txt files is widespread due to the vast number of web crawlers (e.g.~\href{https://en.wikipedia.org/robots.txt}{Wikipedia robot}, \href{https://www.google.com/robots.txt}{Google robot}). Bots from the own Google, Yahoo adhere to the rules defined in robots.txt files, although their \emph{interpretation} might differ.

Robots.txt files \citep{robotstxt} essentially are plain text and always found at the root of a website's domain. The syntax of the files follows a field-name value scheme with optional preceding user-agent. Blocks are separated by blank lines and the omission of a user-agent field (which directly corresponds to the HTTP user-agent field, cleared later in \ref{spoofing}) is seen as referring to all bots. The whole set of possible field names are pinpointed in \citet{google:robottxt}, some important are: user-agent, disallow, allow, crawl-delay, sitemap and host. A standard set of shared interpretation is:

\begin{itemize}
\tightlist
\item
  Finding no robots.txt file at the server (e.g.~HTTP status code 404) implies full permission.
\item
  Sub-domains should have their own robots.txt file, if not it is assumed full permission.
\item
  Redirects from subdomain www to the domain is considered no domain change - so whatever is found at the end of the redirect is considered to be the robots.txt file for the subdomain originally requested.
\end{itemize}

A scraping is explored in the \texttt{polite} \citet{polite} package which combines the effects of \texttt{robotstxt}, \texttt{ratelimitr} \citeyearpar{ratelimitr} to limit sequential session requests together with the \texttt{memoise} \citet{memoise} for robotstxt response caching. Even though the solution meets the requirements (from server and client side) ratelimitr is not designed to run in parallel as documented in the vignette \citet{ratelimitr}, so it is not involved in the final outcome. However the 3 simple and effective ideas wrapped up in the package describes what a ``polite'' session should look like and by doing principles are kept fixed during the scraping:

\begin{quote}
The three pillars of a polite session are seeking permission, taking slowly and never asking twice.

\hfill --- Polite
\end{quote}

The three pillars constitute the \emph{Ethical} web scraping manifesto \citep{densmore_2019} which are common shared \emph{best practices} that are aimed to self regularize scrapers. Still these have to be intended as practices and by no means as law enforcements. However many scrapers themselves, as website administrators or analyst, have fought in daily working tasks with bots and product derivatives. Intensive Crawling might fake out real client navigation log messages and digital footprint and as a consequence might induce distorted analytics.
With that said a custom function that permanently checks the validity of the session request is called once. It has to be invoked prior any scraping function execution and then immediately cached into a variable. In the result below the function applied to the domain of immobiliare.it returns a boolean approving or disallowing the permission.

\begin{verbatim}
## Memoised Function:
## [1] TRUE
\end{verbatim}

Furthermore a custom function based on robotxtst cached results initially checks if the bot can search the inputted address through \texttt{polite\_permission}. Then It observes the suggested delay date, in this particular context no delays are kindly asked. As a polite author choice delay request rate is set equal to 5 seconds. Delayed requests rate are managed through the \texttt{purrr} stack. At first a \texttt{rate} object is initialized based on polite\_permission, therefore a \texttt{rate\_sleep} delay is called within scraping as in \citet{rate_delay}.

\begin{verbatim}
## [1] "immobiliare.it"
\end{verbatim}

\begin{Shaded}
\begin{Highlighting}[]
\NormalTok{get_delay =}\StringTok{ }\ControlFlowTok{function}\NormalTok{(memoised_robot, domain) \{}
    
    \KeywordTok{message}\NormalTok{(}\KeywordTok{glue}\NormalTok{(}\StringTok{"Refreshing robots.txt data for %s... \{domain\}"}\NormalTok{))}
\NormalTok{    temp =}\StringTok{ }\NormalTok{memoised_robot}\OperatorTok{$}\NormalTok{crawl_delay}
    
    \ControlFlowTok{if}\NormalTok{ (}\KeywordTok{length}\NormalTok{(temp) }\OperatorTok{>}\StringTok{ }\DecValTok{0} \OperatorTok{&&}\StringTok{ }\OperatorTok{!}\KeywordTok{is.na}\NormalTok{(temp[}\DecValTok{1}\NormalTok{, ]}\OperatorTok{$}\NormalTok{value)) \{}
\NormalTok{        star =}\StringTok{ }\NormalTok{dplyr}\OperatorTok{::}\KeywordTok{filter}\NormalTok{(temp, useragent }\OperatorTok{==}\StringTok{ "*"}\NormalTok{)}
        \ControlFlowTok{if}\NormalTok{ (}\KeywordTok{nrow}\NormalTok{(star) }\OperatorTok{==}\StringTok{ }\DecValTok{0}\NormalTok{) }
\NormalTok{            star =}\StringTok{ }\NormalTok{temp[}\DecValTok{1}\NormalTok{, ]}
        \KeywordTok{as.numeric}\NormalTok{(star}\OperatorTok{$}\NormalTok{value[}\DecValTok{1}\NormalTok{])}
\NormalTok{    \} }\ControlFlowTok{else}\NormalTok{ \{}
\NormalTok{        5L}
\NormalTok{    \}}
    
\NormalTok{\}}
\KeywordTok{get_delay}\NormalTok{(rbtxt_memoised, }\DataTypeTok{domain =}\NormalTok{ dom)}
\end{Highlighting}
\end{Shaded}

\begin{verbatim}
## [1] 5
\end{verbatim}

\hypertarget{web-client-security-provisions-user-agents-proxies-and-fail-dealers}{%
\section{Web Client Security provisions: User Agents, Proxies and Fail Dealers}\label{web-client-security-provisions-user-agents-proxies-and-fail-dealers}}

HTTP headers are sent via HTTP protocol transactions and allow the client and the server to pass additional information with the request or the response. Some of most important request header fields are User agent, proxies, urls and e-mails addresses. From a very general point of view the process according to which HTTP protocols allow to exchange information can be easily figured out by an everyday real life world analogy. As a generic person A rings to the door's bell of person B. Then A is coming to B door with its personal information, i.e.~name, surname, where he lives etc. Since now B may either positively answer to A requests by opening the door given the set of information he has, or it may not since B is not sure of the real intentions of A. The situation can be transposed on the internet where the user browser (in the example above A) is interacting with a website server (part B) sending packets of information, figure \ref{fig:webworks}. If a server does not trust the information provided by the user, if the requests are too many, if the requests seems to be scheduled due to fixed sleeping time, a server can block requests. In certain cases it can even forbid the user to open a session to the website. Servers are built with a immune-system like software that raises barriers and block users to prevent dossing or other illegal acts.

\begin{figure}
\centering
\includegraphics{images/how_web_works.png}
\caption{\label{fig:webworks}How the web interacts between broswer and server}
\end{figure}

\hypertarget{spoofing}{%
\subsection{HTTP User Agent and Mail Spoofing}\label{spoofing}}

\begin{definition}[User Agents]
\protect\hypertarget{def:useragents}{}{\label{def:useragents} \iffalse (User Agents) \fi{} }The user agent (from now refered as UA) ``retrieves, renders and facilitates end-user interaction with Web content'' \citet{UaDef}.
\end{definition}

In HTTP, the UA string is often considered as \emph{content negotiator} \citep{wiki:UserAgent}. The requested server in the form of code embedded into the hosted website selects the most appropriate content on the basis of operating parameters for the response. Therefore according to the UA, the web server can load different CSS based on the outcome, deliver custom JavaScript, automatically send the correct translation due to UA language preferences \citep{whoishostingthis.com}.
However UA fieldcname has been recently sentenced as superseded in favor of a newer (2020) proactive content negotiator named \emph{Hints} \citet{wiki:UserAgentHints}.
UA is a dense content string that includes many user details: the user application or software, the operating system (and versions), the web client, the web client's version, as well as the web engine responsible for the content display (such as AppleWebKit).
A full components breakdown of UA example might be:

\texttt{Mozilla/5.0\ (Windows\ NT\ 6.3;\ WOW64)\ AppleWebKit/537.36\ (KHTML,\ like\ Gecko)\ Chrome/45.0.2454.85\ Safari/537.36}

\begin{itemize}
\tightlist
\item
  The user agent application is Mozilla version 5.0.
\item
  The operating system is Windows NT 6.3; WOW64, running on Windows
\item
  The client is Chrome version 45.0.2454.85.
\item
  The client is based on Safari version 537.36.
\item
  The engine responsible for displaying content on this device is AppleWebKit version 537.36 (and KHTML, an open-source layout engine, is present too).
\end{itemize}

The UA string is also one of the main responsible according to which Web crawlers and scrapers through a dedicated name field in robotstxt \ref{best-practices} may be ousted from accessing certain parts of a website. Since many requests are sent the server may encounter insistently the same UA and as consequence it may block requests associated to the same UA. In order to avoid server block this scraping technique adopts a rotation of a pool of UAs. Each time requests are sent a different set of headers are drawn from the pool and then combined. The more the pool is populated the larger are the UA combinations. The solution proposed tries in addition to resample periodically the pool as soon as the website from which Agents ID are extracted updates newer UA strings.

\begin{Shaded}
\begin{Highlighting}[]
\KeywordTok{set.seed}\NormalTok{(}\DecValTok{27}\NormalTok{)}
\NormalTok{url =}\StringTok{ "https://user-agents.net/"}
\NormalTok{agents =}\StringTok{ }\KeywordTok{read_html}\NormalTok{(url) }\OperatorTok\StringTok{ }\KeywordTok{html_nodes}\NormalTok{(}\DataTypeTok{css =} \StringTok{".agents_list li"}\NormalTok{) }\OperatorTok\StringTok{ }\KeywordTok{html_text}\NormalTok{()}

\NormalTok{agents[}\KeywordTok{sample}\NormalTok{(}\DecValTok{1}\NormalTok{)]}
\end{Highlighting}
\end{Shaded}

\begin{verbatim}
## [1] "Mozilla/5.0 (Linux; Android 10; BLA-L29; HMSCore 5.0.5.300; GMSCore 20.47.13) AppleWebKit/537.36 (KHTML, like Gecko) Chrome/83.0.4103.106 HuaweiBrowser/11.0.3.304 Mobile Safari/537.36"
\end{verbatim}

The same procedure has been applied to mails attached to the request headers. E-mails, that are randomly generated from a website, are scraped and subsequently stored into a variable. The A further way to see what it has been done for both UA and mails is considering low level API calls to dedicated servers nested into a more general higher level API.

An even more secure approach may be accomplished rotating proxy servers between the back and forth sending-receiving process. A proxy server acts as a gateway between the web user and the web server. While the user is exploiting a proxy server, internet traffic flows through the proxy server on its way to the server requested. The request then comes back through that same proxy server and then the proxy server forwards the data received from the website back to the client. The final combination would give birth to a more complex linear combination, adding a further layer of masking.
Many proxy servers are offered as paid version. In this particular case security barriers are not that high and this suggests to not apply them. As a further disclaimer many online services are providing free proxies server access, but this comes at a personal security cost due to a couple of reasons:

\begin{itemize}
\tightlist
\item
  Free plan Proxies are shared among a number of different clients, so as long as someone has used them in the past for illegal purposes the client is indirectly inheriting their legal infringements.
\item
  Very cheap proxies, for sure all of the ones free, have the activity redirected on their servers monitored, profiling in some cases a user privacy violation issue.
\end{itemize}

\hypertarget{possibly}{%
\subsection{Dealing with failure}\label{possibly}}

During scraping many difficulties coming from different sources are met. Some of them may come from the website's layout changes (\ref{ProperScraping}), some of them may regard internet connection, some other may have been caused by security breaches (section \ref{spoofing}).
One of the most inefficient event it can happen is an unexpected error thrown while sending requests that causes all the data previously acquired going lost. In this particular context is even more worrying since scraping ``main'' functions is able to call 34 different functions each of which points to a different data location. Within a single function invocation, pagination contributes to initialize 10 pages. Each single page includes 25 different single links (\ref{crawling}) leading to a number of 8500 single data pieces. Unfortunately the probability given 8500 associated to one piece being lost, unparsed is frankly high.
For all the reasons said scraping functions needs to deal with the possibility to fail. This is carried out by the implementation of \texttt{purrr} vectorization function \texttt{map} (and its derivatives) and the adverb \texttt{possibly} \citet{possibly}. \emph{Possibly} takes as argument a function (map iteration over a list) and returns a modified version. In this case, the modified function returns an empty dataframe regardless of the error thrown. The approach is strongly encouraged when functions need to be mapped over large objects and time consuming processes as outlined in \citet{Rdatascience} section 21.6. Moreover vecrotizaion is not only applied to a vector of urls, but also to a set of functions defined in the environemnt.

\begin{figure}
\centering
\includegraphics[width=0.7\textwidth,height=\textheight]{images/pseudocode_latex/pseudocode_possibly.jpg}
\caption{\label{fig:pseudocode2}pseudo code for a generic set of functions applied with possibly fail dealers , author's source}
\end{figure}

\hypertarget{parallelscraping}{%
\section{Parallel Scraping}\label{parallelscraping}}

Scraping run time is crucial when dealing with dynamic web pages. This assumption is stronger in Real Estate rental markets where time to market is a massive competitive advantage.
From a run time perspective the dimension of the problem requires as many html session opened as single links crawled (refer to previous section \ref{possibly}). As a result computation needs to be \emph{parallelized} in order to be feasible.
The extraordinary amount of time taken in a non-parallel environment is caused by R executing scraping on a single processor \emph{sequentially} url-by-url in a queue, left part of figure \ref{fig:singlethreaded} (i.e.~single threaded computing).

\begin{definition}[parallel]
\protect\hypertarget{def:parallel}{}{\label{def:parallel} \iffalse (parallel) \fi{} }\emph{Parallel execution} is characterized as multiple operations taking place over overlapping time periods. \citep{eddelbuettel2020parallel}
\end{definition}

This requires multiple execution units and modern processors architecture provide multiple cores on a single processor and a way to redistribute computation (i.e.~multi threaded computing). As a result tasks can be split into smaller chunks over processors and then multiple cores for each processor, right part of figure \ref{fig:singlethreaded}.
Therefore Parallel scraping (sometimes improperly called \href{https://medium.com/@cummingsi1993/the-difference-between-asynchronous-and-parallel-6400729fa897}{asynchronous}) functions are proposed, so that computation do not employ vast cpu time (i.e.~cpu-bound) and space.

\begin{figure}
\centering
\includegraphics{images/parallel_problem.jpg}
\caption{\label{fig:singlethreaded}single threaded computing vs parallel computing, \citet{barney} source}
\end{figure}

Parallel execution heavily depends on hardware and software choice. Linux environments offers multi-core computation through \emph{forking} \citep{wiki:forking} (only on Linux) so that global variables are directly inherited by child processes. As a matter of fact when computation are split over cores they need to import whatever it takes to be carried out, such as libraries, variables, functions. From a certain angle they need to be treated as a containerized stand-alone environments. This can not happen in Windows (local machine) since it does not support multicore.

\begin{Shaded}
\begin{Highlighting}[]
\NormalTok{future}\OperatorTok{::}\KeywordTok{supportsMulticore}\NormalTok{()}
\end{Highlighting}
\end{Shaded}

\begin{verbatim}
## [1] FALSE
\end{verbatim}

\emph{Cluster processing} is an alternative to multi-core processing, where parallelization takes place through a collection of separate processes running in the background. The parent R session instructs the dependencies that needs to be sent to the children sessions.
This is done by registering the parallel back end. Arguments to be supplied mainly regards the strategy (i.e.~multi-core cluster, also said multisession) and the \emph{working group}. The working group is a software concept \citep{parallelr}, that points out the number of processes and their relative computing power/memory allocation according to which the task is going to be split. Moreover from a strictly theoretic perspective the \emph{workers} (i.e.~working group single units) can be greater than the number of physical cores detected. Although parallel libraries as a default choice (and choice for this analysis) initializes \emph{as many workers as} physical HT (i.e.~Hyper Threaded) \emph{cores}.
Parallel looping constructor libraries generally pops up as a direct cause of new parallel packages. The latest research activity by Bengtsson \citet{doFuture} indeed tries to unify all the previous back ends under the same umbrella of \texttt{doFuture}. The latter library allows to register many back ends for the most popular parallel looping options solving both the dependency inheritance problem and the OS agnostic challenge.
The two alternatives proposed for going parallel are \texttt{Future} \citet{future} with \texttt{furrr} \citet{furrr} and \texttt{doFuture} \citeyearpar{doFuture} along with the \texttt{foreach} \citet{foreach} loop constructor. The former is a generic, low-level API for parallel processing as in \citet{bengtsson_2017}. The latter takes inspiration by the previous work and it provides a back-end agnostic version of \texttt{doParallel} \citet{doParallel}.
Further concepts on parallel computing are beyond the scope of the analysis. However they can be explored in \citet{barney}, which may offers a comprehensive perspective on Parallel theory both on hardware and software. Indeed for a full reference on the R parallel ecosystem, run time simulations and advanced algorithm back end design strategies, the authorities are \citet{parallelr}. If the interest is to cut short theory and directly put existing R code into parallel, a valuable resource is covered in \href{https://nceas.github.io/oss-lessons/parallel-computing-in-r/parallel-computing-in-r.html}{blog}, which also investigate the main debugging aspects.

\hypertarget{parallel-furrrfuture}{%
\subsection{Parallel furrr+future}\label{parallel-furrrfuture}}

\textbf{cerca di centrare di più su scraping}

Simulations are conducted on a not-rate-delayed (section \ref{best-practices}) and restricted set of functions which may be considered as a ``lightweight'' version of the final API scraping endpoint.
As a disclaimer run time simulations may not be really representative to the problem since they are performed on a windows 10, Intel(R) Core(TM) i7-8750H 12 cores RAM 16.0 GB local machine. Indeed the API is served on a Linux Ubuntu distro t3.micro 2 cores RAM 1.0 GB server which may adopt forking. Simulations for the reasons said can only offer a run time performance approximation for both of the parallel + looping constructor combinations.

The first simulation considers \texttt{furrr} which enables mapping (i.e.~vectorization with \texttt{map}) through a list of urls with \texttt{purrr} and parallelization with \texttt{Future}. Future gravitates around a programming concept called ``future'', initially introduced in late 70's by Baker \citep{BakerFuture}. Futures are abstractions for values that may be available at a certain time point later \citeyearpar{future}.
These values are result of an evaluated expression, this allows to actually divide the assignment operation from the proper result computation. Futures have two stages \emph{unresolved} or \emph{resolved}. If the value is queried while the future is still unresolved, the current process is blocked until the stage is resolved. The time and the way futures are resolved massively relies on which strategy is used to evaluate them. For instance, a future can be resolved using a \emph{sequential} strategy, which means it is resolved in the current R session. Other strategies registered with \texttt{plan()}, such as \emph{multi-core} (on Linux) and \emph{multisession}, may resolve futures in parallel, as already pointed out, by evaluating expressions on the current machine in forked processes or concurrently on a cluster of R background sessions.
With parallel futures the current/main R process does not get ``bottlenecked'', which means it is available for further processing while the futures are being resolved in separate processes running in the background. Therefore with a ``multisession'' plan are opened as many R background sessions as workers/cores on which chunks of futures (urls) are split and resolved in parallel. From an algorithmic point of view It can be compared to \emph{a divide and conquer} strategy where the target urls are at first redistributed among workers/cores (unresolved) through background sessions and then are scraped in equally distributed chunks (resolved).
Furthermore furrr has also a convenient tuning option which can interfere with the redistribution scheduling of urls' chunks over workers. The argument scheduling can adjust the average number of chunks per worker. Setting it equal to 2 brings \emph{dinamicity} \citeyearpar{furrr} to the scheduling so that if at some point a worker is busier then chunks are sent to the more free ones.

(migliore rappresentazione)

\begin{figure}
\centering
\includegraphics{images/divideconquer.png}
\caption{\label{fig:divideconquer}futures reimagined as divide and conquer, author's source}
\end{figure}

The upper plot in figure \ref{fig:furrrfuture} are 20 simulations of 100 url (2500 data points) performed by the lightweight scraping. On the x-axis are plotted the 20 simulations and on the y-axis are represented the respective elapsed times. One major point to breakdown is the first simulation run time measurement which is considerably higher with respect to the others i.e.~15 sec vs mean 7.72 sec.~Empirical demonstrations traces this behavior back to the opening time for all the background sessions. As a result the more are the back ground sessions/workers, the more it would be the time occupied to pop up all the sessions. As opposite whence many sessions are opened the mean execution time for each simulation is slightly less.
The lower plot in in figure \ref{fig:furrrfuture} tries to capture the run time slope behavior of the scraping function when urls (1 to 80) are cumulated one by one. The first iteration scrapes 1 single url, the second iteration 2, the third 3 etc. Three replications of the experiment has been made, evidenced by three colours. The former urls are more time consuming confirming the hypothesis casted before. Variability within the first 40 urls for the three repetitions does not show diversion. However It slightly increases when the 40 threshold is trespassed. Two outliers in the yellow line are visible in the nearby of 50 and 60. It can be assumed that workers in that urls neighbor may be overloaded but no evidences are witnessed on cores activity as in plot \ref{fig:cpumonitor}. The measured computational complexity of scraping when \(n\) is number of urls seems to be much more less than linear \(\mathcal{O}(0.06n)\).

\begin{figure}
\centering
\includegraphics{images/simulations/final_furrr_future.png}
\caption{\label{fig:furrrfuture}computational complexity analysis with Furrr}
\end{figure}

\begin{figure}
\centering
\includegraphics{images/parallel_computing.jpg}
\caption{\label{fig:cpumonitor}local machine monitoring of cores during parallel scraping}
\end{figure}

\hypertarget{parallel-foreachdofuture}{%
\subsection{Parallel foreach+doFuture}\label{parallel-foreachdofuture}}

A second attempt tries to encapsulate \texttt{foreach} \citep{foreach} originally developed by Microsoft R, being a very fast loop alternative, parallelized with \texttt{doFuture}. The package registered with older back ends required rigorous effort to specify exact dependencies for child process inside foreach arguments \texttt{.export}. From a certain angle the approach could led to an indirect benefit from memory optimization. If global variables needs to be stated than the developer might be forced to focus on limiting packages exporting. Indeed since doFuture implements optimized auto dependency search this problem may be considered solved as in \citet{doFuture}. Two major looping related speed improvements may come from \texttt{.inorder} and \texttt{.multicombine} arguments which both take advantage of parallel split disorder a subsequent recombination of results. In the context where data collection order matters this is extremely wrong, but since in this case order is defined through url composition based on criteria expressed inside nodes contents this can be totally applied. A drawback of enabling .multicombine is a worst debugging experience since errors are thrown at the end when results are reunited and no traceback of the error is given.

The upper part in \ref{fig:foreachdofuture} displays lower initialization lag from R sessions opening and parallel execution that also lead to a lower mean execution time of 6.42 seconds. No other interesting behavior are detected.
THe lower plot displays high similarities with the curves in \ref{fig:furrrfuture} highlighting an outlier in the same proximities of 45/50 urls. The blue simulation repetition shows an uncommon pattern that is not seen in the other plot. Segmented variability from 40 to 80 suggests a higher value which may be addressed do instability. As a result the approach is discarded in favor of furrr + future which also offers both a comfortable \{Tidyverse\} oriented framework and offers and easy debugging experience.

\begin{figure}
\centering
\includegraphics{images/simulations/final_foreach_dofuture_1\&2.png}
\caption{\label{fig:foreachdofuture}computational complexity analysis with Furrr}
\end{figure}

\hypertarget{challenges}{%
\section{Open Challenges and Further Improvemements}\label{challenges}}

Although results are quite encouraging still one of the main challenges remains unsolved. In fact optimization has involved each scraping layer but scraping function must be continuously kept up with the immobiliare.it changes, particularly the crawling part. Indeed the proper scraping part, with some further adjustments can take care of auto-search for exact information within the web page even if the design changes. This idea is massively applied in the package Rcrawler \citet{Rcrawler}, which doesn't apply segmentation in crawling, instead it downloads the entire website and then inspects targeted keywords. The major benefit relies in crawling HTML/XML that are agreed to be generally lightweight in this way the process does not weigh down the run time.
Whence all files are saved locally the algorithm applies search methodologies within the HTML files. Run time performance with algorithm of this kind with respect to the amount of data gathered are very efficient, nevertheless results are not always effective due to keywords disambiguation.
Afterall a way safer and time saving approach to general scraping may be including complex theme specific regular expressions on the HTML text which univocally identify CSS data location. With that said the idea would be an unsupervised algorithm that combines the traditional browser search + a selector gadget for CSS conversion.
As a disclaimer Rcrawler is designed to be flexible to scrape a vast number of websites. As opposite the scraping functions here presented are exclusively built on top of immobiliare.it, even though they can be extended with a small effort to other category related website \ref{webstructure}.
On the parallel computing side a further boosts might be added with parallel distributed processing through HPC (high-performance computing) clusters by \texttt{future.batchtools} \citet{futurebatchtools}. The package implements a generic future wrapper around batchtools with job scheduler like Torque, Slurm, Sge and many more. A higher level API built on top of the Future framework that exploits \href{https://cloudyr.github.io/googleComputeEngineR/articles/massive-parallel.html}{Google Cloud Engine Clusters} i.e.\texttt{cloudyR} allows to distribute computation on Google machines.

\hypertarget{legal-profiles}{%
\section{Legal Profiles}\label{legal-profiles}}

\textbf{rivedere meglio}

``Data that is online and public is always available for all'' is never a good answer to the question ``Can I use those web data to my scope?''. Immobiliare.it does not provide any open source data neither it disposes any paid API.
A careful reading of immobiliare terms, reviewed with a intellectual property expert has been done to get the service running without any legal consequence, as a reference the full policy can be seen in their \href{https://www.immobiliare.it/terms/}{specialized section}. Nevertheless the golden standard for scraping was respected since the robotstxt is neat allowing any actions as demonstrated above in section \ref{best-practices}. So if it may be the case of misinterpretation of their policy, it will be also the case of lack of communication between servers response and immobiliare.it intent to preserve their own intellectual property.

\begin{itemize}
\tightlist
\item
  To the headers are also attached a direct user back tracking and a url pointing to a dedicated address
\end{itemize}

What it was shockingly surprising were the low entry barriers to scrape information with respect to other counterpart online players. Best practices are in any case applied and kind requests (even though politeness was not asked) have been sent to normalize traffic congestion. But scraping criteria followed are once again fully based on common shared best practices (see section \ref{best-practices}), and \emph{not} any sort of general agreements between parties. As a result a plausible approach could be applying scraping procedures without any prevention. It would not surely cause any sort of disservice for the website since budget constraints are set low, but in the long run it will cause lagging as soon as requests rate would increase. Totally different was the approach proposed bya coiounterpart market Idealista.com. Idealista does block requests if they are not in compliance with their servers inner rules. User agents in this case must be rotated quite frequently and proxies are necessary. Delay is kindly asked and it must be specified, consequnetly this slows down scraping function per se.

\begin{itemize}
\tightlist
\item
  Idealista content is composed by Javascript so and html parser can no get that.
\item
  Idealista blocks also certain web browser that have a demonstrated ``career'' in scraping procedures.
\end{itemize}

All of this leads to accept that entry barriers to scrape are for sure higher than the one faced for Immobiliare. The reticence to share data could be a reflex on how big idealista is; as a matter of fact it has a heavy market presence in some of the Europe real estate country as Spain and France. So the hidden intention was to raise awareness on scraping procedure that in a certain remote way can hurt their business. This has been validated by the fact that prior filtering houses on their website a checkbox has to be signed. The checkbox make the user sign an agreement on their platform according to which data can not be misused and it belongs their intellectual property.
\href{https://blawgsearch.justia.com/}{prova qui a vedere}

\hypertarget{Infrastructure}{%
\chapter{API Technology Stack}\label{Infrastructure}}

In order to provide a fast and secure API service to the end user many technologies needs to be considered. Challenges in scraping as pointed out in section \ref{challenges} are many and still some unfortunately remains unsolved. Challenges not only regard scraping per se, but also the way and how many times the service has to interact with users. Some of the main obstacle to tackle in dealing with users are:

\begin{itemize}
\tightlist
\item
  The API has to be executed \(n\) given times at fixed date-time daily and it has to store resulting data on a cloud database. This is done with the explicit goal of tracking the evolution of the phenomenon in time.
\item
  The API has to be very fast otherwise data can not be consumed.
\item
  The API has to be deployed so that it can be shared over different stakeholders, without having them to know what it takes.
\item
  The API maintainer needs to take action be to due to sudden unpredictable immobiliare.it changes, thus it needs to be continuously maintained and updated.
\item
  On the other code behind the service has to be version controlled and freezed, so that the service can guarantee continuity and prevent failures.
\item
  service has to be scalable at need since, due to deployment, when the number of users increases the run time performances should not decrease.
\item
  In addition API inbound traffic has to be managed both in terms traffic and security by securing access only to the ones authorized.
\end{itemize}

The long list of requirements is met by many DevOps technologies, some of them offer a direct R embedding. As a result the technologies stack is the intersection between a comprehensive documentation available and the requirements listed.
Fo these reasons the recipe is to provide a REST Plumber API with 4 endpoints each of which calls scraping functions in Parallel built in section \ref{scraping}. On top of that a daily Cron Job scheduler exposes one precise API endpoint, which produces and later stores a .csv file in a NOSQL mongoDB Atlas could database. Containerization happens through a Linux OS (Ubuntu distr) Docker container hosted by a free tier AWS EC2 server machine equipped with 30GB max capacity. API endpoints are secured with https protocols, load balanced and protected with authentication by NGINX reverse proxy server. On a second server a Shiny App calls one endpoint gievn specified parameters that returns data from the former infrastructure. A sketch of the infrastructure is represented in figure \ref{fig:CompleteStructure}.

Technology stack:

\begin{itemize}
\tightlist
\item
  GitHub version control
\item
  Scheduler cron job, section \ref{scheduler}
\item
  Plumber REST API, section \ref{plumberapi}
\item
  Docker containers, section \ref{docker}
\item
  AWS (Amazon Web Services) EC2 \ref{aws}
\item
  NGINX reverse proxy, section \ref{nginx}
\item
  MongoDB Atlas
\item
  Shiny
\end{itemize}

\begin{figure}
\centering
\includegraphics{images/tot_infra.jpg}
\caption{\label{fig:CompleteStructure}complete infrastructure, author's source}
\end{figure}

As a side note each single part of this thesis has been made according to some of the API inspiring criteria of sharing and self containerization. RMarkdown \citep{rmarkdown1} documents (book's chapters) are compiled and then converted into .html files. Through Bookdown \citep{bookdown2} the resulting documents are put together according to general .yml instruction file and are readble as gitbook.
Files are then pushed to a \href{https://github.com/NiccoloSalvini/thesis}{Github repository}. By a simple trick with GH pages, .html files are dispalyed into a Github subdomain hosted at \href{https://niccolosalvini.github.io/thesis/}{link}. The resulting deployed gitbook can also produce a .pdf version output through a Xelatex engine. Xelatex compiles .Rmd documents according to a .tex template which formatting rules are contained in a further .yml file. The pdf version of the thesis can be obtained by clicking the download button, then choosing pdf output version in the upper banner. For further references on the topic \citet{bookdown2}

Some of the main technologies implied will be viewed singularly, nonetheless for brevity reasons the rest needs to be skipped.

\hypertarget{scheduler}{%
\section{Scheduler}\label{scheduler}}

\begin{definition}[Scheduler]
\protect\hypertarget{def:scheduler}{}{\label{def:scheduler} \iffalse (Scheduler) \fi{} }A Scheduler in a process is a component on a OS that allows the computer to decide which activity is going to be executed. In the context of multi-programming it is thought as a tool to keep CPU occupied as much as possible.
\end{definition}

As an example it can trigger a process while some other is still waiting to finish. There are many type of scheduler and they are based on the frequency of times they are executed considering a certain closed time neighbor.

\begin{itemize}
\tightlist
\item
  Short term scheduler: it can trigger and queue the ``ready to go'' tasks

  \begin{itemize}
  \tightlist
  \item
    with pre-emption
  \item
    without pre-emption
  \end{itemize}
\end{itemize}

The ST scheduler selects the process and It gains control of the CPU by the dispatcher. In this context we can define latency as the time needed to stop a process and to start a new one.

\begin{itemize}
\tightlist
\item
  Medium term scheduler
\item
  Long term scheduler
\end{itemize}

for some other useful but beyond the scope references, such as the scheduling algorithm the reader can refer to \citep{wiki:scheduler}.

\hypertarget{cron-jobs}{%
\subsection{Cron Jobs}\label{cron-jobs}}

\begin{definition}[Cronjob]
\protect\hypertarget{def:cronjob}{}{\label{def:cronjob} \iffalse (Cronjob) \fi{} }Cron job is a software utility which acts as a time-based job scheduler in Unix-like OS. Linux users that set up and maintain software environments exploit cron to schedule their day-to-day routines to run periodically at fixed times, dates, or intervals. It typically automates system maintenance but its usage is very flexible to whichever needed. It is lightweight and it is widely used since it is a common option for Linux users.
\end{definition}
The tasks by cron are driven by a crontab file, which is a configuration file that specifies a set of commands to run periodically on a given schedule. The crontab files are stored where the lists of jobs and other instructions to the cron daemon are kept.
Each line of a crontab file represents a job, and the composition follows the syntax in figure \ref{fig:crontab}

\begin{figure}
\centering
\includegraphics{images/crontab.PNG}
\caption{\label{fig:crontab}Crontab Scheduling Syntax}
\end{figure}

Each line of a crontab file represents a job. This example runs a shell named scheduler.sh at 23:45 (11:45 PM) every Saturday. .sh commands can update mails and other minor routines.

45 23 * * 6 /home/oracle/scripts/scheduler.sh

Some rather unusual scheduling definitions for crontabs can be found in this reference \citep{wiki:cronjob}. Crontab's syntax completion can be made easier through \href{https://crontab.guru/}{this} GUI.

The cron job needs to be ran on scraping fucntions at 11:30 PM every single day. The get\_data.R script first sources an endpoint function, then it applies the function with fixed parameters. Parameters describe the url specification, so that each time the scheduler runs the get\_data.R collects data from the same source. Day after day .json files are generated and then stored into a NOSQL \emph{mongoDB} database whose credentials are public. Data are collected on a daily basis with the explicit aim to track day-by-day changes both in the new entries an goners in rental market, and to investigate the evolution of price differentials over time. Spatio-Temporal modeling is still quite unexplored, data is saved for future used. Crontab configuration for daily 11:30 PM schedules has this appearance:

30 11 * * * /home/oracle/scripts/get\_data.R

To a certain extent what it has been already presented since now might fit for personal use. A scheduler can daily execute the scraping script and can generate a .csv file. Later the same .csv file can be parsed into an application and analysis can be locally reported. The solution proposed is totally \emph{not feasible} in a production environment, since in order to be executed a vast number files has to be sourced and a number of functions should be routinely called. For these reasons the present architecture can not be shared. The solution adopted tries to minimize the analyst/scientist involvement into scraping procedures by offering a compact and fast (due to Parallel execution) service that manages all the processes without having to know how scraping under the hood is working.

\hypertarget{rest-api}{%
\section{REST API}\label{rest-api}}

\begin{definition}[API]
\protect\hypertarget{def:api}{}{\label{def:api} \iffalse (API) \fi{} }API stands for application programming interface and it is a set of definitions and protocols for building and integrating application software. Most importantly APIs let a product or a service communicate with other products and services without having to know how they're implemented.
\end{definition}
APIs are thought of as contracts, with documentation that represents an general agreement between parties.
There are many types of APIs that exploit different media and architectures to communicate with apps or services.
\begin{definition}[REST]
\protect\hypertarget{def:rest}{}{\label{def:rest} \iffalse (REST) \fi{} }The specification REST stands for \textbf{RE}presentational \textbf{S}tate \textbf{T}ransfer and is a set of \emph{architectural principles}.
\end{definition}
When a request is made through a REST API it transfers a representation of the state to the requester. This representation, is submitted in one out of the many available formats via HTTP: JSON (Javascript Object Notation), HTML, XLT, TXT. JSON is the most popular because it is language agnostic \citep{what_is_a_rest_api}, as well as it is more comfortable to be read and parsed.
In order for an API to be considered REST, it has to conform to these criteria:

\begin{itemize}
\tightlist
\item
  A client-server architecture made up of clients, servers, and resources, with requests managed through HTTP.
\item
  Stateless client-server communication, meaning no client information is stored between requests and each request is separate and unconnected.
\item
  Cacheable data that streamlines client-server interactions.
\item
  A uniform interface between components so that information is transferred in a standard form. This requires that:

  \begin{itemize}
  \tightlist
  \item
    resources requested are identifiable and separate from the representations sent to the client.
  \item
    resources can be manipulated by the client via the representation they receive because the representation contains enough information to do so.
  \item
    self-descriptive messages returned to the client have enough information to describe how the client should process it.
  \item
    hypermedia, meaning that after accessing a resource the client should be able to use hyperlinks to find all other currently available actions they can take.
  \end{itemize}
\item
  A layered system that organizes each type of server (those responsible for security, load-balancing, etc.) involved the retrieval of requested information into hierarchies, invisible to the client.
\end{itemize}

REST API accepts http requests as input and elaborates them through end points. An end point identifies the operation through traditional http methods (e.g.~/GET /POST) that the API caller wants to perform. Further documentation and differences between HTTP and REST API can be found to this \href{https://docs.aws.amazon.com/it_it/apigateway/latest/developerguide/http-api-vs-rest.html}{reference}.

open REST API examples:
- BigQuery API: A data platform for customers to create, manage, share and query data.
- YouTube Data API v3: The YouTube Data API v3 is an API that provides access to YouTube data, such as videos, playlists, and channels.
- Cloud Natural Language API: Provides natural language understanding technologies, such as sentiment analysis, entity recognition, entity sentiment analysis, and other text annotations, to developers.
- Skyscanner Flight Search API: The Skyscanner API lets you search for flights \& get flight prices from Skyscanner's database of prices, as well as get live quotes directly from ticketing agencies.
- Openweathermap API: current weather data for any location on Earth including over 200,000 cities.

\begin{figure}
\centering
\includegraphics{images/Rest-API.png}
\caption{API general functioning functioning}
\end{figure}

\hypertarget{plumberapi}{%
\subsection{Plumber REST API}\label{plumberapi}}

Plumber allows the user to create a REST API by adding decoration comments to the existing R code, in this case to scraping code. Decorations are a special type of comments that suggests to Plumber where and when the API specifications parts are. Below a simple example extracted by the documentation:

\begin{Shaded}
\begin{Highlighting}[]
\CommentTok{# plumber.R}

\CommentTok{#* Echo back the input}
\CommentTok{#* @param msg The message to echo}
\CommentTok{#* @get /echo}
\ControlFlowTok{function}\NormalTok{(}\DataTypeTok{msg=}\StringTok{""}\NormalTok{) \{}
  \KeywordTok{list}\NormalTok{(}\DataTypeTok{msg =} \KeywordTok{paste0}\NormalTok{(}\StringTok{"The message is: '"}\NormalTok{, msg, }\StringTok{"'"}\NormalTok{))}
\NormalTok{\}}

\CommentTok{#* Plot a histogram}
\CommentTok{#* @serializer png}
\CommentTok{#* @get /plot}
\ControlFlowTok{function}\NormalTok{() \{}
\NormalTok{  rand =}\StringTok{ }\KeywordTok{rnorm}\NormalTok{(}\DecValTok{100}\NormalTok{)}
  \KeywordTok{hist}\NormalTok{(rand)}
\NormalTok{\}}

\CommentTok{#* Return the sum of two numbers}
\CommentTok{#* @param a The first number to add}
\CommentTok{#* @param b The second number to add}
\CommentTok{#* @post /sum}
\ControlFlowTok{function}\NormalTok{(a, b) \{}
  \KeywordTok{as.numeric}\NormalTok{(a) }\OperatorTok{+}\StringTok{ }\KeywordTok{as.numeric}\NormalTok{(b)}
\NormalTok{\}}
\end{Highlighting}
\end{Shaded}

three endpoints associated to 2 /GET and 1 /POST requests are made available. Functions are made clear without names so that whenever the endpoint is called functions are directly executed.
Decorations are marked as this \texttt{\#*} and they are followed by specific keywords denoted with \texttt{@}.
- the \texttt{@params} keyword refers to parameter that specifies the corpus of the HTTP request, i.e.~the inputs with respect to the expected output. If default parameters are inputted then the API response is the elaboration of the functions with default parameters. As opposite endpoint function elaborates the provided parameters and returns a response.
- \texttt{\#*\ @serializer} specifies the extension of the output file when needed.
- \texttt{\#*\ @get} specifies the method of HTTP request sent.
- \texttt{/echo} is the end point name.
- \texttt{@filter} decorations activates a filter layer which are used to track logs and to parse request before passing the argbody to the end points.

Many more options are available to customize plumber API but are beyond the scope, a valuable resource for further insights can be found in the dedicated package website \citep{an_api_generator_for_r}

\hypertarget{immobiliare.it-parallel-rest-api}{%
\subsection{\texorpdfstring{Immobiliare.it \emph{Parallel} REST API}{Immobiliare.it Parallel REST API}}\label{immobiliare.it-parallel-rest-api}}

(Sanitization, antidossing, logs tracking)

The API service is composed by 4 endpoints \emph{/scrape} , \emph{/links}, \emph{/complete} and \emph{/get\_data}:

\begin{itemize}
\item
  */scrape performs a fast Parallel scraping of the website that leverages a rooted tree shortest path to get to data (250 X 5 predictors in \(\approx 10.91^{''}\)). This comes at the cost of the number of available covariates to scrape which are: title, price, number of rooms, sqmeter, primarykey. By default the end point scrape data from Milan real estate rents. It is a superficial and does not contain geospatial, however it might fit for some basic regression settings. The macrozone parameter allows to specify the NIL (Nucleo Identità Locale), targeting very detailed zones in some of the cities for which is available (Roma, Firenze, Milano, Torino).
\item
  */links: extracts the list of each single advertisement link belonging to each of the npages parameter specified, recall section \ref{webstructure}. It displays sufficient performances in terms of run time. It is strictly needed to apply the following endpoint. .thesis options secures a pre combined url with the data wanted for thesis analysis. The option takes care to decompose the website structure of the url supplied with the aim to apply scraping function in the /complete endpoint.
\item
  */complete: both the function all.links and complete are sourced. The former with the aim to grab each single links and store it into an object. The latter to actually iterate Parallel scraping on each of the extracted link.
\item
  */get\_data: it triggers the data extraction by sourcing the /complete endpoint and then storing .json file into the NOSQL mongoDB ATLAS
\end{itemize}

\begin{figure}
\centering
\includegraphics{images/swagger.PNG}
\caption{Swagger UI screenshot, author's source}
\end{figure}

\hypertarget{APIdocs}{%
\subsection{REST API documentation}\label{APIdocs}}

\begin{itemize}
\tightlist
\item
  Get FAST data, it covers 5 covariates:
\end{itemize}

\begin{verbatim}
      GET */scrape
      @param city [chr string] the city you are interested in (e.g. "roma", "milano", "firenze"--> lowercase, without accent)
      @param npages [positive integer] number of pages to scrape, default = 10, min  = 2, max = 300
      @param type [chr string] "affitto" = rents, "vendita"  = sell 
      @param macrozone [chr string] avail: Roma, Firenze, Milano, Torino; e.g. "fiera", "centro", "bellariva", "parioli" 
      content-type: application/json 
\end{verbatim}

\begin{itemize}
\tightlist
\item
  Get all the links
\end{itemize}

\begin{verbatim}
      GET */link
      @param city [chr string] the city you are interested to extract data (lowercase without accent)
      @param npages [positive integer] number of pages to scrape default = 10, min  = 2, max = 300
      @param type [chr string] "affitto" = rents, "vendita"  = sell 
      @param .thesis [logical] data used for master thesis
      content-type: application/json 
\end{verbatim}

\begin{itemize}
\tightlist
\item
  Get the complete set of covariates (52) from each single links, takes a while
\end{itemize}

\begin{verbatim}
      GET */complete
      @param city [chr string] the city you are interested to extract data (lowercase without accent)
      @param npages [positive integer] number of pages to scrape default = 10, min  = 2, max = 300
      @param type [chr string] "affitto" = rents, "vendita"  = sell 
      @param .thesis [logical] data used for master thesis
      content-type: application/json
\end{verbatim}

Up to this point the API can smoothly run in local and potentially can be deployed to share results without requesting scraping process knowledge. However the API software for now is not portable and is very heavy. In addition it can also run into failures for many reasons, one among the others is package versione incompatibility due to updates. In the end it also fully relies on the laptop computational power that can be heavily stressed when a number of API calls are executed, especially for single threaded programming languages as R.
The approach followed proposes a dedicated lightweight software environment that minimizes dependencies both improving performances and enabling the \emph{cloud computing} coverage. A fast growing techonlogy is what fits the need.

\hypertarget{docker}{%
\section{Docker}\label{docker}}

\begin{definition}[Docker]
\protect\hypertarget{def:docker}{}{\label{def:docker} \iffalse (Docker) \fi{} }\emph{Docker} \citep{docker} is a software tool to create and deploy applications using containers.
\emph{Docker containers} are a standard unit of software (i.e.~software boxes) where everything needed for applications, such as libraries or dependencies can be run reliably and quickly. Containers are also portable, in the sense that they can be taken from one computing environment to the following without further adaptations.
\end{definition}

Containers can be thought as an abstraction that groups code and dependencies together. One major advantage of containers is that multiple containers can run on the same machine with the same OS with their specific dependencies. Each container can run its own isolated process in the user space, so that each task/application is complementary to the other. The fact that containers are treated singularly enables a collaborative framework that it also simplifies bugs isolation.

When images are built \emph{Docker container} are created and can be open sourced through Docker Hub.
\emph{Docker Hub} is a web service provided by Docker for searching and sharing container images with other teams or developers in the community. Docker Hub can connect with GitHub behind authorization entailing an image version control tool. Once the connection is established changes that are pushed with git to the GitHub repository are passed to Docker Hub. The push command automatically triggers the image building. Then docker image can be tagged (salvini/api-immobiliare:latest)so that on one hand it is recognizable and on the other can be reused in the future. Once the building stage is completed the DH repository can be pulled and then run locally on machine or cloud, see section \ref{aws}.
Docker building and testing images can be very time consuming. R packages can take a long time to install because code has to be compiled, especially if using R on a Linux server or in a Docker container.
Rstudio \href{https://packagemanager.rstudio.com/client/\#/}{package manager} includes beta support for pre-compiled R packages that can be installed faster. This dramatically reduces packages time installation \citep{nolis_2020}.
In addition to that an open source project \href{https://www.rocker-project.org/images/}{rocker} has narrowed the path for developers by building custom R docker images for a wide range of usages. What can be read from their own website about the project is: ``The rocker project provides a collection of containers suited for different needs. find a base image to extend or images with popular software and optimized libraries pre-installed. Get the latest version or a reproducible fixed environment''.

\hypertarget{why-docker}{%
\subsection{Why Docker}\label{why-docker}}

\href{https://it.indeed.com/}{Indeed}, an employment-related search engine, released an article on 2019 displaying changing trends from 2015 to 2019 in Technology Job market, a summary of those changes is in figure \ref{fig:indeedstats}. Many changes are relevant in key technologies. Two among the others technologies (i.e.~docker and Azure, arrow pointed) have experienced a huge growth and both refer to the certain demand input: \emph{containers} and \emph{cloud computing}.
The landscape of Data Science is changing from reporting to application building:
In 2015 - Businesses reports drive better decisions.
In 2020 - Businesses need apps to empower better decision making at all levels.

\begin{figure}
\centering
\includegraphics{images/Inkedindeed_jobs_LI.jpg}
\caption{\label{fig:indeedstats}Indeed top skills for 2019 in percent changes, \citet{top_tech_skills} source}
\end{figure}

For all the things said what docker is bringing to business are \citep{red_hat_customer_portal}:

\begin{itemize}
\tightlist
\item
  \emph{Speed application deployment} : containers include the minimal run time requirements of the application, reducing their size and allowing them to be deployed quickly.
\item
  \emph{Portability across machines} : an application and all its dependencies can be bundled into a single container that is independent from the host version of Linux kernel, platform distribution, or deployment model. This container can be transferred to another machine that runs Docker, and executed there without compatibility issues.
\item
  \emph{Version control and component reuse} : you can track successive versions of a container, inspect differences, or roll-back to previous versions. Containers reuse components from the preceding layers, which makes them noticeably lightweight. In addition due to Docker Hub it is possible to establish a connection between Git and DockerHub. Vesion
\item
  \emph{Sharing} : you can use a remote repository to share your container with others. It is also possible to configure a private repository hosted on Docker Hub.
\item
  \emph{Lightweight footprint and minimal overhead} : Docker images are typically very small, which facilitates rapid delivery and reduces the time to deploy new application containers.
\item
  \emph{Fault isolation} :Docker reduces effort and risk of problems with application dependencies. Docker also freezes the environment to the preferred packages version so that it guarantees continuity in deployment and isolate the container from system fails coming from package version updates.
\end{itemize}

The way to tell docker which system requirements are needed in the newly born software is a \emph{Dockerfile}.

\hypertarget{dockerfile}{%
\subsection{Dockerfile}\label{dockerfile}}

Docker can build images automatically by reading instructions from a Dockerfile. A Dockerfile is a text document that contains all the commands/rules a generic user could call on the CLI to assemble an image. Executing the command \texttt{docker\ build} from shell the user can trigger the image building. That executes sequentially several command-line instructions. For thesis purposes a Dockerfile is written with the specific instructions and then the file is pushed to GitHub repository. Once pushed DockerHub automatically parses the repository looking for a plain text file whose name is ``Dockerfile''. When It is matched then it triggers the building of the image.

The Dockerfile used to trigger the building of the docker container has the following sequential set of instructions in figure \ref{fig:dockerfile}) :

\begin{figure}
\centering
\includegraphics{images/dockerfile.PNG}
\caption{\label{fig:dockerfile}Example of a Dockerfile from Docker Hub, author's source}
\end{figure}

where the instructions are:

\begin{itemize}
\item
  \texttt{FROM\ rocker/tidyverse:latest} : The command imports a pre-built image by the rocker team that contains the latest (tag latest) version of base-R along with the tidyverse packages.
\item
  \texttt{MAINTAINER\ Niccolo\ Salvini\ "niccolo.salvini27@gmail.com"} : The command tags the maintainer and its e-mail contact information.
\item
  \texttt{RUN\ apt-get\ update\ \&\&\ apt-get\ install\ -y\ \textbackslash{}\ libxml2-dev\ \textbackslash{}\ libudunits2-dev} :The command update and install Linux dependencies needed for running R packages. \texttt{rvest} requires libxml2-dev and \texttt{magrittr} needs libudunits2-dev. If they are not installed then associated libraries can not be loaded. Linux dependencies needed have been found by trial and error while building containers. Building logs messages print errors and suggest which dependency is mandatory.
\item
  \texttt{RUN\ R\ -e\ "install.packages(c(\textquotesingle{}plumber\textquotesingle{},\textquotesingle{}tibble\textquotesingle{},\textquotesingle{}...\textquotesingle{},dependencies=TRUE)} : the command install all the packages required to execute the files (R files) containerized for the scraping. Since all the packages have their direct R dependencies the option \texttt{dependencies=TRUE} is needed.
\item
  \texttt{RUN\ R\ -e\ "install.packages(\textquotesingle{}https://cran.r-project.org/.../iterators,\ type=\textquotesingle{}source\textquotesingle{})}
  \texttt{RUN\ R\ -e\ "install.packages(\textquotesingle{}https://cran.r-project.org/.../foreach/,\ type=\textquotesingle{}source\textquotesingle{})}
  \texttt{RUN\ R\ -e\ "install.packages(\textquotesingle{}https://cran.r-project.org/.../doParallel,\ type=\textquotesingle{}source\textquotesingle{})}
  DoParallel was not available in package manager for R version later than 4.0.0. For this reason the choice was to install a previous source version by the online repository, as well as its dependencies.
\item
  \texttt{COPY\ \textbackslash{}\textbackslash{}} The command tells Docker copies all the files in the container.
\item
  \texttt{EXPOSE\ 8000} : the commands instructs Docker that the container listens on the specified network ports 8000 at runtime. It is possible to specify whether the port exposed listens on UDP or TCP, the default is TCP (this part needs a previous set up previous installing, for further online documentation It is recommended \citep{docker_documentation_2020} )
\item
  \texttt{ENTRYPOINT\ {[}"Rscript",\ "main.R"{]}} : the command tells docker to execute the file main.R within the container that triggers the API start. In main.R it are pecified both the port and the host where API expects to be exposed (in this case port 8000).
\end{itemize}

In order to make the system stand-alone and make the service available to a wider range of subjects a choice has to be made. The service has to have both the characteristics to be run on demand and to specify query parameters.

\hypertarget{aws}{%
\section{AWS EC2 instance}\label{aws}}

Exporting the API on a server allows to make scraping available to a various number of services thorough multitude of subjects. Since it can not be specified a-priori how many times and users are going to enjoy the service a scalable solutio might fill the needs. Scalable infrastructure through a flexible cloud provider combined with nginx load balancing can offer a stable and reliable infrastructure for a relatively cheap price.
AWS offers a wide range of services each of which for a wide range of budgets and integration. Free tier servers can be rent up to a certain amount of storage and computation that nearly 0s the total bill. The cloud provider also has a dedicated webpage to configure the service needed with respect to the usage named \href{https://aws.amazon.com/en/aws-cost-management/}{amazon cost manager}.

\begin{definition}[AWS EC2]
\protect\hypertarget{def:aws}{}{\label{def:aws} \iffalse (AWS EC2) \fi{} }Amazon Elastic Compute Cloud (EC2) is a web service that contributes to a secure, flexible computing capacity in the AWS cloud. EC2 allows to rent as many virtual servers as needed with customized capacity, security and storage.
\end{definition}

{[}few words still on EC2{]}

\hypertarget{launch-an-ec2-instance}{%
\subsection{Launch an EC2 instance}\label{launch-an-ec2-instance}}

The preliminary step is to pick up an AMI (Amazon Machine Image). AWS AMI are already pre set-up machines with standardized specifications built with the purpose to speed up choosing the a customed machine. Since the project is planned to be nearly 0-cost a ``Free Tier Eligible'' Linux server is chosen. By checking the Free Tier box all the available free tiers machines are displayed. The machine selected has this specification: t2.micro with 1 CPU and 1GB RAM and runs on a Ubuntu distribution OS. First set up settings needs to be left as-is, networking and VPC can always be updated when needed. In the ``add storage'' step 30 GB storage are selected, moreover 30 represent the upper limit since the server can be considered free tier. Tags windows are beyond the scope. Secondly configuration needs to account security and a new entry below SSH connection (port 22) has to be set in. New security configuration has to have TCP specification and should be associated to port 8000. Port 8000, as in dockerfile section \ref{dockerfile}, has been exposed and needs to be linked to the security port opened.

\begin{figure}
\centering
\includegraphics{images/aws.PNG}
\caption{aws\_dashboard}
\end{figure}

At this point instance is prepared to run and in a few minutes is deployed. Key pairs, if never done before, are generated and a .pem file is saved and securely stored. Key pairs are a mandatory step to log into the Ubuntu server via SSH. SSH connection in Windows OS can be handled with \href{https://www.putty.org/}{PuTTY}, which is a SSH and telnet client designed for Windows. At first PuTTYgen, a PuTTY extensions, has to convert the key pair .pem file into a .ppk extension (otherwise PuTTY can not read it). Once .ppk is converted is immediately sourced in the authorization panel. If everything works and authentication is verified then the Ubuntu server CLI appears and interaction with the server is made possible.
Once the CLI pops out some Linux libraries to check file structure (``tree'') and Docker are installed. Then a connection with Docker hub is established providing user login credentials. From the Hub repository the container image is pulled on the machine and is then executed with the docker RUN command.
AWS automatically assign to the server a unique Public DNS address which is going to be the REST API url to call.
the Public DNS has the following form:

\texttt{ec2-15-161-94-121.eu-south-1.compute.amazonaws.com}

\hypertarget{nginx}{%
\section{NGINX reverse proxy server}\label{nginx}}

For analysis purposes NGINX is open source software for reverse proxying and load balancing.
Proxying is typically used to distribute the load among several servers, seamlessly show content from different websites, or pass requests for processing to application servers over protocols other than HTTP.
{[}\ldots{]}

When NGINX proxies a request, it sends the request to a specified proxied server, fetches the response, and sends it back to the client. It is possible to proxy requests to an HTTP server (another NGINX server or any other server) or a non-HTTP server (which can run an application developed with a specific framework, such as PHP or Python) using a specified protocol. Supported protocols include FastCGI, uwsgi, SCGI, and memcached.
{[}\ldots{]}

.conf file and installation on Linux server. Security and Authentication.

\hypertarget{software-development-workflow}{%
\section{Software development Workflow}\label{software-development-workflow}}

\begin{figure}
\centering
\includegraphics{images/SoftwareDevWF.jpg}
\caption{\label{fig:sfmap}Sofwtare development workflow, author's source}
\end{figure}

\hypertarget{further-integrations}{%
\section{Further Integrations}\label{further-integrations}}

From a software point of view a more robust code can be obtained embedding recent R software development frameworks as \texttt{Golem} \citet{colin_fay_2020} into the existing code. The framework stimulates the usage of modules According to the latest literature APIs (as well as Shiny) should be treated as R packages (\citet{plungr} aligns with that) as argued in section 4.2 \citet{colin_fay_2020} and in the \href{https://deanattali.com/2015/04/21/r-package-shiny-app/}{comment} by Dean Attali. As a consequence of that \emph{TDD} (i.e.Test-Driven Development \citet{TDD_2004}) through the tools of \citet{usethis} and \citet{testthat} during package building can make software more robust, organized and production graded. Loadtest \citet{loadtest} can help figure out
CI/CT
It can be missed to cite a popular API development integration service and automate testing tool \href{https://www.postman.com/}{Postman}, which does the best when POST requests endpoints are served, since for the moment they are not required it is not used.

Pins is an r packages \href{https://rstudio.com/resources/rstudioconf-2020/deploying-end-to-end-data-science-with-shiny-plumber-and-pins/?mkt_tok=eyJpIjoiTmprNU1USXhPVEprWXpNMSIsInQiOiJtTUhKVzlvSjVIV2hKc0NRNVU1NTRQYSsrRGd5MWMyemlTazQ5b1lHRGJXNVBLcnpScjZRaWVcL2JGUjBPNGIwV3pwY1dKTW45cnhcL2JzZUlGWndtSFNJZVNaOUcyc1ZXcEJOcnppSVJXSGZRSVU1ZUY1YUU2NWdDamoxZG5VMHZcLyJ9}{this link}
software development framework and tools for testing \href{https://github.com/isteves/plumbplumb}{this work}

\hypertarget{inla}{%
\chapter{INLA computation}\label{inla}}

INLA \citep{Rue2009} stands for Integrated Nested Laplace approximation and constitutes a computational alternative to traditional MCMC methods. INLA does approximate Bayesian inference on special type of models called LGM (Latent Gaussian Models) due to the fact that they are \emph{computationally} convenient. The benefits are many, some among the other are:

\begin{itemize}
\tightlist
\item
  Low computational costs, even for large models.
\item
  It provides high accuracy.
\item
  Can define very complex models within that framework.
\item
  Most important statistical models are LGM.
\item
  Very good support for spatial models.
\item
  Implementation of spatio-temporal model enabled.
\end{itemize}

INLA uses a combination of analytics approximations and numerical integration to obtain an approximated posterior distribution of the parameters in a shorter time period.
The chronological steps in the presentation follows the course sailed by \citet{Moraga2019}, with the author choice to skip details. As a matter of fact the aim of the chapter is to provide a comprehensive intuition oriented to the immediate application of the methodology, without stepping too long on mathematical details. By the way details e.g model assessment and control options are handled under the hood by the package and can be tuned within the main function, most of them are covered by \citet{Bayesian_INLA_Rubio}. Notation is imported from \citet{Blangiardo-Cameletti} and \citet{Bayesian_INLA_Rubio}, and quite differ from the one presented in the original paper by Rue, Chopin and Martino \citeyearpar{Rue2009}. As further notation remarks: bold symbols are considered as vectors, so each time they occur they have to be considered like the \emph{ensamble} of their values. In addition \(\tilde\pi\) in section \ref{approx} are the Laplace approximation of the underlying integrals. Moreover the inner functioning of Laplace approximation and its special usage within the INLA setting is far from the scope, but an easy shortcut oriented to INLA is in \citet{Blangiardo-Cameletti}.

INLA can fit only Latent Gaussian type of models and the following work tries to encapsulate its properties. As a consequence a problem can be reshaped into the LGM framework with the explicit purpose to explore its benefits. When models are reduced to LGMs then joint posterior distribution can be rewritten and then approximated with INLA. A hierarchical bayesian structure on the model will help to integrate many parameter and hyperparameter levels and simplify interpretation.
Generic Information on the project and the R-INLA package are contained in the initial part to last section \ref{inla}. In the end a brief application on a toy spatial dataset is proposed with the aim to familiarize with the methodology and to come to grip with INLA results.

\hypertarget{LGM}{%
\section{Latent Gaussian Models LGM}\label{LGM}}

Given some observations \(y_{i \ldots n}\) in order to define a Latent Gaussain Model within the bayesian framework it is convenient to specify at first an \emph{exponential family} (Gaussian, Poisson, Exponential\ldots) distribution function characterized by some parameters \(\phi_{i}\) (usually expressed by the mean \(\left.E\left(y_{i}\right)\right)\)) and some other hyper-parameters \(\psi_{k} ,\forall k \in \ 1\ldots K\). The parameter \(\phi_{i}\) can be defined as an additive \emph{latent linear predictor} \(\eta_{i}\), as pointed out by Krainski and Rubio (\citeyearpar{Krainski-Rubio}) through a link function \(g(\cdot)\), i.e.~\(g\left(\phi_{i}\right)=\eta_{i}\). A comprehensive expression of the linear predictor takes into account all the possible effects on covariates

\[
\eta_{i}=\beta_{0}+\sum_{m=1}^{M} \beta_{m} x_{m i}+\sum_{l=1}^{L} f_{l}\left(z_{l i}\right)
\]

where \(\beta_{0}\) is the intercept, \(\boldsymbol{\beta}=\left\{\beta_{1}, \ldots, \beta_{M}\right\}\) are the coefficient that quantifies the linear effects on covariates \(\boldsymbol{x}=\left({x}_{1}, \ldots, {x}_{M}\right)\) and \(f_{l}(\cdot), \forall l \in 1 \ldots L\) are a set of random effects defined in terms of a \(\boldsymbol{z}\) set of covariates \(\boldsymbol{z}=\left(z_{1}, \ldots, z_{L}\right)\) (e.g.~rw, ar1). As a consequence of the last assumption the class of LGM can receive a wide range of models e.g.~GLM, GAM, GLMM, linear models and spatio-temporal models. This constitutes one of the main advantages of INLA, which can fit many different models, starting from simpler and ending with more complex. Contributors recently are extending the methodology to many areas as well as models moreover they are trying to incorporate INLA with non gaussian latent models as Rubio \citeyearpar{Bayesian_INLA_Rubio} pointed out.
All the latent components can be conveniently grouped into a variable denoted with \(\boldsymbol{\theta}\) such that: \(\boldsymbol{\theta}=\left\{\beta_{0}, \boldsymbol{\beta}, f\right\}\) and the same can de done for hyper parameters \(\boldsymbol{\psi} = \left\{\psi_{1}, \ldots, \psi_{K}\right\}\).
Then the probability distribution conditioned to parameters and hyper parameters is then:

\[
y_{i} \mid \boldsymbol{\theta}, \boldsymbol{\psi} \sim \pi\left(y_{i} \mid \boldsymbol{\theta},\boldsymbol{\psi}\right)
\]

Since data \(\left(y_{1}, \ldots, y_{n}\right)\) is drawn by the same distribution family but it is conditioned to parameters which are conditional independent (i.e.~\(\pi\left(\theta_{i}, \theta_{j} \mid \theta_{-i, j}\right)=\pi\left(\theta_{i} \mid \theta_{-i, j}\right) \pi\left(\theta_{j} \mid \theta_{-i, j}\right)\)) \citep{GMRFRue} then the joint distribution is given by the product of all the independent parameters i.e.~the likelihood. Moreover the Product operator index \(i\) ranges from 1 to \(n\), i.e.~\(\mathbf{I} = \left\{1 \ldots n \right\}\). When an observation is missing so the corresponding \(i \notin \mathbf{I}\) INLA automatically will not include it in the model avoiding errors \citeyearpar{Bayesian_INLA_Rubio}. As a consequence the likelihood expression is:

\begin{equation}
\pi(\boldsymbol{y} \mid \boldsymbol{\theta}, \boldsymbol{\psi})=\prod_{i \in \mathbb{I}} \pi\left(y_{i} \mid \theta_{i}, \boldsymbol{\psi}\right)
\label{eq:likelihood}
\end{equation}

Each data point is connected to one combination \(\theta_{i}\) out of all the possible linear combinations of elements in \(\boldsymbol{\theta}\) \emph{latent field}. The latent aspect of the field regards the undergoing existence of many parameter combination alternatives. Furthermore hyper parameters are by definition independent, in other words \(\boldsymbol{\psi}\) will be the product of many univariate priors \citep{Bayesian_INLA_Rubio}. A Multivariate Normal distribution is imposed on the latent field \(\boldsymbol{\theta}\) such that it is centered in 0 with precision matrix \(\boldsymbol{Q(\psi)}\) (the inverse of the covariance matrix \(\boldsymbol{Q}^{-1}(\boldsymbol{\psi})\)) depending only on \(\boldsymbol{\psi}\) hyper parameter vector i.e., \(\boldsymbol{\theta} \sim \operatorname{Normal}\left(\mathbf{0}, \boldsymbol{Q}^{-1}(\boldsymbol{\psi})\right)\). As a notation remark some authors choose to keep the covariance matrix expression as \(\boldsymbol{Q}\) and its inverse precision matrix as \(\boldsymbol{Q}^{-1}\). This is strongly not encouraged fro two reasons: the first is that the default hyperpramater option in INLA R package uses the precision matrix, the second it over complicates notation when writing down conditional expectation as Rue pointed out \emph{miss lit}. However notation for covariance function introduced in chapter \ref{Matern} i.e.~Matérn has to be expressed through covariance matrix, this passage will be cleared out in the dedicated section so that confusion is avoided.
The exponential family density function is then expressed through:

\begin{equation}
\pi(\boldsymbol{\theta} \mid \boldsymbol{\psi})=(2 \pi)^{-n / 2}| \boldsymbol{Q(\psi)}|^{1 / 2} \exp \left(-\frac{1}{2} \boldsymbol{\theta} \boldsymbol{Q(\psi)} \boldsymbol{\theta}\right)
\label{eq:gmrf}
\end{equation}

The conditional independence assumption on the latent field \(\boldsymbol{\theta}\) leads \(\boldsymbol{Q(\psi)}\) to be a sparse precision matrix since for a general pair of combinations \(\theta_{i}\) and \(\theta_{j}\) the resulting element in the precision matrix is 0 i.e.~\(\theta_{i} \perp \theta_{j} \mid \theta_{-i, j} \Longleftrightarrow Q_{i j}(\boldsymbol{\psi})=0\) \citeyearpar{Blangiardo-Cameletti}.
A probability distribution function with those characteristics is said \emph{Gaussian Markov random field} (\textbf{GMRF}). GMRF as a matter of fact are Gaussian variables with Markov properties which are encoded in the precision matrix \(\boldsymbol{Q}\) \citep{Rue2009}. (puoi dire di più)
From here it comes the source of run time computation saving, inherited using GMRF for inference. As a consequence of GMRF representation of the latent field, matrices are sparse so numerical methods can be exploited \citep{Blangiardo-Cameletti}. \emph{Moreover when Gaussian Process (see chapter \ref{GP}), which are used to integrate spatial components, are represented as GMRF through SPDE (Stochastic Partial Differential Equations) approach, then INLA can be used as a computing choice. This last assumption will be framed in chapter \ref{inla} and verified in chapter \ref{spde}.}
Once priors distributions are specified for \(\boldsymbol{\psi}\) then the joint posterior distribution for \(\boldsymbol{\theta}\) and \(\boldsymbol{\psi}\) is

\[
\pi(\boldsymbol{\theta}, \boldsymbol{\psi} \mid y)\propto  \underbrace{\pi(\boldsymbol{\psi})}_{\text {prior }} \times \underbrace{\pi(\theta \mid \psi)}_{\text {GMRF }} \times \underbrace{\prod_{i=1}^{n} \pi\left(y_{i} \mid \theta_{i}, \boldsymbol{\psi}\right)}_{\text {likelihood }}
\]

Last expression is said a Latent Gaussian Models, \textbf{LGM}, if the whole set of assumptions imposed since now are met. Therefore all models that can be reduced to a LGM representation are able to host INLA methodology.
Then plugging in the \emph{likelihood} \eqref{eq:likelihood} and \emph{GMRF} \eqref{eq:gmrf} expression the posterior distribution can be rewritten as

\[
\begin{aligned}
\pi(\boldsymbol{\theta}, \boldsymbol{\psi} \mid y) & \propto \pi(\boldsymbol{\psi}) \times \pi(\boldsymbol{\theta} \mid \boldsymbol{\psi}) \times \pi(y \mid \boldsymbol{\theta}, \boldsymbol{\psi}) \\
& \propto \pi(\boldsymbol{\psi}) \times \pi(\boldsymbol{\theta} \mid \boldsymbol{\psi}) \times \prod_{i=1}^{n} \pi\left(y_{i} \mid \theta_{i}, \boldsymbol{\psi}\right) \\
& \propto \pi(\boldsymbol{\psi}) \times|\boldsymbol{Q}(\boldsymbol{\psi})|^{1 / 2} \exp \left(-\frac{1}{2} \boldsymbol{\theta}^{\prime} \boldsymbol{Q}(\boldsymbol{\psi}) \boldsymbol{\theta}\right) \times \prod_{i}^{n} \exp \left(\log \left(\pi\left(y_{i} \mid \theta_{i}, \boldsymbol{\psi}\right)\right)\right)
\end{aligned}
\]

And by joining exponents by their multiplicative property it is obtained

\begin{equation}
\pi(\boldsymbol{\theta}, \boldsymbol{\psi} \mid y) \propto \pi(\psi) \times|\boldsymbol{Q}(\boldsymbol{\psi})|^{1 / 2} \exp \left(-\frac{1}{2} \boldsymbol{\theta}^{\prime} \boldsymbol{Q}(\boldsymbol{\psi}) \boldsymbol{\theta}+\sum^{n} \log \left(\pi\left(y_{i} \mid \theta_{i}, \boldsymbol{\psi}\right)\right)\right)
\label{eq:jointpostdistr}
\end{equation}

\hypertarget{approx}{%
\section{Approximation in INLA setting}\label{approx}}

INLA is not going to try to estimate the whole posterior distribution from expression \eqref{eq:jointpostdistr}. Instead it will try to estimate the posterior marginal distribution effects for each \(\theta_{i}\) combination in the latent parameter \(\boldsymbol{\theta}\), given the hyper parameter priors specification \(\psi_{k}\). Proper estimation methods however are beyond the scope of the analysis, further excellent references are suggested in their respective part by Rubio \citeyearpar{Bayesian_INLA_Rubio} in section 2.2.2 and Blangiardo \& Cameletti \citeyearpar{Blangiardo-Cameletti} in section 4.7.2.
The marginal posterior distribution function for each latent parameter element \(\theta_{i}\) is

\begin{equation}
  \pi(\theta_{i} \mid \boldsymbol{y})=\int \pi(\boldsymbol{\theta}, \boldsymbol{\psi} \mid \mathbf{y}) \pi(\boldsymbol{\psi} \mid \mathbf{y}) d \psi
\label{eq:latentparam}
\end{equation}

The posterior marginal integral for each hyper parameter \$\psi\_\{k\}, \forall k \in 1, \ldots, K \$ is

\[
\pi\left(\psi_{k} \mid y\right)=\int \pi(\boldsymbol{\psi} \mid y) d \psi_{-k}
\]

where the notation \(\psi_{-k}\) is a vector of hyper parameters \(\psi\) without considering \(k\)th element \(\psi_{k}\).

The goal is to have approximated solution for latent parameter posterior distributions. To this purpose A \emph{hierarchical procedure} is now imposed since the ``lower'' hyper parameter integral, whose approximation for the moment does not exist, is nested inside the ``upper'' parameter integral that takes hyper param as integrand. Hierarchical structures are welcomed very warmly since they are convenient later in order to fit a hierarchical bayesian model approached in the next chapter \ref{hiermod}. While many approximation strategies are provided and many others are emerging for both the hyper param and for the latent field, the common ground remains to unnest the structure in two steps such that:

\begin{itemize}
\tightlist
\item
  step 1: compute the Laplace approximation \(\tilde\pi\left(\boldsymbol{\psi} \mid \boldsymbol{y}\right)\) for each hyper parameters marginal: \(\tilde\pi\left(\psi_{k} \mid \boldsymbol{y}\right)\)
\item
  step 2: compute Laplace approximation \(\tilde{\pi}\left(\theta_{i} \mid \boldsymbol{\psi}, \boldsymbol{y}\right)\) marginals for the parameters given the hyper parameter approximation in step 1: \(\tilde{\pi}\left(\theta_{i} \mid \boldsymbol{y}\right) \approx \int \tilde{\pi}\left(\theta_{i} \mid \boldsymbol{\psi}, \boldsymbol{y}\right) \underbrace{\tilde{\pi}(\boldsymbol{\psi} \mid \boldsymbol{y})}_{\text {Estim. in step 1 }} \mathrm{d} \psi\)
\end{itemize}

Then plugging approximation in the integral observed in \eqref{eq:latentparam} it is obtained:

\[
\tilde{\pi}\left(\theta_{i} \mid y\right) \approx \int \tilde{\pi}\left(\theta_{i} \mid  \boldsymbol{\psi}, y\right) \tilde{\pi}(\boldsymbol{\psi} \mid y) \mathrm{d} \psi
\]

In the end INLA by its default approximation strategy thrpugh \emph{simplified Laplace approximation} uses the following numerical approximation to compute marginals:

\[
\tilde{\pi}\left(\theta_{i} \mid y\right) \approx \sum_{j} \tilde{\pi}\left(\theta_{i} \mid \boldsymbol{\psi}^{(j)}, y\right) \tilde{\pi}\left(\boldsymbol{\psi}^{(j)} \mid y\right) \Delta_{j}
\]

where \{\(\boldsymbol{\psi}^{(j)}\)\} are a set of values of the hyper param \(\psi\) grid used for numerical integration, each of which associated to a specific weight \(\Delta_{j}\). The more the weight \(\Delta_{j}\) is heavy the more the integration point is relevant. Details on how INLA finds those points is beyond the scope, but the strategy and grids seraches are offered in the appendix follwing both Rubio and Blangiardo.

\hypertarget{further-approximations-prolly-do-not-note-include}{%
\subsection{further approximations (prolly do not note include)}\label{further-approximations-prolly-do-not-note-include}}

INLA focus on this specific integration points by setting up a regular grid about the posterior mode of \(\psi\) with CCD (central composite design) centered in the mode \citep{Bayesian_INLA_Rubio}.

\begin{figure}
\centering
\includegraphics{images/CCDapplied.PNG}
\caption{CCD to spdetoy dataset, source \citet{Blangiardo-Cameletti}}
\end{figure}

The approximation \(\tilde{\pi}\left(\theta_{i} \mid y\right)\) can take different forms and be computed in different ways. \citet{Rue2009} also discuss how this approximation should be in order to reduce the numerical error \citep{Krainski-Rubio}.

Following \citet{Bayesian_INLA_Rubio}, approximations of the joint posterior for the hyper paramer \(\tilde\pi\left(\psi_{k} \mid \boldsymbol{y}\right)\) is used to compute the marginals for the latent effects and hyper parameters in this way:

\[
\left.\tilde{\pi}(\boldsymbol{\psi} \mid \mathbf{y}) \propto \frac{\pi(\boldsymbol{\theta}, \boldsymbol{\psi}, y)}{\tilde{\pi}_{G}(\boldsymbol{\theta} \mid \boldsymbol{\psi}, y)}\right|_{\boldsymbol{\theta}=\boldsymbol{\theta}^{*}(\boldsymbol{\psi})}
\]

In the previous equation \(\tilde{\pi}_{G}(\boldsymbol{\theta} \mid \boldsymbol{\psi}, y)\) is a gaussian approximation to the full condition of the latent effect \({\theta}^{*}(\boldsymbol{\psi})\) is the mode for a given value of the hyper param vector \(\boldsymbol{\psi}\)

At this point there exists three types of approximations for \(\pi\left(\boldsymbol{\theta} \mid \boldsymbol{\psi}, y\right)\)

\begin{itemize}
\tightlist
\item
  first with a gaussian approximation, estimating mean \(\mu_{i}(\boldsymbol{\psi})\) and variance \(\sigma_{i}^{2}(\boldsymbol{\psi})\).
\item
  second using the \emph{Laplace Approximation.}
\item
  third using \emph{simplified Laplace Approximation}
\end{itemize}

(rivedere meglio)

\hypertarget{rinla}{%
\section{R-INLA package in a bayesian hierarchical regression perspective}\label{rinla}}

\hypertarget{overview}{%
\subsection{Overview}\label{overview}}

INLA computations and methodology is developed by the R-INLA project whose package is available on their \href{http://www.r-inla.org}{website}. Download is not on CRAN (the Comprehensive R Archive Network) so a special source repo link, which is maintained by authors and collaborators, has to be optioned. The website offers also a forum where a daily discussion group is opened and an active community is keen to answer. Moreover It also contains a number of reference books, among which some of them are fully open sourced as gitbook. Furthermore as Havaard Rue has pointed out in a web-lecture on the topic, the project is gaining importance due to its new applications and recent use cases, but by no means it is replacing the older MCMC methods, rather INLA can integrate pre existing procedures.
The core function of the package is \texttt{inla()}and it works as many other regression functions like \texttt{glm()}, \texttt{lm()} or \texttt{gam()}. Inla function takes as arguments the formula (where are response and linear predictor), the data (expects a data.frame obj) on which estimation is desired together with the distribution of the data. Many other methods inside the function can be added through lists, such as \texttt{control.family} and \texttt{control.fixed} which let the analyst specifying priors distribution both for \(\boldsymbol{\theta}\) parameters, \(\boldsymbol{\psi}\) hyper parameters and the outcome precision \(\tau\), default values are non-informative.
\texttt{control.fixed} as said regulates prior specification through a plain list when there only a single fixed effect, instead it does it with nested lists when fixed effects are greater than 2, a guided example might better display the behaviour:
\texttt{control.fixed\ =\ list(mean\ =\ list(a\ =\ 1,\ b\ =\ 2,\ default\ =\ 0))}
In the chuck above it is assigned prior mean equal to 1 for fixed effect ``a'' and equal 2 for ``b''; the rest of the prior means are set equal to 0.
Inla objects are inla.dataframe summary-type lists containing the results from model fitting. Results contained in the object are specified in the table below, even though some of them requires special method: (se riesco più elegante in kable)
Following Krainski \& Rubio \citeyearpar{Krainski-Rubio} observations \(y(s_{1}), \ldots, y(s_{n})\) are taken from a toy generated dataset and a hierarchical linear regression is fitted.

\begin{figure}
\centering
\includegraphics{images/summarytable.PNG}
\caption{summary table list object, source: \citet{Krainski-Rubio}}
\end{figure}

\hypertarget{example}{%
\subsection{Linear Predictor}\label{example}}

SPDEtoy dataset, that has a spatial component, is generated from a \(y_{i}\) Gaussian variable; its moments are \(\mu_{i}\) and precision \(\tau\).

\begin{figure}
\centering
\includegraphics{images/cotour_toy.png}
\caption{SPDEtoy plot, author's source}
\end{figure}

The formula that describe the linear predictor has to be called directly inside the \texttt{inla()} function or it can be stored in the environment into a variable. The mean moment in the gaussian distribution \(\mu_{i}\) is expressed as the \emph{linear predictor} \(\eta_{i}\) (i.e.~\(E\left(y_{i} \mid \beta_{0}, \ldots, \beta_{M}, x_{i 1}, \ldots, x_{i M}\right) = \eta_{i}\) ). The function that maps the linear predictor into the parameter space is identity as in the initial part of section \ref{LGM} i.e.~\(\eta_{i}=\beta_{0}+\sum_{m=1}^{M} \beta_{m} x_{m i}+\sum_{l=1}^{L} f_{l}\left(z_{l i}\right)\).
After including \(s_{1}\) and \(s_{2}\) spatial covariates the linear predictor takes the following form: \(\beta_{0}+\beta_{1} s_{1 i}+\beta_{2} s_{2 i}\), where once again \(\beta_{0}\) is the fixed effect i.e.~intercept and the \(\beta_{j}\) are the linear effect on covariates. INLA allows also to include non-linear effects with the \texttt{f()} method inside the formula. \texttt{f} are foundamental since they are used to incorporate the spatial component in the model through the Matérn covariance function, this will be shown in section (boh).
Once the formula is decided then priors has to be picked up; for the intercept a customary choice is uniform. Prior for Gaussian latent parameters are vague and they have 0 mean and 0.001 precision, then the prior for \(\tau\) is a Gamma with parameters 1 and 0.00005. Prior initial choice can be later adapted.

The summary of the model parameters is:

\[
\begin{aligned}
y_{i} & \sim N\left(\mu_{i}, \tau^{-1}\right), i=1, \ldots, 200 \\
\mu_{i} &=\beta_{0}+\beta_{1} s_{1 i}+\beta_{2} s_{2 i} \\
\beta_{0} & \sim \text { Uniform } \\
\beta_{j} & \sim N\left(0,0.001^{-1}\right), j=1,2 \\
\tau & \sim G a(1,0.00005)
\end{aligned}
\]

\begin{Shaded}
\begin{Highlighting}[]
\KeywordTok{data}\NormalTok{(}\StringTok{"SPDEtoy"}\NormalTok{)}
\NormalTok{formula =}\StringTok{ }\NormalTok{y }\OperatorTok{~}\StringTok{ }\NormalTok{s1 }\OperatorTok{+}\StringTok{ }\NormalTok{s2}
\NormalTok{m0 =}\StringTok{ }\KeywordTok{inla}\NormalTok{(formula, }\DataTypeTok{data =}\NormalTok{ SPDEtoy)}
\end{Highlighting}
\end{Shaded}

\begin{tabular}{lrrrrrrr}
\toprule
  & mean & sd & 0.025quant & 0.5quant & 0.975quant & mode & kld\\
\midrule
(Intercept) & 10.1321487 & 0.2422118 & 9.6561033 & 10.1321422 & 10.6077866 & 10.1321497 & 7e-07\\
s1 & 0.7624296 & 0.4293757 & -0.0814701 & 0.7624179 & 1.6056053 & 0.7624315 & 7e-07\\
s2 & -1.5836768 & 0.4293757 & -2.4275704 & -1.5836906 & -0.7404955 & -1.5836811 & 7e-07\\
\bottomrule
\end{tabular}

The output offers among the others a summary of the posterior marginal values for intercept, coefficient and covariates, as well as precision. Below the plots for the parameters and hyperparameters. From the summary it can be seen that the mean for s2 is negative, so the more the value of the y-coordinates increases the more the output decreases, that is truer looking at the SPDEtoy cotour plot. Plots can be generated by calling the \texttt{plot} function on the inla object, however the one generated below are \texttt{ggplot2} outputs coming from the \$marginals.fixed list object.

\begin{figure}
\centering
\includegraphics{images/marginal_distr.png}
\caption{linear predictor marginals, author's creation}
\end{figure}

R-Inla also has r-base fashion function to compute statistics on marginal posterior distributions for the density, distribution as well as the quantile function respectively \texttt{inla.dmarginal}, \texttt{inla.pmarginal} and \texttt{inla.qmarginal}. One major option which is conveniently packed into a dedicated function computes the higher posterior density credibility interval \texttt{inla.hpdmarginal} for a given covariate's coefficient, such that \(\int_{q_{1}}^{q_{2}} \tilde{\pi}\left(\beta_{2} \mid \boldsymbol{y}\right) \mathrm{d} \beta_{2}=0.90\) zwith .1 Confidence Level, in table @ref(tab:higer\_posterior\_density\_interval).

\begin{tabular}{lrr}
\toprule
  & low & high\\
\midrule
level:0.9 & -2.291268 & -0.879445\\
\bottomrule
\end{tabular}

Recall that the interpretation is different from the frequentist: in Bayesian statistics \(\beta_{j}\) comes from probability distribution, while frequenstists considers \(\beta_{j}\) as fixed unknown quantity whose estimator (random variable conditioned to data) is used to infer the value \citeyearpar{Blangiardo-Cameletti}.

\hypertarget{prdm}{%
\chapter{Point Referenced Data Modeling}\label{prdm}}

Geostatistical data are a collection of samples of geo type data indexed by coordinates (e.g.~latlong, eastings and northings) that originate from a spatially continuous phenomenon \citep{Moraga2019}. Data as such can monitor a vast range of phenomena, as an example disease cancer detection \citep{Bell2006} at several sites, COVID19 spread in China \citep{Li_Li_Ding_Hu_Chen_Wang_Peng_Shen_2020}, PM pollution concentration in a North-Italian region Piemonte \citep{Cameletti2012}. Moreover house prices variation, as observed in \citet{Bayesian_INLA_Rubio}, where selling prices smoothly vary between closer neighborhoods. All the Examples taken before might document a spatial nature of data according to which closer observations can display similar values, this phenomenon is named spatial autocorrelation. Spatial autocorrelation conceptually originates from geographer Waldo Tobler whose famous quote, known as first law of geography, inspires geostatisticians:

\begin{quote}
``Everything is related to everything else,
but near things are more related than distant things''

\hfill --- Waldo R. Tobler
\end{quote}

Spatial models are explicitly designed to take into account this behavior and can separate spatial patterns from simply random spatial variance.
Spatial data can be partitioned into three spatial data type whose modeling tools are specific with respect to their category.

\begin{itemize}
\tightlist
\item
  Areal Data
\item
  \textbf{Point Referenced Data}
\item
  Point Pattern Data
\end{itemize}

\begin{figure}
\centering
\includegraphics{images/map.png}
\caption{point referenced data example, Milan Rental Real Estate, Author's Source}
\end{figure}

REST API designed in chapter \ref{Infrastructure} extracts point referenced data, so modeling methodologies described in this analysis will exclusively take into account point referenced techniques.
In order to extend the notion from discrete measurements (i.e.~point referenced) to a continuous spatial surface a stochastic process, namely Gaussian Process, has to be introduced and constrained to Stationarity and Isotropy. GP are then evaluated with a specific covariance function, i.e.~Matèrn. The reason why Matérn is selected as candidate for covariance function will be much more clear in the next chapter \ref{spde}.
Hedonic Price Models are at first introduced and then a brief literature review is offered. Hedonic Prices brings to this work the theoretical basis but they do not suggest estimation methods, which are essentially the major issue in geostatistics. For this reason Hedonic Models are exploited into a spatial bayesian regression framework with the aim to apply INLA (seen in chpater \ref{inla}) methodology.
At first standard Bayesian regression is presented as introduction, then the spatial component in the form of a GP is added to the model. Many parameters are considered so far, as a consequence a hierarchy structure is imposed. To this extent an entire section is dedicated to hierarchy which simplifies model building and methodology understanding as well as allowing to bring in many different parameters that come from different levels through the exchangeability property.
As a matter of fact parameters originate from the Gaussian latent field, but also from Matérn covariance function tuning hyper parameters.
Then INLA is applied and a GMRF representation of GP is\ldots{}
Spatial kriging is essential to predict the process at new locations so that the spatial surface can be plotted and analyzed.
In the end models have to be checked and verified with resampling schemes which are once again specific to the data type and the scope of the analysis.

\emph{(forse mettere alla fine come further developments)}
As a side note Spatial data can also be measured according to a further dimension which is the Time. Latest literature suggests that spatio temporal models are the most accurate, as a consequence it might be interesting to research time correlation between subsequent spatial data time points, a valuable reference is offered in \citet{PACI2017149}. This will not take an enormous effort due ti the fact that on a daily basis REST API generates data which are stored as .json file on a DB. Future research on this data might consider the idea to include the time component in the model.

\hypertarget{GP}{%
\section{Gaussian Process (GP)}\label{GP}}

For simplicity lets consider \(y\) point of interest observations \(y\left(\boldsymbol{s}_{1}\right),y\left(\boldsymbol{s}_{2}\right), \ldots, y\left(\boldsymbol{s}_{n}\right)\)
from a random spatial process \(Y\), such that: \(Y\left(\boldsymbol{s}_{1}\right),Y\left(\boldsymbol{s}_{2}\right), \ldots, Y\left(\boldsymbol{s}_{n}\right)\) observed at location \(\boldsymbol{s}_{1}, \ldots, \boldsymbol{s}_{n}\). In the context of geostatistical data each observation has to be considered as a partial realization of an unobserved random spatial process.
\(\left\{Y(s): s \in D \subset \mathbb{R}^{2}\right\}\), where surface \(D\) is a subset of r-dimensional Euclidean space \(\mathbb{R}^{r}\).
Moreover When \(r = 1\) it is the most simple stochastic process widely explored in literature i.e.~time series process. However geostatistical data always have \(r = 2\) (i.e.~lat and long, eastings and northings) or eventually \(r = 3\), when elevation data is available. The stochastic process \(Y\) is observed in a fixed set of ``monitoring stations'' and inference can be done regarding moments of the realized process. This information are essential to build a spatially continuous surface over the y-studied variable in order to predict the phenomenon at locations not yet observed.

\begin{figure}
\centering
\includegraphics{images/prdprocess.png}
\caption{3D scatterplot and surface, Stockton data.}
\end{figure}

\begin{definition}[GP definition]
\protect\hypertarget{def:GP}{}{\label{def:GP} \iffalse (GP definition) \fi{} }A collection of \(n\) random variables, such as \(Y(s_{1}), Y(s_{2}) , \ldots, Y(s_{n})\) that are \emph{valid} spatial processes are said to be a \textbf{GP} if for any set of spatial index \(n\) and for each set of corresponding locations \(\left\{y\left(s_{1}\right), \ldots, y\left(s_{n}\right)\right\}\) follows a multivariate \emph{Gaussian} distribution with mean \(\boldsymbol{\mu}=\left\{\mu\left(s_{1}\right), \ldots, \mu\left(s_{n}\right)\right\}\) and covariance matrix \(\mathbf{Q}^{-1}_{i,j}, \forall i \neq j\)
\end{definition}

Even tough sometimes it is more convenient to express the covariance matrix as its inverse i.e.~precision matrix \(\boldsymbol{Q}_{i,j}\) \citep{Blangiardo-Cameletti}. The covariance matrix relates each observation to each of the others through a covariance function defined as \(\mathcal{C}(\cdot)\).

GP in the spatial context must check two important properties in order to exploit INLA, even though both of these assumptions can be relaxed:

\begin{itemize}
\tightlist
\item
  \textbf{Stationary}.
\item
  \textbf{Isotropy}.
\end{itemize}

\textbf{Stationarity} in a stochastic process can be \emph{strong}, \emph{weak} or \emph{intrinsic}. The strong property forces the distribution of the process \(\left\{y\left(s_{1}\right), \ldots, y\left(s_{n}\right)\right\}\) for any given spatial index \(n\) and its correspondent location sets \(s_{1,\ldots,n}\) to be the same as the one in \(\left\{y\left(s_{1}+\boldsymbol{h}\right), \ldots, y\left(s_{n}+\boldsymbol{h}\right)\right\}\), where \(h\) is a number belonging to \(R^{2}\).
On the other hand the weak property ensures that if the GP mean moment is constant over the study domain \(\mu(\mathbf{s}) \equiv \mu\) (e.g.~\(E[Y(s)]=\mu, \forall s \in D\)) then the covariance functions does depend only on the distance (euclidean \(\left\|s_{i}-s_{j}\right\|\) distance) between each couple points.
Weak stationarity consequences are the most interesting: It does not matter whether observations are placed either in a specific region, nor the direction towards they are oriented, the covariance functions \(\mathcal{C}(h)\) can summarize the process through the separation vector \(\mathbf{h}\) i.e.~\(\mathcal{C}(\mathbf{s}, \mathbf{s}+\mathbf{h})=\mathcal{C}(\mathbf{h}), \forall \mathbf{h} \in \mathbb{R}^{r}\) \citep{Banerjee-Gelfand}. In other words weak stationarity in GP implies being invariant under \emph{translation} \citeyearpar{Krainski-Rubio}. The relationship between strong and weak is not bijective since being strong implies also being weak, but the opposite is not always true for non-Gaussian process.
Furthermore through the intrinsic stationary property it is meant that \(E[Y(\mathbf{s}+\mathbf{h})-Y(\mathbf{s})]=0\), the second moment of the latter expression can be written as \(E[Y(\mathbf{s}+\mathbf{h})-Y(\mathbf{s})]^{2}\) leading to \(\operatorname{Var}(Y(\mathbf{s}+\mathbf{h})-Y(\mathbf{s}))\). Last expression is called \emph{variogram} and can be expressed with \(2 \gamma(\mathbf{h})\), even tough its half,i.e.~\(\gamma(\mathbf{h})\), is more interpretable, namely \emph{semivariogram} \citep{Cressie_2015}.

Semivariograms are characterized by mainly 3 tuning parameters:

\begin{itemize}
\tightlist
\item
  \emph{range} \(\sigma^{2}\): At some offset distance, the variogram values will stop changing and reach a sort of ``plateau''. The distance at which the effect occurs is called the range \(\frac{\Delta\gamma(\mathrm{h})}{h} \approx 0\).
\item
  \emph{sill} \(\tau^{2}\): The ``plateau'' value at which the variogram stops changing \(\frac{\Delta\gamma(\mathrm{h})}{h} = 0\).
\item
  \emph{nugget} \(\tau^{2}+\sigma^{2}\): The discontinuity at the origin. Although this theoretically should be zero, sampling error and short scale variability can cause it to be non-zero \(\gamma(\mathrm{0})\).
\end{itemize}

\begin{figure}
\centering
\includegraphics{images/variogram.png}
\caption{variogram example}
\end{figure}

presi i dati con le relative distanze euclidee a coppie di punti si binnano le distanze grazie ad un offset ottenendo i valori per il semivariogram. ottenuti i valori si fitta il semivargiogram a quei valori, un modo è la likelihood. A questo punto si calcolano le tre grandezze nugget sill e range per poi poter far uscire le funzioni di covarianza.

The process is said to be \textbf{Isotropic} if the covariance function depends only on the between-points distance \(\left\|\mathbf{h}\right\|\) so it is invariant under \emph{rotation} \citeyearpar{Krainski-Rubio}. A further way of seeing the property is that Isotropy implies concentric decaying contours that resemble the vanishing of spatial dependence, and so covariance values too.
then if the last assumption does not hold and direction towards point are distant from each other matters within the spatial domain \(D\), then is said to be \textbf{Anisotropic}.
Formalizing the results:

\[\mathcal{C}(\mathbf{h})=\mathcal{C}(\|\mathbf{h}\|)\]

\begin{figure}
\centering
\includegraphics{images/isotropyVSanisotropy.png}
\caption{isotropy VS anisotropy, source \citet{blanchetscalliet}}
\end{figure}

\hypertarget{spatial-covariance-function}{%
\section{Spatial Covariance Function}\label{spatial-covariance-function}}

The covariance function \(\mathcal{C}(\cdot)\) ensures that all the values that are close together in input space will produce output values that are close together. \(\mathcal{C}(\cdot)\) needs to inherits the \emph{validity} characteristics from the random spatial process, furthermore it has to be \emph{positive definite}.
In addition covariance function must share characteristic properties of functions, such as:

(cerca di capire queste\ldots)

\begin{itemize}
\tightlist
\item
  Multiply valid covariance functions (summing independent random variables)
\item
  Mixing covariance functions (mixing distributions)
\item
  Convolving covariance functions, this will be very important \ldots{}
\end{itemize}

Covariance functions under stationary and isotropic GPs displays two important properties: they are constant in mean within \(D\) i.e.~\(\mathcal{C}(\mathbf{s}, \mathbf{s}+\mathbf{h})=\mathcal{C}(\mathbf{h}), \forall \mathbf{h} \in \mathbb{R}^{r}\) and they depends on distance vector \(\mathbf{h}\), not direction i.e.~\(\mathcal{C}(\mathbf{h})=\mathcal{C}(\|\mathbf{h}\|)\)
There are many covariance functions and ways to relate distant points on a spatial domain \(D\). Typically the choice of the Covariance can depend either on data or the scope of the analysis. Covariance functions are wrapped into special hyper parameters which are mainly three:

\begin{enumerate}
\def\labelenumi{\arabic{enumi}.}
\tightlist
\item
  \emph{Range}: At some offset distance, the variogram values will stop changing and reach a ``plateau''. The distance at which this occurs is called the range.
\item
  \emph{Sill}: The ``plateau'' value at which the variogram stops changing.
\item
  \emph{Nugget}: The discontinuity at the origin. Although this theoretically should be zero, sampling error and short scale variability can cause it to be non-zero
\end{enumerate}

(espressione della covariance function insieme a alle \(\sigma^2\) come: \(C(\mathbf{s}+\mathbf{h}, \mathbf{s} \mid \theta)=\sigma^{2} \mathbf{R}(\|h\| ; \phi)\))
spiega anche queste due sotto

\[
\mathbf{w}=\left(w\left(\mathbf{s}_{1}\right), \ldots, w\left(\mathbf{s}_{n}\right)\right)^{\prime} \sim \mathrm{N}\left(\mathbf{0}, \sigma^{2} \mathbf{R}(\phi)\right) \text { where } \left.\mathbf{R}(\phi)_{i j}=\rho\left(\left\|\mathbf{s}_{i}-\mathbf{s}_{j}\right\| ; \phi\right)\right)
\]

\(\Sigma_{\theta}=\sigma^{2} \mathbf{R}(\phi)+\tau^{2} I_{n}\)

A summary of the most used covariance functions are presented below.

\[
\begin{aligned}
&\text { Exponential } \quad \mathcal{C}(\mathbf{h})=\left\{\begin{array}{cl}
\tau^{2}+\sigma^{2} & \text { if }  h=0 \\
\sigma^{2} \exp (-\phi h) & \text { if } h>0 
\end{array}\right.\\
&\text { Gaussian } \quad \mathcal{C}(\mathbf{h})=\left\{\begin{array}{cl}
\tau^{2}+\sigma^{2} & \text { if } h=0 \\
\sigma^{2} \exp \left(-\phi^{2} h^{2}\right) & \text { if } h>0 
\end{array}\right. \\
&\text { Matérn } \quad \mathcal{C}(\mathbf{h})=\left\{\begin{array}{cl}
\tau^{2}+\sigma^{2} & \text { if } h=0 \\
\frac{\sigma^{2}}{2^{\nu-1} \Gamma(\nu)}(\phi h)^{\nu} K_{\nu}(\phi h) & \text { if } h>0
\end{array}\right.
\end{aligned}
\]

\hypertarget{Matern}{%
\subsection{Matérn Covariance Function}\label{Matern}}

Matérn is special since when it is used together with a stationary and isotropic GP, the SPDE approach can provide a GMRF representation of the same process, chapter \ref{spde} discloses this fundamental property.
Matérn can also be accounted as the most used in geostatistics \citep{Krainski2018} and \citep{Bayesian_INLA_Rubio} and is tuned mainly by two parameters, a scaling one \(\kappa>0\), usually set equal to the range by the relation \(\sigma^{2}=\frac{\sqrt{8 \lambda}}{\kappa}\)) and a smoothing one \(\nu>0\). A \emph{stationary} and \emph{isotropic} Matérn covariance function has this form:

\[
\mathcal{C}(\mathbf{h})=\left\{\begin{array}{ll}
\tau^{2}+\sigma^{2} & \text { if } h=0 \\
\frac{\sigma^{2}}{2^{\nu-1} \Gamma(\nu)}(\phi t)^{\nu} K_{\nu}(\phi t) & \text { if } h>0
\end{array}\right.
\]

\(\Gamma(\nu)\) is a Gamma function depending on \(\nu\) values, \(K_{\nu}(\cdot)\) is a modified Bessel function of second kind. The smoothness parameter \(\nu\) in the figure below takes 4 different values showing the potentiality of Matérn to relates distances to covariance values. When \(\nu = 1\) \ldots{} When \(\nu = 1/2\) it becomes the exponential covariance function, When \(\nu = 3/2\) it uncovers a convenient closed form, when \(\nu \approx \infty\), in this case for representation purposes \(\nu = 80\) it becomes Gaussian covariance function.

\begin{figure}
\centering
\includegraphics{images/matern.png}
\caption{matern correlation function for 4 diff values of nu with phi fixed, author's source}
\end{figure}

ancora di più su matern, forse di più in spde

\hypertarget{hedonic-models-literature-review-and-spatial-hedonic-price-models}{%
\section{Hedonic models Literature Review and Spatial Hedonic Price Models}\label{hedonic-models-literature-review-and-spatial-hedonic-price-models}}

The theoretical foundation of the Hedonic Price Models (from now on HPM) resides in the consumer utility theory of \citet{Lancaster} together with \citet{Rosen} market equilibrium. According to Lancaster the utility of a commodity does not exist by itself, instead it exists as the sum of the utilities associated to its separable characteristics. Integrating Lancater, Rosen introduces HPM and suggests that each separate commodity characteristics are priced by the markets on the basis of supply and demand equilibrium. Applying HPM to Real Estate in a market context, from the buy side house prices (but also rents) are set as the unit cost of each household attributes, conversely from the selling side the expenditures associated to build of each them.
Formalizing the results, Hedonic Price \(P\) in Real Estate is expressed as a general \(f\) functional form that takes as input the house characteristics vector \(\mathbf{C}\).

\[P=f\left(c_{1}, c_{2}, c_{3}, \ldots, c_{n}\right)\]

Vector \(\mathbf{C}\) since now might contain a unidentified and presumably vast number of ungrouped characteristics. In this setting \citet{Malpezzi} tried to organize house features by decomposing \(\mathbf{C}\) into mutually exclusive and exhaustive subgroups. An overview of the vector components involved is given by \citet{Ling} according to which \(P\) represents the house price, \(S\) is the structural characteristics of the house, \(N\) represents the neighborhood characteristics, \(L\) signifies the locational characteristics, \(C\) describes the contract conditions and \(T\) is time. \(\beta\) is the vector of the parameters to be estimated. Then

\[P=f\left(S, N, L, C, T, \beta\right)\]

Historically a first attempt to include spatial effect in urban economic literature is provided by \emph{Alonso (1964) miss ref}. Its contribution was to raise voice on house prices (also rent) mainly depending on land price and a number of purely spatial covariates like CBD, the distance from City Business District. Other covariates were transport cost per kilometer and community income, even though they were defined also as spatial parameters through distances. The model proposed by Alonso is called monocentric since the centroid from which distances are calculated is only one. Moreover a first touch to spatial data thory was done since the CBD was defined as areal unit with well-defined boundaries of regular or irregular shape. However applications of the model were not convincing since empirical studies offered a different picture. Results instead displayed a Poly-centric areal structure (universities and Malls) which might be better explaining prices. The model also assumed that covariates like CBD are only informative within city center boundaries and then displayed no significance out of the core of the city. Poly-centric theory was also more coherent with the architectural and socio-economical evolution of cities during that times, therefore mono centric theory was then criticized and abandoned. Critics regarded also neighborhood quality measure and boundary problems \emph{Dubin (1987) miss ref}. Dubin for these reasons developed a model including areal effects in the error term since handling these covariates was posing several hard challenges. Areal data choice for Dubin was forced since he was interested in land values, geostatics interest was not a focus also due to the difficulties in gathering accurate data. Coming to recent literature a change in focus has been made by switching from theory based model to estimation methods. As a consequence to the change in focus \citet{Ling} said that practitioners should spend more time in variable selection and model specification with respect to their specific need.
As Ling has observed the emerging trends are in the field of semi-parametric and non-parametric methods \citeyearpar{Ling}. Historically semi-parametric regression considers models indexed by spatial coordinates \emph{Pace RK (1995)}. At the same time \emph{Kammann and Wand (2003)} gave birth to geoadditive models where the spatial component is added as a covariate. {[}\ldots{]}

A further aspect of the problem is posed by scholars that do not consider rents to be representative for the actual value of real estate. Nevertheless in empirical analysis rent value are considered a proxy for real estate pricing \citep{Herath_Maier_2011}. A further argument to endorse this hypothesis is brought by \citet{sellingVSrental} considering housing a commodity, then the selling or the rental should be considered interchangeable economic actions with respect to same inner need to be satisfied. This is also truer to the thesis' extent since Manganelli, Morano, and Tajani have centered their analysis exactly on italian real estate data. Moreover \citet{Capozza_Seguin_1996} discussed on how much rent-price ratio predicts future changes both in rents and prices. Among all the other discussions raised they brought the decomposition of rent-price ratio into two parts: the predictable part and the unexplained residuals part. The predictable part was discovered to be negatively correlated with price changes, in other words cities in which prices are relatively high with respect to rents are associated with higher capital gains that might justify that misalignment. This is also true for the opposite, that is cities in which prices are lower with respect to the rents, and this effect can not be associated to any local condition, realize lower capital gains. A further argument is offered by Clark \citep{Clark_1995} which went after the Capozza and Seguin work. Rent-price ratio is negatively correlated with following future changes in rents. In other words prices are still higher when areas in which they are observed documents an increase in rent prices. All the literature review above is oriented to a long-run alignment of price and rent.

\hypertarget{univariateregr}{%
\section{Point Referenced Regression for univariate spatial data}\label{univariateregr}}

Since in HPM the relationships between the characteristics of the house, i.e.~vector \(\mathbf{C}\) and the price \(P\) is not in any case fixed by econometric literature it is possible to assume any \(f\) functional form. The open possibility to apply a wide range of relationship between covariates fit in the INLA setting, since Latent Gaussian Models are prepared to accept a any linear and non linear \(f\) functions \ref{LGM} through the \texttt{f()} method. Hedonic price models are, as a consequence, a subset of models that can be fitted into LGM and therefore by INLA method.

Moreover what the vast majority of econometric literature \emph{(Greene, 2018)} suggest to apply a is log-linear / square root model. This is due to the fact that log transformation / square root smooths the skewness of prices normalizing the curve, leading to more accurate estimates. Having an exponential family generating process lowers even further computational cost for reasons linked to the \(\tilde\pi(\boldsymbol{\psi})\) hyper param INLA approximation \citep{Blangiardo-Cameletti}. Notation is taken from the previous chapter \ref{inla}, for brevity purposes \(\boldsymbol{\beta}\) \(\mathbf{X}\) and \(\boldsymbol{y}\) indicates vectors incorporating all their respective realizations and the \(s\) spatial component is left out in favor of the observation pedix \(i\).\\
The simplest log-linear bayesian regression model assumes linear relationship between predictors and a Normal data generating process: (log has been taken out for simplicity, bu it will be then considered in the regression setting) (valuta l'idea che per interpretabilità di modellarla come Gamma exponential family anzichè tenerla normale)

\[
\log{(y_{i})} \sim \operatorname{Normal}(\mu_{i}, \sigma^{2})
\]

\[
y_{i}=\mu_{i}+\varepsilon_{i}
\]

then by the following relationship \(E\left(y_{i} \mid \beta_{0}, \ldots, \beta_{M}, x_{i 1}, \ldots, x_{i M}\right)=\beta_{0}+\sum_{m=1}^{M} \beta_{m} x_{i m}\) it is possible to specify a more general linear predictor (seen also in chapter \ref{inla}) through an identity link function i.e.~\(\eta_{i}=g\left(\mu_{i}\right)=\mu_{i}\) obtaining:

\[
\eta_{i}=\beta_{0}+\sum_{m=1}^{M} \beta_{m} x_{m i}+\sum_{l=1}^{L} f_{l}\left(z_{l i}\right)
\]

Where, once again, the mean structure linearly depends on some \(\mathbf{X}\) covariates, \(\boldsymbol{\beta}\) coefficients, \(f_{l}(\cdot), \forall l \in 1 \ldots L\) are a set of random effects defined in terms of a \(\boldsymbol{z}\) set of covariates \(\boldsymbol{z}=\left(z_{1}, \ldots, z_{L}\right)\) (e.g.~rw, ar1) and \(\varepsilon_{i}\) white noise error.
Priors have to be specified and a non informativeness for \(\tau^2 = 1/\sigma^2\) and \(\boldsymbol{\beta}\) is chosen, such that \(\pi(\tau^2) \propto 1\) and \(\pi(\boldsymbol\beta) \propto 1\). As a consequence the conditional posterior for the parameters of interest \(\boldsymbol{\beta}\) is:

\[
\boldsymbol{\beta} \mid \sigma^{2}, \boldsymbol{y}, \boldsymbol{X} \sim \operatorname{MVNormal}\left(\left(\boldsymbol{X}^{\prime} \boldsymbol{X}\right)^{-1} \boldsymbol{X}^{\prime} \boldsymbol{y}, \sigma^{2}\left(\boldsymbol{X}^{\prime} \boldsymbol{X}\right)^{-1}\right)
\]

where the mean structure corresponds to the OLS estimator: \(\left(\boldsymbol{X}^{\prime} \boldsymbol{X}\right)^{-1} \boldsymbol{X}^{\prime} \boldsymbol{y}\) for \(\beta\) and then to obtain the marginal posterior for \(\boldsymbol{\beta}\) it is needed to integrate with respect to \(\sigma^2\).

In order to engage the spatial coordinate components into the regression setting \(w_{i}\) has to be added to the equation. \(w_{i}\) is set as a stationary and isotropic GP with mean 0 and variance as covariance function expressed as Matérn.
Recall that GP
The new regression setting integrates the \emph{spatial error} part in the name of \(w_{i}\) and a \emph{non-spatial error} part \(\varepsilon_{i}\) distributed normally with mean 0 and variance \(\tau^2\) ,i.e.~\(\mathrm{N}\left(0, \tau^{2}\right)\), which offers its contribution error to the nuggets via the covariance function.
Consequently there is one more parameter to estimate. It is worth mentioning that the distribution of \(w_{i}\) at a finite number of points is considered a realization of a multivariate Gaussian distribution. In this case, the likelihood estimation is possible and it is the multivariate Gaussian distribution with covariance \(\Sigma\).

\[
\log(y_{i})= \beta_{0} + (\mathbf{X})^{\prime}\boldsymbol{\beta}+w_{i}+\varepsilon_{i}
\]

The covariance of the marginal distribution of \(y_{i}\) at a finite number of locations is \(\Sigma_{y} = \Sigma + \tau^2\mathbf{I}\), where \(\mathbf{I}\) denotes the indicator function (i.e., \(\mathbf{I}(i = i^{\prime})= 1\) if \(i = i^{\prime}\), and 0 otherwise). This is a short extension of the basic GF model, and gives one additional parameter to estimate

\hypertarget{hiermod}{%
\section{Hierarchical Bayesian models}\label{hiermod}}

Spatial Models are characterized by many parameters which in turn are tuned by other hyper-parameters. Traditionally Bayesian hierarchical models are not widely adopted since they have high computational burdens, indeed they can handle very complex interactions via random components, especially when dealing with spatio temporal data \citet{Ling}. Blangiardo e Cameletti \citeyearpar{Blangiardo-Cameletti} tried to approach the problem from a different angle offering an intuitive solution on how hierarchy relates different levels parameters. This is done by reversing the problem and starting from data back to parameters, instead the other way round. So taking a few steps back the problem can be reformulated by starting from grouping observation into categories and then trying to impose a hierarchical structure on data based on the categories. As a result observations might fall into different categories, underlining their natural characteristics, such as: some of them might belong to category \emph{levels} like males or females, married or not-married. Moreover diving into the specific problem house prices can be faceted by which floor they belong or whether they are assigned to different energy classes and many others more. As an example Blangiardo and Cameletti example consider grouping data according to just a single 9 \emph{levels} category. Data for the reasons stated before can be organized such that each single observation (squares in figure below) belongs to its respective mutually exclusive and collectively exhaustive category (circles in figure).

\begin{figure}
\centering
\includegraphics{images/simple.PNG}
\caption{9 levels cat vs observaitions, source \citet{Blangiardo-Cameletti}}
\end{figure}

Furthermore data can be partitioned into two meta-categories, \emph{fist level} and \emph{second level}, highlighting the parameter and hyper paramter chain roles. \emph{First level} are identified by sampling observations which are drawn by the same probability distribution (squares) . \emph{Second level} (circles) are categories and might be associated to a set of parameters \(\theta=\left\{\theta_{1}, \ldots, \theta_{J}\right\}\).
Since the structure is hierarchical, a DAG (Directed Acyclical Graph) \citeyearpar{Blangiardo-Cameletti} representation might sort out ideas. If categories are represented by different \(\theta_{j}\) nodes and edges (arrows in the figure) are the logical belonging condition to the category then a single parameter \(\theta\) model has the right figure form:

\includegraphics{images/thetas.PNG} \includegraphics{images/chis.PNG}

To fully take into account the hierarchical structure of the data the model should also consider further lelvels. Since \(\left\{\theta_{1}, \ldots, \theta_{J}\right\}\) are assumed to come from the same distribution \(\pi(\theta_{j})\), then they are also assumed to be sharing information \citep{Blangiardo-Cameletti}, (left figure). When a further parameter \(\boldsymbol{\psi}=\left\{\psi_{1}, \ldots, \psi_{K}\right\}\) is introduced, for which a prior distribution is specified, then the conditional distribution of \(\boldsymbol{\theta}\) given \(\boldsymbol{\psi}\) is:

\[
\pi\left(\theta_{1}, \ldots, \theta_{J} \mid \boldsymbol{\psi}\right)=\int \prod_{j=1}^{J} \pi\left(\theta_{j} \mid \psi\right) \pi(\psi) \mathrm{d} \psi
\]
This is possible thanks to the conditional independence property already encountered in chapter \ref{inla}, which means that each single \(\theta\) is conditional independent given \(\psi\)
This structure can extended to allow more than two levels of hierarchy since the marginal prior distributions of \(\theta\) can be decomposed into the product of their conditional priors distributions given some hyper parameter \(\psi\) as well as their prior distribution \(\pi(\psi)\).

\[
\pi(\boldsymbol{\theta})=\int \pi\left(\boldsymbol{\theta} \mid \boldsymbol{\psi}_{1}\right) \pi\left(\boldsymbol{\psi}_{1} \mid \boldsymbol{\psi}_{2}\right) \ldots \pi\left(\boldsymbol{\psi}_{L-1} \mid \boldsymbol{\psi}_{L}\right) \pi\left(\boldsymbol{\psi}_{L}\right) \mathrm{d} \boldsymbol{\psi}_{1} \ldots \mathrm{d} \boldsymbol{\psi}_{L}
\]

\(\boldsymbol{\psi}_{l}\) identifies the hyper pram for the \(l_{th}\) level of hierarchy. Each further parameter level \(\psi\) is conditioned to its previous in hierarchy level \(l-1\) so that the parameter hierarchy chain is respected and all the linear combinations of parameters are carefully evaluated. The \emph{Exchangeability} property enables to have higher \(H\) nested DAG (i.e.~add further \(L\) levels) and to extend the dimensions in which the problem is evaluated, considering also time together with space. From a theoretical point of view there are no constraints to how many \(L\) levels can be included in the model, but as a drawback the more the model is nested the more it suffers in terms of interpretability and computational power. Empirical studies have suggest that three levels are the desired amount since they offer a good bias vs variance trade-off.

\hypertarget{finalregr}{%
\section{INLA model through spatial hierarchical regression}\label{finalregr}}

INLA model seen in section \ref{LGM} can be rearranged according to the hierarchical structure considering also the regression settings for point referenced data stated in the previous section \ref{univariateregr}.

As an initial step it is required to specify a probability distribution for \(\boldsymbol{y} = \left(y\left(s_{1}\right), \ldots, y\left(s_{n}\right)\right)=\left(y_{1}, \ldots, y_{n}\right)\), this is a mandatory step looking at the \ref{example} methods needed to compute the \texttt{inla()} function. A Normal distribution for simplicity is chosen.

As \emph{first level} is picked up an \textbf{exponential family} sampling distribution (i.e.~Normally distributed, Gamma one other choice), which is \emph{exchangeable} with respect to the \(\boldsymbol{\theta}=\left\{\beta_{0}, \boldsymbol{\beta}, f\right\}\) \emph{latent field} and hyper parameters \(\boldsymbol{\psi_{1}}\), which includes also the ones coming from the latent Matérn GP process \(w_{i}\). The Spatial Guassian Process is centered in 0 and with Matérn covariance function as \(\tau^2\). \(w_{i}\) addresses the spatial autocorrelation between observation through a Matérn covariance function \(\mathcal{C}(\cdot | \boldsymbol\psi_{1})\) which in turn is tuned by hyper param included in \(\boldsymbol{\psi_1}\). Moreover the \(w_{i}\) surface has to be passed in the formula method definition \ref{example} via the \texttt{f()} function, so that INLA takes into cosideration the spatial component.

\[
\boldsymbol{y} \mid \boldsymbol{\theta}, \boldsymbol{\psi}_{1} \sim \mathrm{N}\left(\beta_{0}+ (\mathbf{X}_{i})^{\prime}\boldsymbol{\beta} + w_{i} ,  \tau^2 I_{n}\right)=\prod_{i=1}^{n} \mathrm{N}\left(y_{i} \mid \theta_{i}, \psi_{1}\right)
\]

Then at the \emph{second level} the latent field \(\boldsymbol{\theta}\) is characterized by a Normal distribution given the remaining hyper parameters \(\boldsymbol{\psi}_2\), recall the covariance matrix \(\boldsymbol{Q}^{-1}(\boldsymbol{\psi_{2}})\), depending on \(\boldsymbol{\psi_{2}}\) hyperparameters, is handled now by a Matérn covariace function depeding on its hyperparamter. This is done in order to map the GP spatial surface into a GMRF by SPDE solutions.

\[
\boldsymbol{\theta} \mid \boldsymbol{\psi}_{2} \sim \mathrm{N}\left(\boldsymbol{0}, \mathcal{C}( \cdot , \cdot  \mid \boldsymbol{\psi}_{2})\right)
\]

In the end a \emph{third level} hyper parameters \(\boldsymbol{\psi}=\left\{\boldsymbol{\psi_{1}}, \boldsymbol{\psi}_{2}\right\}\) having some specified prior distribution i.e.~\(\boldsymbol{\psi} \sim \pi(\boldsymbol{\psi})\),

\hypertarget{spatial-kriging}{%
\section{Spatial Kriging}\label{spatial-kriging}}

In Geostatistics the main interest resides in the spatial prediction of the spatial latent field pr the response variable at location not yet observed.
Assumed the model in the previous section, suppose that \(y^{\star}\) is not a observed occurrence of the response variable at location \(s_{0}\) (not in the data) of the GP \(w_{i}\) spatial surface estimated through observed refereced points in \(\boldsymbol{y}\). As a consequence of exchangeability (first step previous section \ref{finalregr}) then \(\boldsymbol{y}^{\otimes}=\left\{\boldsymbol{y}, y^{\star}\right\}\). Then considering INLA notation it is obtained:

\[
\begin{aligned}
&\pi\left(y^{\star} \mid \boldsymbol{y}\right)=\frac{\pi\left(\boldsymbol{y}, y^{\star}\right)}{\pi(\boldsymbol{y})} \text { from the conditional probability }\\
&=\frac{\int \pi\left(y^{\star} \mid \theta\right) \pi(\boldsymbol{y} \mid \theta) \pi(\theta) \mathrm{d} \theta}{\pi(\boldsymbol{y})} \text { by exchangeability }\\
&=\frac{\int \pi\left(y^{\star} \mid \theta\right) \pi(\theta \mid y) \pi(y) \mathrm{d} \theta}{\pi(y)} \text { applying Bayes' theorem }\\
&=\int \pi\left(y^{\star} \mid \boldsymbol{\theta}\right) \pi(\boldsymbol{\theta} \mid \boldsymbol{y}) \mathrm{d} \boldsymbol{\theta}
\end{aligned}
\]

A DAG representation might offr the intuition behind Prediction in spatial models:

\begin{figure}
\centering
\includegraphics{images/spatial_prediction.jpg}
\caption{Spatial prediction representation through DAG, source \citet{Blangiardo-Cameletti}}
\end{figure}

where \(\pi\left(y^{\star} \mid \boldsymbol{y}\right)\) is said predictive distribution and it is meaningful only in the Bayesian framework since the posterior distribution is treated as a random variable, which is totally not true in frequentist statistics.

\hypertarget{model-checking}{%
\section{Model Checking}\label{model-checking}}

(Incrociarlo con altri tesi)

Once the model is set up and fitted a resampling scheme has to be chosen in order to evaluate the model performance. One of the most used method to assess beyasian model quality is LOOCV cross validation and defualt choice fo R-INLA package. From data is left out one single observation and so that the Validation set is \(\boldsymbol{y}_{v} = \boldsymbol{y}_{-i}\) and the Assessement set is a \(\boldsymbol{y}_{a} = \boldsymbol{y}_{i}\)
the rest of the observations. Two KPI are assumed to be representative:

\begin{itemize}
\tightlist
\item
  CPO conditional predictive ordinate (pettit, 1990): \(CPO_{i} = \pi(y^{\star} \mid \boldsymbol{y}_{v})\)
\item
  PIT probability integral tranform (dawid, 1984): \(PIT_{i} = \pi(y^{\star} < y_{i} \mid \boldsymbol{y}_{v})\)
\end{itemize}

These quantities are used by default by setting control options in the \texttt{inla(control.compute\ =\ list())} list object by setting them equal o TRUE. Inla also provides an inner method to authomatically handlee failing in computing those two quantities, leadind to values of 1 when predictions are not reliable and the ipposite for 0.Moreover the empirical distribution of the PIT can be used to asses predictive performance: if it is Uniform, so there are not values that strongly differ from the others then the model is correctly checked. Otherwise if the dostribtuon almost approxiamtes any of the other possibles then the Cross validation assessement prediction has led incorrectly predict the ``out of the bag'' validation sample.

Posteerior checking method exploits a full cross validation where \(\boldsymbol{y}_{a} = \boldsymbol{y}_{v}\) and it is called predictive checks. Th assessement set now is equal to the validation set,a s a consequence all the observation are evaluated twice. 4 quantities are driver to model estimate quality:

\begin{itemize}
\tightlist
\item
  the \emph{posterior predictive distribution}: \(\pi(y^{\star} \mid \boldsymbol{y}) = \int \pi(y^{\star} \mid \theta_{i})\pi({\theta_{i}} \mid \boldsymbol{y})\mathrm{d}\theta_{i}\) which is the likelihood of a replicate observation. When values are small that indicates that are those values are coming from tails, since the area under the curve (i.e.~probability) is less. If this happens for many observation then outliers are driving the model leading to poor estimates
\item
  the \emph{posterior predictive p-value} whose math expression is:\(\pi(y^{\star} \leq y_{i} \mid \boldsymbol{y})\) for which values near to 0 and 1 indicates poor perfomances.
\item
  \emph{Root Mean Square Predictive Error RMSE}: \(\sqrt{\frac{1}{n} \sum_{i=1}^{n}(y_{i}-{y}^{\star}_{i})^{2}}\)
\item
  \(R^2\)
\end{itemize}

R-INLA has already antiticipated in chapter 4 section\ref{example} have designed function to compute statistics on posterior distribution as \texttt{inla.pmarginal()} returning the cumulative density distribution.

\hypertarget{prior-specification}{%
\section{Prior Specification}\label{prior-specification}}

\hypertarget{spde}{%
\chapter{SPDE approach}\label{spde}}

Observations in the spatial problem setting are considered as realizations of a stationary, isotropic unobserved GP \(w(s)\) that we aim to estimate (\ref{GP}). Before approaching the problem with SPDE, GPs were treated as multivariate Gaussian densities and Cholesky factorizations were applied on the covariance matrices and then fitted with likelihood. Matrices in this settings were very dense and they were scaling with the order of \(O\left(n^{3}\right)\), leading to obvious big-n problem.
The breakthrough, came with \citet{Lindgren2011} that proves that a stationary, isotropic GP with Matérn covariance can be represented as a GMRF using SPDE solutions by finite element method \citep{Krainski-Rubio}. In other words given a GP whose covariance matrix is \(\boldsymbol{Q}\), SPDE can provide a method to approximate \(\boldsymbol{Q}\) without computational constraints. As a matter of fact SPDE are equations whose solutions are GPs with a chosen covariance function focused on satisfying the relationship SPDE specifies.
Benefits are many but the most important is that the representation of the GP through a GMRF provides a sparse representation of the spatial effect through a sparse precision matrix \(\boldsymbol{Q}^-1\) . Sparse matrices enable convenient inner computation properties of GMRF that can be exploited with INLA. Bayesian inference on GMRF can take advantage of lower computational cost because of these properties stated before leading to a more feasible big-O \(O\left(n^{3 / 2}\right)\). The following chapter will provide a intuition on SPDE oriented to practitioners. The chapeter once again will follow the track of Krainski \& Rubio \citeyearpar{Krainski-Rubio} and Blangiardo and Cameletti \citeyearpar{Blangiardo-Cameletti} works, together with the street-opener paper from \citet{Miller2019} as compendium. SPDE might be complex for those who are not used to applied mathematics and physics making it difficult not only to grab the concept, but also to find its applications. One more obstacle regards SPDE software implementation, since without deep technical expertise it might be difficult to customize code with the aim to extend the methodology to different models. For a gentle introduction on what a SPDE is from a mathemathical perspective a valuable reference is \citet{Miller2019} in secction 2.1, then also its application to Matérn in 2.3.

\hypertarget{set-spde-problem}{%
\section{Set SPDE Problem}\label{set-spde-problem}}

Given the statistical model already encountered in chapter \ref{univariateregr}:

\[
y\left(\mathbf{s}_{i}\right)=\mathbf{x}\left(\mathbf{s}_{i}\right)^{\prime} \beta_{j}+w(\mathbf{s})+\varepsilon\left(\mathbf{s}_{i}\right)
\]

where \(\eta(\mathbf{s}_{i}) = g (\mathbf{x}\left(\mathbf{s}_{i}\right)^{\prime}\beta_{j})\) is the linear predictor, whose link function \(g( \cdot )\) is identity (can be also extended to GLM), where \(w(\mathbf{s})\) is a Gaussian Process with mean structure 0 and \(C(\cdot)\) covariance structure ( where \(\boldsymbol{Q}\) is the covariance matrix and \(Q^{-1}\) precision matrix). Then \(w(s) \sim MV\mathcal{N}(0, \boldsymbol{Q}_{i,j}^{-1})\) and where \(\varepsilon(\mathbf{s}_{i})\) is white noise error such that \(\varepsilon(\mathbf{s}_{i}) \sim \mathcal{N}\left(0, \tau^{2}\right)\).
Comprehending \(w\) in the model brings two major issues, specify a covariance function for observations as well as how to fit the model. Among all the possible reachable solutions including the SPDE, the common goal is to define covariance function between locations by approximating the precision matrix \(\boldsymbol{Q}^{-1}\), since they are an effective tool to represent covariance function as in section \ref{LGM}.
For those reasons SPDE approach implies finding an SPDE whose solution have the precision matrix, that is desired for \(w\). \citet{Lindgren2011} proves that an approximate solution to SPDE equations is to represent \(w\) as a sum of basis function multiplied by coefficients \citeyearpar{Miller2019}. Moreover the basis function coefficients are in reality a GMRF (for which fast method comptations already exists).

\hypertarget{spde-within-r-inla}{%
\section{SPDE within R-INLA}\label{spde-within-r-inla}}

First point addresses the assumption that a GP with Matérn covariance function and \(\nu >0\) is a solution to \emph{SPDE} equations.
Second point addressed the issues of solving SPDE when grids are irregular, as opposite with the one seen in first point (regular grid for irregular distribution. In here comes FEM used in mathematics and engineering application with the purpose to solve differential equations. Notation is kept coherent with the one for the previous chapter.

\hypertarget{first-point-krainsky-rubio-too-technical}{%
\section{First Point Krainsky Rubio TOO TECHNICAL}\label{first-point-krainsky-rubio-too-technical}}

A regular 2D grid lattice is considered with infinite number of location points as vertices.

\begin{figure}
\centering
\includegraphics{images/lattice.png}
\caption{lattice 2D regular grid}
\end{figure}

\hypertarget{exploratory}{%
\chapter{Exploratory Analysis}\label{exploratory}}

Data flows out the REST API end point \texttt{*/complete} in a .json format. Data can be filtered out On the basis of the options set in the API endpoint argument body. Some of the options supplied to the API, as in section \ref{APIdocs}, might regard the real estate \texttt{city} interested, \texttt{npages} as the number of pages to scrape, \texttt{type} as the choice between rental of selling market. Since to the analysis extent data should come from the same geographic area API, city and filter parameters are kept permanent (e.g.~Milan rental real estate within ``circonvallazione'' approximated geo-borders). As a consequence a dedicated endpoint \texttt{.thesis} parameter is passed in the argument body. By setting the option equal to to TRUE the API caller requests thesis data. In other words the latter option under the hood secures to specify to the API an already composed query url to be passed to the scraping endppoint, which corresponds to precise zones imposed while searching for advertisemnts on Immobiliare.it. To help figure out the idea behind the operation it can be thought as refreshing everyday the same immobiliare.it url on their website looking for accomodations within a specified zone.
Parameters specified are also npages = 120, leading to to 3000 observations. The \texttt{*} refers to the EC2 public DNS.

\texttt{http://*/complete/120/milano/affitto/.thesis=true}

As a further data source is available a mondgoDB ATLAS cluster which, because of the scheduler, stocks daily .csv information from Milan real estate. Credentials have to be supplied. For run time reasons also related to the bookdown files continuous building the API endpoint is not called and code chunks outputs are cached due to heavy computation. Instead data is extracted from the MongoDB cluster. A summary table of the columns involved is presented with the goal to introduce the reader to API incoming data. Data needs some heavy preprocess steps to get modeled which is briefly covered in the data preparation part in section \ref{prep}. Data coming from the /complete enndpoint has a geo-statistical spatial component and consequently a spatial representation of the dataset is showed. One further plot points out that geographic coordinates are non-linearly related \ref{fig:NonLinearSpatialRel} to the price response variable so dedicated techniques are required. Exploration starts with factor counts evidencing a ``Bilocale'' prevalence which is then compared to other cities. This suggest some critical Milan real estate market demand information and consequently reflections on the offer. Heating and cooling systems, two of the covariates extracted, are grouped and then arranged by descending order prevalence. They both do not display any significative price change but they bring to the surface an important environmental concern. The same is done by highlighting ridges distribution for other two newly engineered covariates. Data displays bimodality in prices distribution for different n-roomed accommodations and the model should take account of the behavior. Then a piece-wise linear regression is fitted for each n-roomed accommodation sub-group whose single predictor is the square meter footage. The analysis emphasize some valuable economic consequences both for investors interested into property expansions and for tenants that are planning to partition single properties into rentable sub-units. The previous analysis brings along a major question which regards the most valuable properties per single square meter surface and a answer based on data is given. Then a further log linear regression setting is proposed to evaluate the impact of some other presumably important covariate. A Tie Fighter plot displays for which coefficient, associated to each dummyied predictor, there are surprisingly high prices compared to the effect of the square meter footage expansion. A partial conclusion is that having 2 or 3 bathrooms truly pays an extra monthly gain, also caused by the number of tentants the accomodations could host. Then missing assessement and imputation takes place. At first is made a brief a revision of missing and randomnes by \citet{Little}, then theory is applied by visualizing missing in combination with heat-map and co-occurrence plot. Combined missing observation test is able to detect whether data is missing because of inner scraping faiilures or simple rarity in data appereance. Then for the observations that passed the test imputation is made through INLA posterior expectation. That is the case of data lost in predictors so the missing covariates ( \emph{condominium} ) are brought into a model as response variable whose this time predictors are explanatory ones. Through a method specified within the INLA function the posterior statistics are computed and then finally imputed in the place of missing ones.

Visualisations are done with ggplot2 in a Tidyverse approach. Maps are done with ggplot2 too and Leaflet, together with its extensions.
A preliminary API data exploratory analysis evidences 34 covariates and 250 rows, which are once again conditioned to the query sent to the API. Immobiliare.it furnishes many information regarding property attributes and estate agency circumstances. Data displays many NA in some of the columns but georeference coordinates, due to the design of scraping functions, are in any case present.

\begin{longtable}{ll}
\toprule
name & ref\\
\midrule
ID & ID of the apartements\\
LAT & latitude coordinate\\
LONG & longitude coordinate\\
LOCATION & the complete address: street name and number\\
CONDOM & the condominium monthly expenses\\
\addlinespace
BUILDAGE & the age in which the building was contructed\\
FLOOR & the property floor\\
INDIVSAPT & indipendent property type versus apartement type\\
LOCALI & specification of the type and number of rooms\\
TPPROP & property type residential or not\\
\addlinespace
STATUS & the actual status of the house, ristrutturato, nuovo, abitabile\\
HEATING & the heating system Cen\_Rad\_Gas (centralizzato a radiatori, alim a gas), Cen\_Rad\_Met,\\
AC & air conditioning hot and cold, Autonomo, freddo/caldo, Centralizzato, freddo/caldo\\
PUB\_DATE & the date of publication of the advertisement\\
CATASTINFO & land registry information\\
\addlinespace
APTCHAR & apartement main characteristics\\
PHOTOSNUM & number of photos displayed in the advertisement\\
AGE & real estate agency name\\
LOWRDPRICE\_ORIGINAL\_PRICE & If the price is lowered it flags the starting price\\
LOWRDPRICE\_CURRENT\_PRICE & If the price is lowered it flags the current price\\
\addlinespace
LOWRDPRICE\_PASSED\_DAYS & If the price is lowered indicates the days passed since the price has changed\\
LOWRDPRICE\_DATE & If the price is lowered indicates the date the price has changed\\
ENCLASS & the energy class according to the land registers\\
CONTR & the type of contract\\
DISP & if it is still avaiable or already rented\\
\addlinespace
TOTPIANI & the total number of the building floors\\
PAUTO & number of parking box or garages avaibable in the property\\
REVIEW & estate agency review, long chr string\\
HASMULTI & it if has multimedia option, such as 3D house vitualization home experience or videos\\
PRICE & the monthly price <- response\\
\addlinespace
SQFEET & square meters footage\\
NROOM & the number of rooms in the house, and their types\\
TITLE & title of published advertisement\\
\bottomrule
\end{longtable}

\hypertarget{prep}{%
\section{Data preparation}\label{prep}}

Data needs to undergo to many previous cleaning preprocess steps, this is a forced stage since API data comes in human readable format, which is not prepared to be modeled. Cleaning steps mainly regards:

\begin{itemize}
\tightlist
\item
  encoding from UTF-8 to Latin due to Italian characters incorrectly parsed.
\item
  \emph{floors} covariate needs to be separated by its \emph{ascensore} and \emph{accdisabili} components, adding 2 more bivariate covariates.
\item
  \emph{locali} needs to be separated too. 5 category levels drain out: \emph{totlocali}, \emph{camereletto}, \emph{altro}, \emph{bagno}, \emph{cucina}. \emph{nroom} is a duplicate for \emph{totlocali}, so it is discarded.
\item
  \emph{aptchar} is a character strign column that contains a various number of different features per house. The preprocess steps include cleaning the string from unnecessary characters, then finding the whole set of unique elements across the character column by splitting on a regex pattern, in the end recoding newly created bivariate columns ``yes'' or ``no'' accoeding to a matching pattern whether the feature appears in the string not. A slice from the API output APTCHAR is:
\end{itemize}

fibra ottica videocitofono impianto di allarme porta blindata reception balcone portiere intera giornata impianto tv centralizzato parzialmente arredato esposizione doppia

\hypertarget{maps-and-geo-visualisations}{%
\subsection{Maps and Geo-Visualisations}\label{maps-and-geo-visualisations}}

Geographic coordinates can be represented on a map in order to reveal first symptoms of spatial autocorrelation. Observations are spread almost equally throughout the surface even though the response var \emph{price} indicates unsurprisingly that higher prices are nearer to the city center.
The map in figure @ref(fig:leaflet\_visuals) is a leaflet object, which needs to be overlapped with layers indicating different maps projections. This is interactive in the .html version, and static is proposed in the .pdf output version. The map object takes a input the latitute and longitude coordinates coming from THE API, and they do not need any CRS (Coordinate Reference System) projection since leaflet can accept the data type.

\begin{figure}
\includegraphics[width=1\linewidth]{images/leaflet_prezzi} \caption{Leaflet Map}\label{fig:LeafletVisuals}
\end{figure}

Predictors, in this case latitude and longitude appear to have nonlinear relationships with the outcome price. The relationship appears to be Gaussian whose mean points to the city center, red dashed line represent latitude and longitude coordinates for the Dome of Milan. Non linearities can be treated with regression splines

\begin{figure}
\centering
\includegraphics{07-exploratory_files/figure-latex/NonLinearSpatialRel-1.pdf}
\caption{\label{fig:NonLinearSpatialRel}Non Linear Spatial Relationship disclosed}
\end{figure}

ggplot2 visualzitaion matt dancho inspiration::

\hypertarget{counts-and-first-orientations}{%
\section{Counts and First Orientations}\label{counts-and-first-orientations}}

Arranged Counts for categorical columns can give a sense of the distribution of categories across the dataset suggesting also which predictor to include in the model. The visualization in figure \ref{fig:fctCounts} offers the rearranged factor \emph{TOTLOCALI}.
Bilocali are the most common option for rent, then trilocali comes after. The intuition behind suggests that Milan rental market is oriented to ``lighter'' accommodations in terms of space and squarefootage. This should comes natural since Milan is both a vivid study and working area, so short stayings are warmly welcomed.

\begin{figure}
\centering
\includegraphics{07-exploratory_files/figure-latex/fctCounts-1.pdf}
\caption{\label{fig:fctCounts}Most common housedolds categories}
\end{figure}

Two of the most requested features for comfort and livability in rents are the heating/cooling systems installed. Moreover rental market demand, regardless of the rent duration, strives for a ready-to-accomodate offer to meet clients needs. In this sense accomodation coming with the newest and most techonological systems are naturally preferred with respect the contrary.
x-axis in figure \ref{fig:PricePerAc} represents log\_10 price for both of the two plots. Logarithmic scale is needed to smooth distributions and the resulting price interpretation have to considered into relative percent changes. Furthermore factors are reordered with respect to decreasing price.\\
y-axis are the different level for the categorical variables recoded from the original data due to simplify lables and to hold plot dimension. Moreover counts per level are expressed between brackets close to their respective factor.
The top plot displays the most prevalent heating systems categories, among which the most prevalent is ``Cen\_Rad\_Met'' by far. This fact is extremely important since metano is a green energy source and if the adoption is wide spread and pipelines are well organized than it brings enormous benefit to the city. As a consequence one major concern regards that for many years policies have been oriented to reduce vehicles emission (euro1 euro2\ldots) instead of focusing on house emissions. This was also a consequence of the lack of house data especially in rural areas. According to data there are still a 15\% portion of houses powered by oil fired.
Then in bottom plot Jittering is then applied to point out the number of outliers outside the IQR (Inter Quantile Range) .25 and their impact on the distribution. A first conclusion is that outliers are mainly located in autonomous systems, which leads of course to believe that the most expensive houses are heated by autonomoius heating systems. Indedd in any case this fact that does not affect monthly price. The overlapping IQR signifies that the covariates levels do not impact the response variable.

\begin{figure}
\centering
\includegraphics{07-exploratory_files/figure-latex/PricePerAc-1.pdf}
\caption{\label{fig:PricePerAc}Log Monthly Price per Heating/Cooling system?}
\end{figure}

this visualization intersects allows to discover bimodality in the response variable. Log scales was needed since they are all veru skewd and log scale then is needed also in the model.

(qui ci puoi mettere a confronto per variabile bianria, così vedi cosa includere nel modello esempio sotto dove commentato, )

\includegraphics{07-exploratory_files/figure-latex/unnamed-chunk-1-1.pdf}

What it might be really relevant to research is how monthly prices change with respect to house square footage for each house configuration. The idea is to asses how much adding a further square meter affetcs the monthly price for each n-roomed flat.
One implication is how the property should be developed in order to request a greater amount of money per month. As an example in a situation in which the household has to lot its property into different sub units he can be helped to decide the most proficient choice in economic terms by setting ex ante the square footage extensions for each of the sub-properties.
A further implication can regard economic convenience to enlarge new property acquisitions under the expectation to broadened the square footage (construction firms). Some of the potential enlargements are economically justified, some of the other are not.
The plot \ref{fig:GlmPriceSq} has two continuous variables for x (price) and y (sqfeet) axis, the latter is log 10 scaled due to smoothness reasons. Coloration discretizes points for the each \(j\) household rooms totlocali. A sort of overlapping piece-wise linear regression (log-linear due to transformation) is fitted on each totlocali group, whose response variable is price and whose only predictor is the square footage surface (i.e.~\(\log_{10}(\mathbf{price_j}) \sim +\beta_{0,j}+\beta_{1,j}\mathbf{sqfeet_j}\)). Five different regression models are proposed in the top left. The interesting part regards the models slopes \(\hat\beta_{1,j}\). The highest corresponds to ``Monolocale'' for which the enlargement of a 10 square meters in surface enriches the apartment of a 0.1819524\% monthly price addition. Almost the same is witnessed in ``Bilocale'' for which a 10 square meters extension gains a 0.1194379\% value. One more major thing to notice is the ``ensamble'' regression line obtained as the interpolation of the 5 plotted ones. The line suggests a clear slope descending pattern (logarithmic trend) from Pentalocale and beyond whose assumption is strengthened by looking at the decreasing trend in the \(\hat\beta_1\) predictor slopes coefficients. Furthermore investing into an extension for ``Quadrilocale'' and ``Trilocale'' is \emph{coeteris paribus} an interchangeable economic choice.

\begin{figure}
\centering
\includegraphics{07-exploratory_files/figure-latex/GlmPriceSq-1.pdf}
\caption{\label{fig:GlmPriceSq}Monthly Prices change wrt square meters footage in different n-roomed apt}
\end{figure}

In table (\ldots) resides the answer to the question ``which are the most profitable properties per month in terms of the price per square meter footage ratio''. The covariate floor together with the totpiani are not part of the model, indeed they can explain the importance and the height of the building justifying extraordinary prices. The first 4 observations are unsurprisingly ``Bilocale'', the spatial column location, not a regressor, can lend a hand to acknowledge that the street addresses point to very central and popular zones. The zones are, first City Life, second Brera and third Moscova, proving that in modeling real estate rents the spatial component is fundamental , even more in Milan.

\begin{tabular}{llrrllr}
\toprule
location & totlocali & price & sqfeet & floor & totpiani & abs\_price\\
\midrule
viale cassiodoro 28 & Bilocale & 1750 & 30 & 9 piano & 10 piani & 58.33333\\
via della spiga 23 & Bilocale & 2750 & 55 & 2 piano & 4 piani & 50.00000\\
corso giuseppe garibaldi 95 & Bilocale & 2700 & 56 & 2 piano & 5 piani & 48.21429\\
piazza san babila C.A. & Bilocale & 1833 & 42 & 4 piano & 4 piani & 43.64286\\
ottimo stato piano terra, C.A. & Trilocale & 3000 & 80 & Piano terra & 3 piani & 37.50000\\
\addlinespace
via federico confalonieri 5 & Monolocale & 750 & 20 & 1 piano & 3 piani & 37.50000\\
\bottomrule
\end{tabular}

Then as a further point it might be important to investigate a linear model whose response is price and whose covariates are the newly created abs\_price and some other presumably important ones e.g.~floor, bagno, totpiani. The model fitted is \texttt{log2(price)\ \textasciitilde{}\ log2(abs\_price)\ +\ bagno\ +\ floor\ +\ totpiani}.
The plot in figure \ref{fig:TieFighterPlot} has the purpose to demonstrate how monthly price is affected by covariates conditioned to their respective square meter footage. The interpretation of the plot starts by fixing a focus point on 0, which is the null effect highlighted by the red dashed line. Then the second focus is on house surface effect (i.e.~House Surface (doubling) in the plot, the term log2(abs\_price) has been converted to more familiar House Surface (doubling)), which contributes to increase the price of an estimated coefficient of \(\approx .6\) for each doubling of the square meter footage. Then what it can be noticed with respect to the two focus points are the unusual effects provoked by the other predictors t the right of the house surface effect and to the far left below 0. ``2 and 3 bagni'' are unusually expensive with respect to the square meter footage increment, on the other hand" al piano rialzato" and ``al piano terra'' are undervalued with respect to their surface. The fact that 2 and 3 bathrooms can guarantee a monthly extra check is probably caused to a minimum rent plateau requested for each occupant. the number of bathrooms are a proxy to both house extension since normally for each sleeping room there also exist at least 1 bathroom as well as prestigious houses dispose of more than 1 toilette services. So the more are the occupants regardless of the square meter footage dedicated to them, the more the house monthly returns,

it can be noticed is that ultimo piano, otgether with 2 abagni ad 3 bagni are unusually expensive with respect to their proper square meter footage. On the other hand the piano rialzato and piano terra are unusually undervalued given their surface.\\
In other words the to help with the interpretation. The fact that 2 and 3 bathrooms can guarantee a monthly extra check is probably caused to a minimum rent plateau requested for each occupant. So the more are the occupants regardless of the square meter footage dedicated to them, the more the house returns. The conclusion

\begin{figure}
\centering
\includegraphics{07-exploratory_files/figure-latex/TieFighterPlot-1.pdf}
\caption{\label{fig:TieFighterPlot}Coefficient Tie fighter plot for the linear model: log2(price) \textasciitilde{} log2(abs\_price) + condom + other\_colors}
\end{figure}

\hypertarget{text-mining-in-estate-review}{%
\section{Text Mining in estate Review}\label{text-mining-in-estate-review}}

The word network in figure \ref(fig:WordNetworkgraph) tries to summarize relevant information from real estate agency review into each advertisement. avg\_totprice expresses the sum of the price per month plus the condominium in order to fully integrate inner property characteristics together with building exclusivity. Tokenized words are then filtered with ``stopwords-iso'' italian dictionary.
Nodes associated with hotter colours are also associated to more expensive in and out-house characteristics. The size of nodes keeps track of the number of reviews in which the specific word appears. A table of the most common words can help highlight both the real estate jargon as well as words that brings up house values.

\begin{tabular}{lrrr}
\toprule
word & count & reviews & avg\_totprice\\
\midrule
bagno & 249 & 192 & 1888.622\\
cucina & 247 & 190 & 2088.814\\
ingresso & 194 & 173 & 1964.062\\
soggiorno & 182 & 159 & 1872.500\\
camera & 200 & 158 & 1936.945\\
\addlinespace
piano & 197 & 157 & 1982.234\\
arredato & 184 & 152 & 1744.614\\
composto & 158 & 146 & 1758.911\\
riscaldamento & 171 & 144 & 1877.404\\
zona & 282 & 139 & 1930.213\\
\bottomrule
\end{tabular}

Furthermore it is possible to grossly divide the plot in figure \ref(fig:WordNetworkgraph) into 3 sub-groups of nodes, each of which addresses a specific part of the house comprehensive evaluation.
In the far right side of the plot are considered the external appliances like neighbor stores, subway stations and services and are associated to mean prices. The correspondent number of reviews are not justifying by any type of price increasing effect. Whereas slightly moving the view to the left, the area centered in portineria evidences a sub-groups of nodes associated to relatively higher avg-totprice. Some of them are servizio signorile palazzo. The previous set of nodes indicates services that are proper to the building can lead to some sort of extra payment. Then still moving
Possiamo immaginare di dividere il network in 3 raggruppamenti di nodi, ognuno dei quali parla di uni specifico tema. nella parte alta sinistra csi parla delle circostanze estenre dell'appartamente, i negozi i mezzi serizi la metri, i prezzi evidenzziati dal colore nei nodi sono neutri, indicando che non impattano il prezzo in maniera significativa. poco più sotto è possibile vedere un altro centroide verso il quale puntano una serie di edges peritenti che riguardano i servizi interni al building come la portinerua, l'ingresso, il palazzo. in questo caso i colori sono più caldi e i servizi sembrano essere pagati di più. successivamente sosptandoci veros il centro del'network si nota un nodo di gravità attorno alquale si trovamo molti outgoing edges, che riscaldamento. Attorno a riscaldamento che vista la grandezza ricorre spesso nelle recensioni, si sviluppano tutti i servizi non descritti da immobiliare all'interno della casa, insiema a tutte le caratteristiche cbe distinguono la casa revisionata dalle altre. i colori degradano spostandosi da sinistra verso destea, accanto a riscaldmaento si nota un cluster che associati a prezzi minoro come spese condominiali e arredato arredato. nel caso delle spese condominiali i cluster sono associati a prezzi minori perchè il prezzo del conominio spesso non è commisurato al prezzo nè al prestigio dell'appartamento. Speso infatti include costi variabili come utenze gas e luce, o acqua che vengono scontati con prezzi più bassi di affitto. la somma di condominio e prezzo offrirebbe un panorama più chiaro.

\begin{figure}
\centering
\includegraphics{07-exploratory_files/figure-latex/WordNetworkgraph-1.pdf}
\caption{\label{fig:WordNetworkgraph}Word Network Graph for 250 Estate Agencies Review}
\end{figure}

\hypertarget{missing-assessement-and-imputation}{%
\section{Missing Assessement and Imputation}\label{missing-assessement-and-imputation}}

As already pointed out some data might be lost since immobiliare provides the information that in turn are pre filled by estate agencies or privates through standard document formats. Some of the missing can be reverse engineered by other information in the web pages e.g.~given the street address it is possible to trace back the lat and long coordinates. Some other information can be encountered in .json files hidden inside each of the single web pages.
The approach followed in this part is to prune redundant data and rare predictors trying to limit the dimensionality of the dataset.

\hypertarget{missing-assessement}{%
\subsection{Missing assessement}\label{missing-assessement}}

The first problem to assess is why information are missing. As already pointed out in the preliminary part as well as in section \ref{ContentArchitecture} many of the presumably important covariates (i.e.~price lat, long, title ,id \ldots) undergo to a sequence of forced step inside scraping functions with the aim to avoid to be lost. If at the end of the sequence covariates are still missing, the correspondent observation is not considered and it is left out of the resulting scraped dataset. The choice originates from empirical missing patterns suggesting that when important information are missing then the rest of the covariates are more likely to be missing to, as a consequence the observation should be discarded.
The missing profile is crucial since it can also raise suspicion on the scraping failures. By Taking advantage of the missing pattern in observations the maintainer can directly identify the problem and derivatives and immediately debug the error. In order to identify if the nature of the pattern a revision of missing and randomness is introduced by \citet{Little}.
Missing can be devided into 3 categories:

\begin{itemize}
\tightlist
\item
  \emph{MCAR} (missing completely at random) likelihood of missing is equal for all the information, in other words missing data are one idependetn for the other.
\item
  \emph{MAR} (missing at random) likelihood of missing is not equal.
\item
  \emph{NMAR} (not missing at random) data that is missing due to a specific cause, scarping can be the cause.
\end{itemize}

MNAR is often the case of daily monitoring clinical studies \citep{Kuhn}, where patient might drop out the experiment because of death and so all the relating data starting from the death time +1 are lost.
To identify the pattern a \emph{heat map} plot \ref{fig:Heatmap} clarifies the idea:

\begin{figure}
\centering
\includegraphics{07-exploratory_files/figure-latex/Heatmap-1.pdf}
\caption{\label{fig:Heatmap}Missingness Heatmap plot}
\end{figure}

Looking at the top of the heat map plot, right under the ``Predictor'' label, the first tree split divides predictors into two sub-groups. The left branch considers from \emph{TOTPIANI} to \emph{CATASTINFO} and there are no evident patterns. Then missingness can be traced back to MAR. Imputation needs to be applied up to \emph{CONDOM} included, the others are discarded due to rarity: i.e.~\emph{BUILDAGE}: 14\% missing, \emph{CATASTINFO}: 21\% and \emph{AC}: 24\%. Moreover \emph{CUCINA} and \emph{ALTRO} are generated as ``childred'' of the original \emph{LOCALI} variable, so it should not surprise that their missing behavior is similar ,whose prevalence is respectively 13\% and 14\%, for that reason are discarded.
In the far right hand side \emph{ENCLASS} and \emph{DISP} data are completely missing and a pattern seems to be found. The most obvious reason is a scraping fail in capturing data. Further inspection of the API scraping functions focused on the two covariates is strongly advised. From \emph{LOWRDPRICE.} covariates gorup class it seems to be witnessing a missing underlining pattern NMAR which is clearer by looking at the co\_occurrence plot in figure \ref{fig:cooccurrence}. Co-occurrence analysis might suggest frequency of missing predictor in combination and \emph{LOWRDPRICE.} class covariates are displaying this type of behavior. \emph{PAUTO} is missing in the place where \emph{LOWRDPRICE.} class covariates are missing, but this is not happening for the opposite, leading to the conclusion that \emph{PAUTO} should be treated as a rare covariate MAR, therefore \emph{PAUTO} is dropped.
After some further investigation on \emph{LOWRDPRICE.}, the group class flags when the \emph{PRICE} covariate is effectively decreased and this is unusual. That is solved by grouping the covariate's information and to encode it as a two levels categorical covariate if lowered or not. Further methods to feature engineer the \emph{LOWRDPRICE.} class covariates can be with techniques typical of profile data, further references are on \citet{Kuhn}.

\begin{figure}
\centering
\includegraphics{07-exploratory_files/figure-latex/cooccurrence-1.pdf}
\caption{\label{fig:cooccurrence}Missingness co-occurrence plot}
\end{figure}

\hypertarget{covariates-imputation}{%
\subsection{Covariates Imputation}\label{covariates-imputation}}

A relatively simple approach to front missingness is to build a regression model to explain the covariates that have some missing and plug-back-in the respective estimates (e.g.~posterior means) from their predictive distributions \citet{Little}. This approach is fast and easy to implement in most of the cases, but it ignores the uncertainty behind the imputed values \citep{Bayesian_INLA_Rubio}. However it has the benefit to be a more than a reasonable choice with respect to the number of computation required, especially with INLA and in a spatial setting. That makes it the first choice method to follow since imputation regards also a small portion of data and predictors. At first it is considered the predictor \emph{condominium} for which some observation are missing. Indices are:

\begin{verbatim}
##  [1]  19  74  77  90  99 113 116 120 179 249
\end{verbatim}

A model is fitted based on missing data for which the response var is condominium and predictors are other important explanatory ones, i.e.\texttt{condom\ \textasciitilde{}\ 1\ +\ sqfeet\ +\ totlocali\ +\ floor\ +\ heating\ +\ ascensore}. In addition to the formula in the inla function a further specification has to be provided with the command \texttt{compute\ =\ TRUE} in the argument control.predictor. The command \texttt{compute} estimates the posterior means of the predictive distribution in the response variable for the missing points. The estimated posetior mean quantities are then imputeda are in table @red(tab:CondomImputation)

\begin{tabular}{lrr}
\toprule
  & mean & sd\\
\midrule
fitted.Predictor.019 & 198.11095 & 19.67085\\
fitted.Predictor.074 & 162.96544 & 13.29456\\
fitted.Predictor.077 & 99.38197 & 32.34108\\
fitted.Predictor.090 & 331.73519 & 33.05035\\
fitted.Predictor.099 & 170.54068 & 12.30267\\
\addlinespace
fitted.Predictor.113 & 196.61593 & 15.86545\\
fitted.Predictor.116 & 108.40482 & 20.79689\\
fitted.Predictor.120 & 162.86977 & 25.61622\\
fitted.Predictor.179 & 165.03632 & 20.53485\\
fitted.Predictor.249 & 117.24234 & 30.80290\\
\bottomrule
\end{tabular}

A further method for imputation has been designed by \emph{Gómez-Rubio, Cameletti, and Blangiardo 2019) miss lit} by adding a sub-model for the imputations to the final model through the inla function. This is directly handled inside the predictor formula adding a parameter in the latent field. However the approach makes the model more complex with a further layer of uncertainty to handle.
At first the additive regression model with all the covariates is called including the covariates with missing values. The response variable \emph{PRICE} displays no missing values and the model fitted is:

\hypertarget{model-specification}{%
\section{Model Specification}\label{model-specification}}

\hypertarget{mesh-building}{%
\section{Mesh building}\label{mesh-building}}

\emph{PARAFRASARE}
The SPDE approach approximates the continuous Gaussian field \(w_{i}\) as a discrete Gaussian Markov random field by means of a finite basis function defined on a triangulated mesh of the region of study. The spatial surface can be interpolated performing this approximation with the inla.mesh.2d() function of the R-INLA package. This function creates a Constrained Refined Delaunay Triangulation (CRDT) over the study region, that will be simply referred to as the mesh. Mesh should be intended as a trade off between the accuracy of the GMRF surface representation and the computational cost, in other words the more are the vertices, the finer is the GF approximation, leading to a computational funnel.

\begin{figure}
\centering
\includegraphics{images/triangle.jpg}
\caption{Traingularization intuition, \citet{Krainski-Rubio} source}
\end{figure}

Arguments can tune triangularization through inla.mesh.2d() :

\begin{itemize}
\tightlist
\item
  \texttt{loc}:location coordinates that are used as initial mesh vertices
\item
  \texttt{boundary}:object describing the boundary of the domain,
\item
  \texttt{offset}: argument is a numeric value (or a length two vector) and it is used
  to set the automatic extension distance. If positive, it is the extension distance
  in the same scale units. If negative, it is interpreted as a factor relative to the
  approximate data diameter; i.e., a value of -0.10 (the default) will add a 10\%
  of the data diameter as outer extension.
\item
  \texttt{cutoff}: points at a closer distance than the supplied value are replaced by a single vertex. Hence, it avoids small triangles
\item
  \texttt{max.edge}: A good mesh needs to have triangles as regular as possible in size and shape.
\item
  \texttt{min.angle}argument (which can be scalar or length two vector) can be used to specify the minimum internal angles of the triangles in the inner domain and the outer extension
\end{itemize}

A convex hull is a polygon of triangles out of the domain area, in other words the extension made to avoid the boundary effect. All meshes in Figure 2.12 have been made to have a convex hull boundary. If borders are available are generally preferred, so non convex hull meshes are avoided.

\hypertarget{shinyapp-for-mesh-assessment}{%
\subsection{Shinyapp for mesh assessment}\label{shinyapp-for-mesh-assessment}}

INLA includes a Shiny (Chang et al., 2018) application that can be used to tune the mesh params interactively

The mesh builder has a number of options to define the mesh on the left side. These include options to be passed to functions inla.nonconvex.hull() and inla.mesh.2d() and the resulting mesh displayed on the right part.

\hypertarget{building-spde-model-on-mesh}{%
\subsection{BUilding SPDE model on mesh}\label{building-spde-model-on-mesh}}

\hypertarget{spatial-kriging-prediction}{%
\section{Spatial Kriging (Prediction)}\label{spatial-kriging-prediction}}

QUI INCERTEZZE

\hypertarget{modelspec}{%
\chapter{Model Selection \& Fitting}\label{modelspec}}

\hypertarget{model-criticism}{%
\section{Model Criticism}\label{model-criticism}}

evaluation of the variables to include in the mode,, assumptions of the model i.e.~exchangeablity and independence prior distribution to assign to parameters and hyper parameters.

\hypertarget{spatial-kriging-1}{%
\section{Spatial Kriging}\label{spatial-kriging-1}}

\hypertarget{model-checking-1}{%
\section{Model Checking}\label{model-checking-1}}

if (models \textgreater{} 2)\{

\begin{verbatim}
## Model Selection




  IDEA: proporre due modelli uno più interepretabile con distribuzione normale, e un ò'altro con sempre exponential family ma con Gamma distribution function, ora vedo se riesco a fare tutto. dovrei sacrificare applicazione.
\end{verbatim}

\}

\hypertarget{application}{%
\chapter{Shiny Application}\label{application}}

with UI build with free tool for front end design ion shiny \href{https://fomantic-ui.com/}{fomantic-ui}. prendi shiny app e rifai interface. in questo blog vedi Hacaton tirato e vincitori \href{https://blog.rstudio.com/2020/11/10/the-appsilon-shiny-semantic-pocontest/}{blog}.

Senno app paula moraga che ha già simil modello dentro,

senno flexdashboard paula moraga.

\href{https://demo.appsilon.ai/apps/polluter/}{this inspiration}

\hypertarget{appendix}{%
\chapter*{Appendix}\label{appendix}}
\addcontentsline{toc}{chapter}{Appendix}

\hypertarget{gp-basics}{%
\section{GP basics}\label{gp-basics}}

\href{https://www.youtube.com/watch?v=4vGiHC35j9s\&t=164s}{Nando de Freitas 1}

\href{https://www.youtube.com/watch?v=MfHKW5z-OOA}{Nando de Freitas 2}

lets say there are a cloud of points represented by two variables x1 and x2. the cloud of points describes a realization of this two variable i.e.~height and weight and then you just plot it , you might get measurement like that,

\includegraphics{appendix_images/gp_base_1.jpg}
or:

each circle is a mesuduraments. now when we use multivariate gaussian we fit gaussian to data, the process of learning is to fit a gaussian to data, the ultimate goal is to describe the data, the smartest gaussian in the first image is to center the mean in the 0 and the draw a cricle containin all the other observation. Instaed for the second image it is still centering the mean in 0 but now it is an ellipse describing the variability, the size of the elipse descrube the variability of the data.
the center is a vector \(\mu_{i}\) that it is beacuse we have two components \(x_1\) and \(x_2\) whhose mean is 0 for each of the other. This is true for all the observation which have two coordinates too \(x_1\) and \(x_2\). in vector notations we have for the mean:

\[
\boldsymbol{\mu}=\left[\begin{array}{ll}
\mu_{x_1} \\
\mu_{x_2}
\end{array}\right]
\]

for each of the points, e.g.~for point 1:

\[
\mathbf{x_1}=\left[\begin{array}{ll}
x_1 \\
x_2
\end{array}\right]
\]

the can be neagtive positive, the Real numbers, usually we have \(\mathbb{R}^{2}\) extending from - infinity to + infinity, to the power of two because we have 2 dimensions, a Real plane.

any point is gaussian distributed when with mean .. an variance.
how we explain covariance, thorugh \emph{correlation.}
we do it by correlation with its noraml forms. the covariance is the term that goes insisde the matrices in the upper right of the matrxi we have the expectation of \(x_1\) times \(x_2\), like \(\mathbb{E}(x_1 \cdot x_2)\), where the extactation in the gaussian case is the mean which is 0, so the corresponding values is 0.
the covariance essentially is the dot product \href{https://mathinsight.org/dot_product_matrix_notation}{ref dot product} of \(x_1\) and \(x_2\) variable, so what happens when you take the dot product of vectors,
if for example you take a vector that looks like 1 and 0 and you take the dot product of one other vecto 1 and 0, so that:

\[
\left[\begin{array}{ll}
1 \\
0
\end{array}\right]\left[\begin{array}{ll}
1 & 0 \\
\end{array}\right] = 1
\]
You will end up with 1, recall dot productm first element first vcetor times first element second vectro and second element first vector times second element secon vector. So identical vector will get a high dot product value leading to a high similarity measure. Dot product can be indeded as a similarity measure.
\ldots{} But if you take two different vector as 1 0 and 0 1 then:

\[
\left[\begin{array}{ll}
1 \\
0
\end{array}\right]\left[\begin{array}{ll}
0 & 1 \\
\end{array}\right] = 0
\]

This time the multiplication leads to 0 value, as a matter of fact they are different. They are no similar.
IF two points are closed the dot product will be high in 2D. What the covariance should be? if variances are assumed to be 1 then in this case i qould expect to be 0, i.e.~covariance matrix is:

\[
\left[\begin{array}{ll}
1 & 0 \\
0 & 1
\end{array}\right] = \mathbf{cov_{plot1}}(x_1,x_2)
\]

because I can picka poin tin two pointa in this cloud. Suppose i increase x1 then my chance of getting a x2 point that is positive or negative is the samee, knowing somthin about x1 give nothign about x2. no information is proivede. On the other hand i the second plot knowing a positive value of 1 can suggest with a certain probability that x2 will be positive (great proabibility. So some information is provided), e.g.

\[
\left[\begin{array}{ll}
1   & 0.5 \\
0.5 & 1
\end{array}\right] = \mathbf{cov_{plot2}}(x_1,x_2)
\]

Some positive number idicates that i expect a positive inc rease iwhen boht of the two are increasing singularly. thsi is what the correlation, the basis to do linear regresssion and non linear- thei is a bivariate gaussian. If the entri3es are means that they are uncoorellated, if they are non-zero then they are correlated, theby can be both positive or negative (correlatiob)

now lets generate a gassiian distrivution so x\_1 and x\_2 in 2D and then a third dimension hwere we express probability, this is said joint distribution. So i am going to cu this gaussian at certain point for x\_1 and cut a plain rigght thgouth this gauissan imagine to ahava cake and then taka kkniw and cut it.(see the image)

form the man perspective you are goin to see a gaussian distribution, you will be lookong at x\_ and you will be seeing a gaussain plot in green. this is the probability of x\_1 gievn x\_1. also said ``conditioned'' probabolity. This gaussian has a mean like the one alreasdy seen and this is the center of the gaussian, we can rewrite the mena and variance of the multivariate gaussian describing the cloud of points. sigma are the covaraince martix sigma.

\ldots{}
sigma 1 and signa 2 if you have 1 d varibale the widjth has to be postive, for mulitvariate gaussian equl so here positive definitness: covariance mateix symetric.\\
\ldots{}
any artibitray variable transposed x time the covarince matrix nedds to be positive.
what is the mean of this gaussian i might want to know what is the widht of this gaussian would it be great if there is a formula that guven the cloud of point and likelihood estimation. we coilf obtain the red bell in figure.
Compute the green curve how it is done? this requires some work and it is said matrix \emph{invesrion lemma}, this is foudamental for machine learnign. let's assume it. The theorem says that the the mean fof the gaussian is the mean of x\_1 and then some other operation with sigma, see below from paci (miss ref)

\includegraphics{appendix_images/inverse_lemma.jpg}
the theorem says toi ocnsider a multivariate gaussian a vector 1 and a vector 2 each vecto compinent has a mena and a covarianc matrix, this by lemma gives us the expression and the math behind is no tremendous, but it is long. What it is important is to undestand fto go from a joont to a conditional distribution in our case. thats i the value od the theorem.

One background further thing: assume that we have a gaussian variable distribution that we want to sample fromm, we had now ewe are going to do the opposite, before we had poitns and we tried to figure out the curve, now we have the curve and we are gointg to try to rpoduce data. I need to be able to draw sample froma gaussian distribution. i will assume that i have a meachnism that produces a uniform samples, so you have a random numebr generatior with equal probabolity from 0 to 1, I assume a also the cumulative aof a gaussian.

the cumulative of a gaussian is what you get if you syrta summing the area under the curve of the gaussian as you move from the left. value after valure you can plot the cumulative ahead (see figure) the point where there is a flex point is the mean beacuse tha gassias is symmetric. The asymptot is 1 becuase the are under the curve sumes to 1.
If i can draw a random number form Uniform and the project it to thre cumulative and then finally projct it back to the gaussian distribution. Inverse cumulative mapping. If oyu do this multiple times you are going to have many sample palced next to the mean and as sparse as the variance. in this process of sampling try to sample a point i froma gaussian that has mean 0 a variance 1, now letes try to draw a point from a gaussian with mean mu and variance sigma. \ldots{}

In the multivariate case suppose that we have evctor with two variables how do i draw a vector from a multivariate gasussian with 0 means and plot 1 covarianc ematrix. the theormeem also says that the marginal distribution can be seen by civariance matrix , fist take the men\_1 and take upper left element from the covariance matrix obtaining the marginal rpobabiloty for x\_1, i.e.

\[
\pi(x_1) = \mathbf{N}(\mu_1, \Sigma_{11}) \\
\pi(x_2) = \mathbf{N}(\mu_2, \Sigma_{22}) 
\]
then in our problem:

\[
\pi(x_1) = \mathbf{N}(0, 1) \\
\pi(x_2) = \mathbf{N}(0, 1) 
\]
Then for simplicity we can simplyfy by groupign vector into:
(vectore exoression multivariate)

I need a wau to take square trotto of matrices, if x come sfroma MVG

35:01--

  \bibliography{refs.bib,Rpackages.bib}

\end{document}
